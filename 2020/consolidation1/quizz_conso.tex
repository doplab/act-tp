\documentclass[a4paper]{article}
\usepackage{times}
\usepackage[utf8]{inputenc}
\usepackage{selinput}
\usepackage{upquote}
\usepackage[margin=2cm, rmargin=4cm, tmargin=3cm]{geometry}
\usepackage{tcolorbox}
\usepackage{xspace}
\usepackage[french]{babel}
\usepackage{url}
\usepackage{hyperref}
\usepackage{fontawesome5}
\usepackage{marginnote}
\usepackage{ulem}
\usepackage{tcolorbox}
\usepackage{graphicx}
%\usepackage[top=Bcm, bottom=Hcm, outer=Ccm, inner=Acm, heightrounded, marginparwidth=Ecm, marginparsep=Dcm]{geometry}


\newtcolorbox{Example}[1]{colback=white,left=20pt,colframe=slideblue,fonttitle=\bfseries,title=#1}
\newtcolorbox{Solutions}[1]{colback=white,left=20pt,colframe=green,fonttitle=\bfseries,title=#1}
\newtcolorbox{Conseils}[1]{colback=white,left=20pt,colframe=slideblue,fonttitle=\bfseries,title=#1}
\newtcolorbox{Warning}[1]{colback=white,left=20pt,colframe=warning,fonttitle=\bfseries,title=#1}

\setlength\parindent{0pt}

  %Exercice environment
  \newcounter{exercice}
  \newenvironment{Exercice}[1][]
  {
  \par
  \stepcounter{exercice}\textbf{Question \arabic{exercice}:} (\faClock \enskip \textit{#1})
  }
  {\bigskip}
  

% Title
\newcommand{\titre}{\begin{center}
  \section*{Algorithmes et Pensée Computationnelle}
\end{center}}
\newcommand{\cours}[1]
{\begin{center} 
  \textit{#1}\\
\end{center}
  }


\newcommand{\exemple}[1]{\newline~\textbf{Exemple :} #1}
%\newcommand{\attention}[1]{\newline\faExclamationTriangle~\textbf{Attention :} #1}

% Documentation url (escape \# in the TP document)
\newcommand{\documentation}[1]{\faBookOpen~Documentation : \href{#1}{#1}}

% Clef API
\newcommand{\apikey}[1]{\faKey~Clé API : \lstinline{#1}}
\newcommand{\apiendpoint}[1]{\faGlobe~Url de base de l'API \href{#1}{#1}}

%Listing Python style
\usepackage{color}
\definecolor{slideblue}{RGB}{33,131,189}
\definecolor{green}{RGB}{0,190,100}
\definecolor{blue}{RGB}{121,142,213}
\definecolor{grey}{RGB}{120,120,120}
\definecolor{warning}{RGB}{235,186,1}

\usepackage{listings}
\lstdefinelanguage{texte}{
    keywordstyle=\color{black},
    numbers=none,
    frame=none,
    literate=
           {é}{{\'e}}1
           {è}{{\`e}}1
           {ê}{{\^e}}1
           {à}{{\`a}}1
           {â}{{\^a}}1
           {ù}{{\`u}}1
           {ü}{{\"u}}1
           {î}{{\^i}}1
           {ï}{{\"i}}1
           {ë}{{\"e}}1
           {Ç}{{\,C}}1
           {ç}{{\,c}}1,
    columns=fullflexible,keepspaces,
	breaklines=true,
	breakatwhitespace=true,
}
\lstset{
    language=Python,
	basicstyle=\bfseries\footnotesize,
	breaklines=true,
	breakatwhitespace=true,
	commentstyle=\color{grey},
	stringstyle=\color{slideblue},
  keywordstyle=\color{slideblue},
	morekeywords={with, as, True, False, Float, join, None, main, argparse, self, sort, __eq__, __add__, __ne__, __radd__, __del__, __ge__, __gt__, split, os, endswith, is_file, scandir, @classmethod},
	deletekeywords={id},
	showspaces=false,
	showstringspaces=false,
	columns=fullflexible,keepspaces,
	literate=
           {é}{{\'e}}1
           {è}{{\`e}}1
           {ê}{{\^e}}1
           {à}{{\`a}}1
           {â}{{\^a}}1
           {ù}{{\`u}}1
           {ü}{{\"u}}1
           {î}{{\^i}}1
           {ï}{{\"i}}1
           {ë}{{\"e}}1
           {Ç}{{\,C}}1
           {ç}{{\,c}}1,
    numbers=left,
}

\newtcbox{\mybox}{nobeforeafter,colframe=white,colback=slideblue,boxrule=0.5pt,arc=1.5pt, boxsep=0pt,left=2pt,right=2pt,top=2pt,bottom=2pt,tcbox raise base}
\newcommand{\projet}{\mybox{\textcolor{white}{\small projet}}\xspace}
\newcommand{\optionnel}{\mybox{\textcolor{white}{\small Optionnel}}\xspace}
\newcommand{\advanced}{\mybox{\textcolor{white}{\small Pour aller plus loin}}\xspace}
\newcommand{\auto}{\mybox{\textcolor{white}{\small Auto-évaluation}}\xspace}


\usepackage{environ}
\newif\ifShowSolution
\NewEnviron{solution}{
  \ifShowSolution
	\begin{Solutions}{\faTerminal \enskip Solution}
		\BODY
	\end{Solutions}
  \fi}


  \usepackage{environ}
  \newif\ifShowConseil
  \NewEnviron{conseil}{
    \ifShowConseil
    \begin{Conseils}{\faLightbulb \quad Conseil}
      \BODY
    \end{Conseils}

    \fi}

    \usepackage{environ}
  \newif\ifShowWarning
  \NewEnviron{attention}{
    \ifShowWarning
    \begin{Warning}{\faExclamationTriangle \quad Attention}
      \BODY
    \end{Warning}

    \fi}
  

%\newcommand{\Conseil}[1]{\ifShowIndice\ \newline\faLightbulb[regular]~#1\fi}


\usepackage{array}
\newcolumntype{C}[1]{>{\centering\let\newline\\\arraybackslash\hspace{0pt}}m{#1}}

\begin{document}
% Change the following values to true to show the solutions or/and the hints
\ShowSolutiontrue
\ShowConseiltrue
\titre

\section{Quizz général}

\subsection{Python}

\begin{Exercice}[2 minutes] Exercice 1\\
En python, 'Hello' équivaut à "Hello". 

\begin{enumerate}[label=\Alph*]
    \item - Vrai
    \item - Faux
\end{enumerate}
\begin{solution}
    \textbf{Vrai}: En python, les doubles guillemets et les guillemet sont équivalents. 
\end{solution}
\end{Exercice}


\begin{Exercice}[2 minutes] Exercice 1\\
À la fin d'une fonction, nous pouvons utiliser les commandes \lstinline{print()} ou \lstinline{return}, elles ont la même utilité.
\begin{enumerate}[label=\Alph*]
    \item - Vrai
    \item - Faux
\end{enumerate}
\begin{solution}
    \textbf{Faux}\\
    \lstinline{print()} permet uniquement d'afficher un message dans la console. Autrement dit, \lstinline{print()} sert à communiquer un message à l'utilisateur final du programme, celui-ci n'ayant pas accès au code.\\\\
    \lstinline{return} est une déclaration qui s'utilise à l'intérieur d'une fonction pour renvoyer le résultat de la fonction lorsqu'elle a été appellée. Exemple: la fonction \lstinline{len(L)} renvoie la longeur de la liste L.
\end{solution}
\end{Exercice}


\begin{Exercice}[2 minutes] Exercice 1\\
Lorsque l'on fait appel à une fonction, les arguements doivent nécessairement avoir le(s) même(s) noms tel(s) que définit dans la fonction. Exemple:\\
\begin{lstlisting}
def recherche_lineaire(Liste, x):
    for i in Liste:
        if i == x:
            return x in Liste
    return -1

Liste = [1,3,5,7,9]
x = 3

recherche_lineaire(Liste,x)

\end{lstlisting}
\begin{enumerate}[label=\Alph*]
    \item - Vrai
    \item - Faux
\end{enumerate}
\begin{solution}
    \textbf{Faux}\\
    Le noms des variables données en argument n'a aucune importance tant que le type de variable est respecté. Dans notre exemple, la fonction attend s'attend à une \textbf{liste} en premier argument et un \textbf{entier} en deuxième argument. Ici, nous aurions pu nommer la liste \lstinline{"nbr_impair"} et x \lstinline{"valeur"} et ainsi appelé la fonction \lstinline{recherche_lineaire(nbr_impair, valeur)}

\end{solution}
\end{Exercice}

\begin{Exercice}[2 minutes] Exercice 1\\
Si le programme python contient une erreur, celle-ci sera detectée avant l'exécution du programme. 

\begin{enumerate}[label=\Alph*]
    \item - Vrai
    \item - Faux
\end{enumerate}
\begin{solution}
    \textbf{Faux}: En python, les erreurs sont détectées pendant l'exécution du programme.
\end{solution}
\end{Exercice}


\begin{Exercice}[2 minutes] Exercice 1\\
Il est possible de faire appel à une fonction définie "plus bas" dans le code sans que cela ne pose problème.

\begin{lstlisting}
import math

nombre_decimal_pi(4)

def nombre_decimal_pi(int):
    return round(math.pi,int)
\end{lstlisting}

\begin{enumerate}[label=\Alph*]
    \item - Vrai
    \item - Faux
\end{enumerate}
\begin{solution}
    \textbf{Faux}: À l'excéption des fonctions intégrées (il s'agit des fonctions déjà intégrées au language python telles que \lstinline{print(), len(), abs(), etc}... une fonction doit nécessairement être définie \textbf{avant} d'être appellée.
\end{solution}
\end{Exercice}


\begin{Exercice}[5 minutes] Exercice 1\\
Les trois fonctions suivantes renvoient-elles systématiquement des résultats identiques ?\\Les fonctions sont censées retourner le nombre pi avec le nombre de décimales (au moins une et au maximum 15) indiqué en paramètre.
\begin{multicols}{3}
\begin{lstlisting}
import math

def nombre_decimal_pi(int):
    if int > 15:
        int = 15
    elif int < 0:
        int = 1
    resultat = round(math.pi,int) 
    return resultat

print(nombre_decimal_pi(-2))
print(nombre_decimal_pi(4))
print(nombre_decimal_pi(20))



\end{lstlisting}
\columnbreak

\begin{lstlisting}
import math

def nombre_decimal_pi(int):
    if int > 15:
        resultat = round(math.pi,15)
    elif int < 0:
        resultat = round(math.pi,1)
    else: 
        resultat = round(math.pi,int)
    return resultat

print(nombre_decimal_pi(-2))
print(nombre_decimal_pi(4))
print(nombre_decimal_pi(20))

\end{lstlisting}
\columnbreak

\begin{lstlisting}
import math

def nombre_decimal_pi(int):
    if int > 15:
        return round(math.pi,15)
    elif int < 0:
        return round(math.pi,1)
    else: 
        return round(math.pi,int)
    

print(nombre_decimal_pi(-2))
print(nombre_decimal_pi(4))
print(nombre_decimal_pi(20))

\end{lstlisting}
\end{multicols}

\begin{enumerate}[label=\Alph*]
    \item - Vrai
    \item - Faux
\end{enumerate}
\begin{solution}
    \textbf{Vrai}: Les trois fonctions produisent des résultats identiques. Si besoin, exécutez le code dans IntelliJ.
\end{solution}
\end{Exercice}



\subsection{Java}


\begin{Exercice}[3 minutes] Exercice 1\\
Observez les deux codes suivants en java. Lequel a-t-il la bonne structure et peut être compilé sans erreur ?

\begin{multicols}{2}
\begin{lstlisting}
//Code A
public class Main {

    public static void main(String[] args) {
        ma_function();
        autre_fonction();
    

        static void ma_function(){
            System.out.println("Voici ma fonction!");
        }
    
        static void autre_fonction(){
            System.out.println("Une autre fonction!");
        }
    }
}



\end{lstlisting}
\columnbreak

\begin{lstlisting}
//Code B
public class Main {

    public static void main(String[] args) {
    ma_function();
    autre_fonction();
    }

    static void ma_function(){
        System.out.println("Voici ma fonction!");
    }

    static void autre_fonction(){
        System.out.println("Une autre fonction!");
    }
}
\end{lstlisting}
\columnbreak

\end{multicols}
\begin{enumerate}[label=\Alph*]
    \item (à gauche)
    \item (à droite)
\end{enumerate}
\begin{solution}
    \textbf{Le code B}\\
    Le fichier dans son ensmble représente une \lstinline{class} java, ici la class s'appelle \textbf{Main}. À l'intérieur de cette class se trouve la fonction \lstinline{public static void main()}, il s'agit de fonction principale du programme, celle que l'on exécute et celle dans laquelle nous rédigeons notre code.\\
    Les autres fonctions, qui peuvent être appellées, se définissent au sein de la classe au même échelon que la fonction \lstinline{public static void main()} comme dans le Code B ci-dessus. 
\end{solution}
\end{Exercice}

\begin{Exercice}[2 minutes] Exercice 1\\
L'indentation des lignes de code en java est aussi importante qu'en python.
\begin{enumerate}[label=\Alph*]
    \item - Vrai
    \item - Faux
\end{enumerate}
\begin{solution}
    \textbf{Faux}
    \\En java, le compilateur ne prend pas en compte l'indentation pour interprété le programme, il comprend la structure à l'aide des parenthèses, des accolades et encore des point-virgule qui indique la fin d'une commande. Toutefois, l'indentation est un aspect important de la programmation car elle sert à bien structurer visuellement votre code.\\\\
    En python, l'indentation défini la structure de votre code. Elle est donc indispensable pour la bonne interprétation et la bonne exécution de votre programme. 
\end{solution}
\end{Exercice}

\end{document}