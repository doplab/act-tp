\documentclass[a4paper]{article}
\usepackage{times}
\usepackage[utf8]{inputenc}
\usepackage{selinput}
\usepackage{upquote}
\usepackage[margin=2cm, rmargin=4cm, tmargin=3cm]{geometry}
\usepackage{tcolorbox}
\usepackage{xspace}
\usepackage[french]{babel}
\usepackage{url}
\usepackage{hyperref}
\usepackage{fontawesome5}
\usepackage{marginnote}
\usepackage{ulem}
\usepackage{tcolorbox}
\usepackage{graphicx}
%\usepackage[top=Bcm, bottom=Hcm, outer=Ccm, inner=Acm, heightrounded, marginparwidth=Ecm, marginparsep=Dcm]{geometry}


\newtcolorbox{Example}[1]{colback=white,left=20pt,colframe=slideblue,fonttitle=\bfseries,title=#1}
\newtcolorbox{Solutions}[1]{colback=white,left=20pt,colframe=green,fonttitle=\bfseries,title=#1}
\newtcolorbox{Conseils}[1]{colback=white,left=20pt,colframe=slideblue,fonttitle=\bfseries,title=#1}
\newtcolorbox{Warning}[1]{colback=white,left=20pt,colframe=warning,fonttitle=\bfseries,title=#1}

\setlength\parindent{0pt}

  %Exercice environment
  \newcounter{exercice}
  \newenvironment{Exercice}[1][]
  {
  \par
  \stepcounter{exercice}\textbf{Question \arabic{exercice}:} (\faClock \enskip \textit{#1})
  }
  {\bigskip}
  

% Title
\newcommand{\titre}{\begin{center}
  \section*{Algorithmes et Pensée Computationnelle}
\end{center}}
\newcommand{\cours}[1]
{\begin{center} 
  \textit{#1}\\
\end{center}
  }


\newcommand{\exemple}[1]{\newline~\textbf{Exemple :} #1}
%\newcommand{\attention}[1]{\newline\faExclamationTriangle~\textbf{Attention :} #1}

% Documentation url (escape \# in the TP document)
\newcommand{\documentation}[1]{\faBookOpen~Documentation : \href{#1}{#1}}

% Clef API
\newcommand{\apikey}[1]{\faKey~Clé API : \lstinline{#1}}
\newcommand{\apiendpoint}[1]{\faGlobe~Url de base de l'API \href{#1}{#1}}

%Listing Python style
\usepackage{color}
\definecolor{slideblue}{RGB}{33,131,189}
\definecolor{green}{RGB}{0,190,100}
\definecolor{blue}{RGB}{121,142,213}
\definecolor{grey}{RGB}{120,120,120}
\definecolor{warning}{RGB}{235,186,1}

\usepackage{listings}
\lstdefinelanguage{texte}{
    keywordstyle=\color{black},
    numbers=none,
    frame=none,
    literate=
           {é}{{\'e}}1
           {è}{{\`e}}1
           {ê}{{\^e}}1
           {à}{{\`a}}1
           {â}{{\^a}}1
           {ù}{{\`u}}1
           {ü}{{\"u}}1
           {î}{{\^i}}1
           {ï}{{\"i}}1
           {ë}{{\"e}}1
           {Ç}{{\,C}}1
           {ç}{{\,c}}1,
    columns=fullflexible,keepspaces,
	breaklines=true,
	breakatwhitespace=true,
}
\lstset{
    language=Python,
	basicstyle=\bfseries\footnotesize,
	breaklines=true,
	breakatwhitespace=true,
	commentstyle=\color{grey},
	stringstyle=\color{slideblue},
  keywordstyle=\color{slideblue},
	morekeywords={with, as, True, False, Float, join, None, main, argparse, self, sort, __eq__, __add__, __ne__, __radd__, __del__, __ge__, __gt__, split, os, endswith, is_file, scandir, @classmethod},
	deletekeywords={id},
	showspaces=false,
	showstringspaces=false,
	columns=fullflexible,keepspaces,
	literate=
           {é}{{\'e}}1
           {è}{{\`e}}1
           {ê}{{\^e}}1
           {à}{{\`a}}1
           {â}{{\^a}}1
           {ù}{{\`u}}1
           {ü}{{\"u}}1
           {î}{{\^i}}1
           {ï}{{\"i}}1
           {ë}{{\"e}}1
           {Ç}{{\,C}}1
           {ç}{{\,c}}1,
    numbers=left,
}

\newtcbox{\mybox}{nobeforeafter,colframe=white,colback=slideblue,boxrule=0.5pt,arc=1.5pt, boxsep=0pt,left=2pt,right=2pt,top=2pt,bottom=2pt,tcbox raise base}
\newcommand{\projet}{\mybox{\textcolor{white}{\small projet}}\xspace}
\newcommand{\optionnel}{\mybox{\textcolor{white}{\small Optionnel}}\xspace}
\newcommand{\advanced}{\mybox{\textcolor{white}{\small Pour aller plus loin}}\xspace}
\newcommand{\auto}{\mybox{\textcolor{white}{\small Auto-évaluation}}\xspace}


\usepackage{environ}
\newif\ifShowSolution
\NewEnviron{solution}{
  \ifShowSolution
	\begin{Solutions}{\faTerminal \enskip Solution}
		\BODY
	\end{Solutions}
  \fi}


  \usepackage{environ}
  \newif\ifShowConseil
  \NewEnviron{conseil}{
    \ifShowConseil
    \begin{Conseils}{\faLightbulb \quad Conseil}
      \BODY
    \end{Conseils}

    \fi}

    \usepackage{environ}
  \newif\ifShowWarning
  \NewEnviron{attention}{
    \ifShowWarning
    \begin{Warning}{\faExclamationTriangle \quad Attention}
      \BODY
    \end{Warning}

    \fi}
  

%\newcommand{\Conseil}[1]{\ifShowIndice\ \newline\faLightbulb[regular]~#1\fi}


\usepackage{array}
\newcolumntype{C}[1]{>{\centering\let\newline\\\arraybackslash\hspace{0pt}}m{#1}}

\begin{document}
% Change the following values to true to show \documentclass[a4paper]{article}
\usepackage{times}
\usepackage[utf8]{inputenc}
\usepackage{selinput}
\usepackage{upquote}
\usepackage[margin=2cm, rmargin=4cm, tmargin=3cm]{geometry}
\usepackage{tcolorbox}
\usepackage{xspace}
\usepackage[french]{babel}
\usepackage{url}
\usepackage{hyperref}
\usepackage{fontawesome5}
\usepackage{marginnote}
\usepackage{ulem}
\usepackage{tcolorbox}
\usepackage{graphicx}
%\usepackage[top=Bcm, bottom=Hcm, outer=Ccm, inner=Acm, heightrounded, marginparwidth=Ecm, marginparsep=Dcm]{geometry}


\newtcolorbox{Example}[1]{colback=white,left=20pt,colframe=slideblue,fonttitle=\bfseries,title=#1}
\newtcolorbox{Solutions}[1]{colback=white,left=20pt,colframe=green,fonttitle=\bfseries,title=#1}
\newtcolorbox{Conseils}[1]{colback=white,left=20pt,colframe=slideblue,fonttitle=\bfseries,title=#1}
\newtcolorbox{Warning}[1]{colback=white,left=20pt,colframe=warning,fonttitle=\bfseries,title=#1}

\setlength\parindent{0pt}

  %Exercice environment
  \newcounter{exercice}
  \newenvironment{Exercice}[1][]
  {
  \par
  \stepcounter{exercice}\textbf{Question \arabic{exercice}:} (\faClock \enskip \textit{#1})
  }
  {\bigskip}
  

% Title
\newcommand{\titre}{\begin{center}
  \section*{Algorithmes et Pensée Computationnelle}
\end{center}}
\newcommand{\cours}[1]
{\begin{center} 
  \textit{#1}\\
\end{center}
  }


\newcommand{\exemple}[1]{\newline~\textbf{Exemple :} #1}
%\newcommand{\attention}[1]{\newline\faExclamationTriangle~\textbf{Attention :} #1}

% Documentation url (escape \# in the TP document)
\newcommand{\documentation}[1]{\faBookOpen~Documentation : \href{#1}{#1}}

% Clef API
\newcommand{\apikey}[1]{\faKey~Clé API : \lstinline{#1}}
\newcommand{\apiendpoint}[1]{\faGlobe~Url de base de l'API \href{#1}{#1}}

%Listing Python style
\usepackage{color}
\definecolor{slideblue}{RGB}{33,131,189}
\definecolor{green}{RGB}{0,190,100}
\definecolor{blue}{RGB}{121,142,213}
\definecolor{grey}{RGB}{120,120,120}
\definecolor{warning}{RGB}{235,186,1}

\usepackage{listings}
\lstdefinelanguage{texte}{
    keywordstyle=\color{black},
    numbers=none,
    frame=none,
    literate=
           {é}{{\'e}}1
           {è}{{\`e}}1
           {ê}{{\^e}}1
           {à}{{\`a}}1
           {â}{{\^a}}1
           {ù}{{\`u}}1
           {ü}{{\"u}}1
           {î}{{\^i}}1
           {ï}{{\"i}}1
           {ë}{{\"e}}1
           {Ç}{{\,C}}1
           {ç}{{\,c}}1,
    columns=fullflexible,keepspaces,
	breaklines=true,
	breakatwhitespace=true,
}
\lstset{
    language=Python,
	basicstyle=\bfseries\footnotesize,
	breaklines=true,
	breakatwhitespace=true,
	commentstyle=\color{grey},
	stringstyle=\color{slideblue},
  keywordstyle=\color{slideblue},
	morekeywords={with, as, True, False, Float, join, None, main, argparse, self, sort, __eq__, __add__, __ne__, __radd__, __del__, __ge__, __gt__, split, os, endswith, is_file, scandir, @classmethod},
	deletekeywords={id},
	showspaces=false,
	showstringspaces=false,
	columns=fullflexible,keepspaces,
	literate=
           {é}{{\'e}}1
           {è}{{\`e}}1
           {ê}{{\^e}}1
           {à}{{\`a}}1
           {â}{{\^a}}1
           {ù}{{\`u}}1
           {ü}{{\"u}}1
           {î}{{\^i}}1
           {ï}{{\"i}}1
           {ë}{{\"e}}1
           {Ç}{{\,C}}1
           {ç}{{\,c}}1,
    numbers=left,
}

\newtcbox{\mybox}{nobeforeafter,colframe=white,colback=slideblue,boxrule=0.5pt,arc=1.5pt, boxsep=0pt,left=2pt,right=2pt,top=2pt,bottom=2pt,tcbox raise base}
\newcommand{\projet}{\mybox{\textcolor{white}{\small projet}}\xspace}
\newcommand{\optionnel}{\mybox{\textcolor{white}{\small Optionnel}}\xspace}
\newcommand{\advanced}{\mybox{\textcolor{white}{\small Pour aller plus loin}}\xspace}
\newcommand{\auto}{\mybox{\textcolor{white}{\small Auto-évaluation}}\xspace}


\usepackage{environ}
\newif\ifShowSolution
\NewEnviron{solution}{
  \ifShowSolution
	\begin{Solutions}{\faTerminal \enskip Solution}
		\BODY
	\end{Solutions}
  \fi}


  \usepackage{environ}
  \newif\ifShowConseil
  \NewEnviron{conseil}{
    \ifShowConseil
    \begin{Conseils}{\faLightbulb \quad Conseil}
      \BODY
    \end{Conseils}

    \fi}

    \usepackage{environ}
  \newif\ifShowWarning
  \NewEnviron{attention}{
    \ifShowWarning
    \begin{Warning}{\faExclamationTriangle \quad Attention}
      \BODY
    \end{Warning}

    \fi}
  

%\newcommand{\Conseil}[1]{\ifShowIndice\ \newline\faLightbulb[regular]~#1\fi}



\ShowSolutiontrue
\ShowConseiltrue
\titre
\cours{Architecture des ordinateurs}

Le but de cette séance est de comprendre le fonctionnement d'un ordinateur. La série d'exercices sera axée autour de de conversions en base binaire, décimale ou hexadécimal, de calcul de base en suivant le modèle Von Neumann. \\

\section{Conversions}

\begin{Exercice}[10 minutes]  \textbf{Conversion }\\
    \begin{enumerate}
        \item Convertir le nombre FFFFFF$_{(16)}$ en base 10.
        \item Convertir le nombre 4321$_{(5)}$ en base 10.
        \item Convertir le nombre ABC$_{(16)}$ en base 2.
        \item Convertir le nombre 254$_{(10)}$ en base 15.
        \item Convertir le nombre 11101$_{(2)}$ en base 10.
    \end{enumerate}
    
        \begin{conseil}
            N'oubliez pas qu'en Hexadécimal, A vaut 10, B vaut 11, C vaut 12, D vaut 13, E vaut 14 et F vaut 15.
    \end{conseil}

    \begin{solution}
        1. FFFFFF$_{(16)}$ = 16777215$_{(10)}$\\
        2. 4321$_{(5)}$ = 586$_{(10)}$\\
        3. ABC$_{(16)}$ = 101010111100$_{(2)}$\\
        4. 254$_{(10)}$ = 11E$_{(15)}$\\
        5. 11101$_{(2)}$ = 29$_{(10)}$
    \end{solution}
\end{Exercice}

\begin{comment}
    % Les étudiants n'ont pas rencontré de difficultés sur des exercices similaires
\section{Arithmétique binaire}

\begin{Exercice}[5 minutes] \textbf{Addition et soustraction de nombres binaires}
    \begin{enumerate}
        \item Additionner \lstinline{10110101}$_{(2)}$ et \lstinline{00010101}$_{(2)}$
        \item Soustraire \lstinline{11000101}$_{(2)}$ et \lstinline{01000000}$_{(2)}$

    \end{enumerate}
    
     \begin{conseil}
        Cf: exercice 4,5 week 1
    \end{conseil}
    
  

    \begin{solution}
        1. {10110101}$_{(2)}$ + {00010101}$_{(2)}$ = {11001010}$_{(2)}$\\
        2. {11000101}$_{(2)}$ - {01000000}$_{(2)}$ = {10000101}$_{(2)}$
    \end{solution}
\end{Exercice}








\end{comment}
\section{Conversion et arithmétique}
\begin{Exercice}[5 minutes] \textbf{Conversion, addition et soustraction:}\\
    Effectuer les opérations suivantes:
    \begin{enumerate}
        \item 10110101$_{(2)}$ + 00101010$_{(2)}$ = ...$_{(10)}$
        \item 70$_{(10)}$ - 10101010$_{(2)}$ = ...$_{(10)}$
    \end{enumerate}
        \begin{conseil}
        Convertissez dans une base commune avant d'effectuer les opérations.
    \end{conseil}
        
    \begin{solution}
        1. 10110101$_{(2)}$ + 00101010$_{(2)}$ = 202$_{(10)}$\\
        3. 70$_{(10)}$ - 10101010$_{(2)}$ = 240$_{(10)}$
    \end{solution}
\end{Exercice}


\section{Modèle de Von Neuman}
Dans cette section, nous allons simuler une opération d'addition dans le \textbf{modèle de Van Neumann}, il va vous être demandé à chaque étape (FDES) de donner la valeur des registres.\\

\textbf{État d'origine:}\\
A l'origine, notre \lstinline{Process Counter (PC)} vaut \lstinline{00100001}.\\

Dans la mémoire, les instructions sont les suivantes:

\begin{tabular}{| C{0.1\textwidth} | C{0.1\textwidth} |} 
    \hline
    \textbf{Adresse} & \textbf{Valeur}\\ [0.5ex]
    \hline
    00100001 & 00110100\\ [0.5ex] 
    \hline
    00101100 & 10100110\\ [0.5ex] 
    \hline
    01110001 & 111111101\\ [0.5ex]
    \hline
\end{tabular}
\\\\
Les registres sont les suivants:

\begin{tabular}{| C{0.1\textwidth} | C{0.1\textwidth} |} 
    \hline
    \textbf{Registre} & \textbf{Valeur}\\ [0.5ex]
    \hline
    00 & 01111111\\ [0.5ex] 
    \hline
    01 & 00100000\\ [0.5ex] 
    \hline
    10 & 00101101\\ [0.5ex] 
    \hline
    11 & 00001100\\ [0.5ex]
    \hline
\end{tabular}
\\\\
Les opérations disponibles pour l'unité de contrôle sont les suivantes:
\\
\begin{tabular}{| C{0.1\textwidth} | C{0.1\textwidth} |} 
    \hline
    \textbf{Numéro} & \textbf{Valeur}\\ [0.5ex]
    \hline
    00 & ADD\\ [0.5ex] 
    \hline
    01 & XOR\\ [0.5ex] 
    \hline
    10 & MOV\\ [0.5ex] 
    \hline
    11 & SUB\\ [0.5ex]
    \hline
\end{tabular}
\\\\


\begin{Exercice}[5 minutes]\textbf{Fetch}\\
    À la fin de l'opération \lstinline{FETCH}, quelles sont les valeurs du \lstinline{Process Counter} et de l'\lstinline{Instruction Register}?
\end{Exercice}
   \begin{conseil}
    Pour rappel, l’unité de contrôle (Control Unit) commande et contrôle le fonctionnement du système. Elle est chargée du séquençage des opérations. Après chaque opération FETCH, la valeur du Program Counter est incrémentée (valeur initiale + 1).
    \end{conseil}
\begin{solution}
    PC = 00100001$_{(2)}$ + 1 = 00100010$_{(2)}$\\
    IR = 00110100$_{(2)}$
\end{solution}

\begin{Exercice}[5 minutes] \textbf{Decode}
    \begin{enumerate}
        \item Quelle est la valeur de l'opération à exécuter?
        \item Quelle est l'adresse du registre dans lequel le résultat doit être enregistré?
        \item Quelle est la valeur du premier nombre de l'opération?
        \item Quelle est la valeur du deuxième nombre de l'opération?
    \end{enumerate}
\end{Exercice}
   \begin{conseil}
    Pensez à décomposer la valeur de l’Instruction Register pour obtenir toutes les informations demandées.\\
    Utilisez la même convention que celle présentée dans les diapositives du cours (Architecture des ordinateurs (Semaine 2) - Diapositive 15)\\
    Les données issues de la décomposition de l’\lstinline{Instruction Register} ne sont pas des valeurs brutes, mais des références. Trouvez les tables concordantes pour y récupérer les valeurs.

    \end{conseil}
\begin{solution}
    00 : \lstinline{ADD} (valeur de l'opération à exécuter)\\
    11 : Adresse du registre dans lequel le résultat doit être enregistré\\
    01 : 00100000$_{(2)}$ (premier nombre)\\
    00 : 01111111$_{(2)}$ (deuxième nombre)
\end{solution}

\begin{Exercice}[5 minutes] \textbf{Execute}\\
    Quel est résultat de l'opération?
\end{Exercice}

   \begin{conseil}
        Toutes les informations permettant d’effectuer l’opération se trouvent dans les données de l’\lstinline{Instruction Register}.
    \end{conseil}

\begin{solution}
    00100000$_{(2)}$ + 01111111$_{(2)}$ = 10011111$_{(2)}$
\end{solution}

\section{Systèmes d'Exploitation}

\begin{Exercice}[5 minutes]
    Sous Linux et MacOS, laquelle de ces commandes modifie le \lstinline{filesystem}?
    \begin{enumerate}
        \item \lstinline{ls -la}
        \item \lstinline{sudo rm -rf ~/nano}
        \item \lstinline{sudo kill -9 3531}
        \item \lstinline{more nano.txt}
        \item Aucune réponse n'est correcte.
    \end{enumerate}
    \begin{solution}
        La commande \lstinline{sudo rm -rf ~/nano} permet de supprimer le répertoire \lstinline{nano} situé dans le dossier \lstinline{/Users/<Utilisateur\_courant>} en mode super-utilisateur (utilisateur ayant des droits étendus sur le système).
    \end{solution}
    \begin{conseil}
        \textbf{Attention!}Certaines commandes listées ci-dessus peuvent avoir des conséquences irréversibles.\\
        Pour avoir une description détaillée d'une commande, vous pouvez ajouter \lstinline{man} devant chaque commande sous Linux/MacOS ou ajouter \lstinline{-h, --help} ou \lstinline{/?} après chaque commande sous Windows.
    \end{conseil}
\end{Exercice}


\end{document}