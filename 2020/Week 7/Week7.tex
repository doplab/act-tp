\documentclass[a4paper]{article}
\usepackage{times}
\usepackage[utf8]{inputenc}
\usepackage{selinput}
\usepackage{upquote}
\usepackage[margin=2cm, rmargin=4cm, tmargin=3cm]{geometry}
\usepackage{tcolorbox}
\usepackage{xspace}
\usepackage[french]{babel}
\usepackage{url}
\usepackage{hyperref}
\usepackage{fontawesome5}
\usepackage{marginnote}
\usepackage{ulem}
\usepackage{tcolorbox}
\usepackage{graphicx}
\usepackage{verbatimbox}
\usepackage{amsmath}
\usepackage{hyperref}
%\usepackage[top=Bcm, bottom=Hcm, outer=Ccm, inner=Acm, heightrounded, marginparwidth=Ecm, marginparsep=Dcm]{geometry}


\newtcolorbox{Example}[1]{colback=white,left=20pt,colframe=slideblue,fonttitle=\bfseries,title=#1}
\newtcolorbox{Solutions}[1]{colback=white,left=20pt,colframe=green,fonttitle=\bfseries,title=#1}
\newtcolorbox{Conseils}[1]{colback=white,left=20pt,colframe=slideblue,fonttitle=\bfseries,title=#1}
\newtcolorbox{Warning}[1]{colback=white,left=20pt,colframe=warning,fonttitle=\bfseries,title=#1}

\setlength\parindent{0pt}

  %Exercice environment
  \newcounter{exercice}
  \newenvironment{Exercice}[1][]
  {
  \par
  \stepcounter{exercice}\textbf{Question \arabic{exercice}:} (\faClock \enskip \textit{#1})
  }
  {\bigskip}
  

% Title
\newcommand{\titre}{\begin{center}
  \section*{Algorithmes et Pensée Computationnelle}
\end{center}}
\newcommand{\cours}[1]
{\begin{center} 
  \textit{#1}\\
\end{center}
  }


\newcommand{\exemple}[1]{\newline~\textbf{Exemple :} #1}
%\newcommand{\attention}[1]{\newline\faExclamationTriangle~\textbf{Attention :} #1}

% Documentation url (escape \# in the TP document)
\newcommand{\documentation}[1]{\faBookOpen~Documentation : \href{#1}{#1}}

% Clef API
\newcommand{\apikey}[1]{\faKey~Clé API : \lstinline{#1}}
\newcommand{\apiendpoint}[1]{\faGlobe~Url de base de l'API \href{#1}{#1}}

%Listing Python style
\usepackage{color}
\definecolor{slideblue}{RGB}{33,131,189}
\definecolor{green}{RGB}{0,190,100}
\definecolor{blue}{RGB}{121,142,213}
\definecolor{grey}{RGB}{120,120,120}
\definecolor{warning}{RGB}{235,186,1}

\usepackage{listings}
\lstdefinelanguage{texte}{
    keywordstyle=\color{black},
    numbers=none,
    frame=none,
    literate=
           {é}{{\'e}}1
           {è}{{\`e}}1
           {ê}{{\^e}}1
           {à}{{\`a}}1
           {â}{{\^a}}1
           {ù}{{\`u}}1
           {ü}{{\"u}}1
           {î}{{\^i}}1
           {ï}{{\"i}}1
           {ë}{{\"e}}1
           {Ç}{{\,C}}1
           {ç}{{\,c}}1,
    columns=fullflexible,keepspaces,
	breaklines=true,
	breakatwhitespace=true,
}
\lstset{
    language=Python,
	basicstyle=\bfseries\footnotesize,
	breaklines=true,
	breakatwhitespace=true,
	commentstyle=\color{grey},
	stringstyle=\color{slideblue},
  keywordstyle=\color{slideblue},
	morekeywords={with, as, True, False, Float, join, None, main, argparse, self, sort, __eq__, __add__, __ne__, __radd__, __del__, __ge__, __gt__, split, os, endswith, is_file, scandir, @classmethod},
	deletekeywords={id},
	showspaces=false,
	showstringspaces=false,
	columns=fullflexible,keepspaces,
	literate=
           {é}{{\'e}}1
           {è}{{\`e}}1
           {ê}{{\^e}}1
           {à}{{\`a}}1
           {â}{{\^a}}1
           {ù}{{\`u}}1
           {ü}{{\"u}}1
           {î}{{\^i}}1
           {ï}{{\"i}}1
           {ë}{{\"e}}1
           {Ç}{{\,C}}1
           {ç}{{\,c}}1,
    numbers=left,
}

\newtcbox{\mybox}{nobeforeafter,colframe=white,colback=slideblue,boxrule=0.5pt,arc=1.5pt, boxsep=0pt,left=2pt,right=2pt,top=2pt,bottom=2pt,tcbox raise base}
\newcommand{\projet}{\mybox{\textcolor{white}{\small projet}}\xspace}
\newcommand{\optionnel}{\mybox{\textcolor{white}{\small Optionnel}}\xspace}
\newcommand{\auto}{\mybox{\textcolor{white}{\small Auto-évaluation}}\xspace}


\usepackage{environ}
\newif\ifShowSolution
\NewEnviron{solution}{
  \ifShowSolution
	\begin{Solutions}{\faTerminal \enskip Solution}
		\BODY
	\end{Solutions}
  \fi}


  \usepackage{environ}
  \newif\ifShowConseil
  \NewEnviron{conseil}{
    \ifShowConseil
    \begin{Conseils}{\faLightbulb \quad Conseil}
      \BODY
    \end{Conseils}

    \fi}

    \usepackage{environ}
  \newif\ifShowWarning
  \NewEnviron{attention}{
    \ifShowWarning
    \begin{Warning}{\faExclamationTriangle \quad Attention}
      \BODY
    \end{Warning}

    \fi}
  

%\newcommand{\Conseil}[1]{\ifShowIndice\ \newline\faLightbulb[regular]~#1\fi}



\usepackage{array}
\newcolumntype{C}[1]{>{\centering\let\newline\\\arraybackslash\hspace{0pt}}m{#1}}

\begin{document}

% Change the following values to true to show the solutions or/and the hints
\ShowSolutiontrue
\ShowConseiltrue
\titre
\cours{Graph Algorithms}

Le but de cette séance est de comprendre le fonctionnement des graphes et d'appliquer des algorithmes courant sur des graphes simples. \\

\section{Breadth-First Search}

\begin{Exercice}[5 minutes]  \textbf{Adjacency list et adjacency matrix : Python}\\
    L'adjacency list et l'adjacency matrix sont les 2 moyens que l'on considère pour représenter une graphe. Utilisez ces 2 méthodes de représentations pour stockez le graphe ci-dessous dans un programme python :\\
    \includegraphics[]{Graphe 1.PNG}   

    \begin{conseil}
    Pour l'adjacency matrix, utilisez un dictionnaire python. Pour l'adjacency matrix, utilisez une liste de liste.
    \end{conseil}
    \begin{solution}
        \lstinputlisting[language=python]{Question1_solution.py}
        Le fait que le graphe soit dirigé joue un rôle important pour la construction de ces représentations. Par exemple, pour l'adjacency list, 'B' apparaît dans la liste correspondant à la clé 'A' mais l'inverse n'est pas vrai. Cela siginfie que l'on peut aller du sommet A au sommet B mais pas du sommet B au sommet A.\\
    \end{solution}
\end{Exercice}


\begin{Exercice}[15 minutes]  \textbf{Breadth-First Search algorithm : Python}\\
    Nous allons maintenant nous intéresser au premier algorithme portant sur les graphes : \textbf{Breadth-First search}. Le but du Breadth-first search est de trouver tout les sommets atteignables à partir d'un sommet de départ.\\
    Implémentez l'algorithme en suivant les étapes suivante :\\
    \begin{enumerate}
    
    \item Partez du sommet initial, visitez les sommets adjascents, sauvegardez-les comme \textbf{visités}, insérez-les dans une \textbf{queue}.
    
    \item Parcourez la \textbf{queue}. Pour chaque éléments de la queue, visitez les sommets adjascents. Si ils ne sont pas dans la liste des somments visités, ajoutez-le à cette dernière et ajoutez le à la queue. Une fois que cela est fait, suprrimez l'élément parcouru de la queue.
    
    \item Répétez l'étape 2 jusqu'à ce que la queue soit vide.\\

    \end{enumerate}

    \begin{conseil}
        Quelques hints pour l'implémentation de votre algorithme :
        \begin{enumerate}
            \item Utilisez l'adjacency list.
            \item Pour le point 3), utilisez une boucle while avec la condition appropriée.
            \item Pour parcourir les sommets adjascents, utilisez une boucle for .
            \item L'algorithme devrait retourner une liste contenant l'ensemble des sommets atteignables.
            \item Vous pouvez utilisez l'image du graphe pour déterminer si l'output de votre lagorithme est correct.
        \end{enumerate}
    \end{conseil}
    \begin{solution}
        \lstinputlisting[language=python]{Question2_solution.py}
        Si vous avez correctement codé votre algorithme, l'output de celui-ci avec comme input notre graphe et le sommet 'A' devrait être ['A', 'B', 'C', 'F', 'D', 'E']
    \end{solution}
\end{Exercice}

\newpage

\section{Minimum Spanning tree}
    Nous allons maintenant nous intéresser à \textbf{l'algorithme de Kruskal}. Ce dernier s'applique uniquement aux \textbf{weighted graph}. Ces derniers sont des graphes où les arrêts ont des poids, représentant par exemple une distance. L'algorithme de Kruskal a pour but de trouver un \textbf{minimum spanning tree}. Un minimum spanning tree S de G est une sous-graphe connexe de G tel que :
    \begin{enumerate}
        \item V' = V, C'est à dire que tout les sommets de G sont aussi dans S
        \item (V',E') ne contient pas de cycle (pas de cycle dans S)
        \item S est le graphe satisfaisant 1) et 2) et ayant la plus petite somme des poids
    \end{enumerate}
    
    \textbf{\\}
    
\begin{Exercice}[5 minutes] \textbf{Algorithme de Kruskal : Papier}\\
    Appliquez l'algorithme de Kruskal au graphe suivant :\\
    \includegraphics[]{Kruskal.PNG}
    \begin{conseil}
        L'algorithme de Kruskal fonctionne de la façon suivante :
        \begin{enumerate}
            \item Classer les arrêtes par ordre croissant de poids.
            \item Prendre l'arrête avec le poids le plus faible et l'ajouter à l'arbre (si 2 arrêtes ont le même poids, choisir arbitrairement une des 2).
            \item Vérifiez que l'arrête ajoutée ne crée pas de cycle, si c'est le cas, supprimez la.
            \item Répétez les étapes 2) et 3) jusqu'à ce que tout les sommets soient atteints.
        \end{enumerate}
        Un minimum spanning tree, si il existe, a toujours un nombre d'arrêtes égale au nombre de sommets moins un. Par exemple, ici notre graphe a 6 sommets. L'algorithme devrait donc s'arrêter lorsque 5 arrêtes ont été choisies.
    \end{conseil}
    \begin{solution}
        Vous trouverez ci-dessous les étapes de la construction du MST(Minimum spanning tree) :\\
        \includegraphics[]{Week 7/K1.PNG}\\
        \includegraphics[]{Week 7/K2.PNG}\\
        \includegraphics[]{Week 7/K3.PNG}\\
    \end{solution}
    \begin{solution}
        \includegraphics[]{Week 7/K4.PNG}\\
        \includegraphics[]{Week 7/K5.PNG}\\
        L'algorithme s'arrête car tout les sommets ont été atteints. On voit bien que seule 5 arrêtes ont été nécessaire.
    \end{solution}
\end{Exercice}
\begin{Exercice}[15 minutes] \textbf{Algorithme de Kruskal : Python}\\
    Vous trouverez ci-dessous L'algorithme de Kruskal implémenté en python. Parcourez la fonction \textbf{kruskal\_algo(Graph)} afin de vous assurez que vous en avez bien compris le fonctionnement :\\
    \lstinputlisting[language=python]{Question3_solution.py}
    \begin{conseil}
        La partie class Graph correspond à de la programmation orientée objet. Vous ne devez pas la comprendre actuellement. L'output de l'algorithme est : \\
        1 - 2: 2\\
        2 - 5: 2\\
        2 - 3: 3\\
        3 - 4: 3\\
        0 - 1: 4\\
        Il se lit comme une liste d'arrête et de poids à l'arrête correspondante.
    \end{conseil}
\end{Exercice}

\newpage

\begin{Exercice}[10 minutes] \textbf{Social Network Analysis : Papier}\\
    Les graphes peuvent être utilisés pour représenter une multitude de choses.  L'une d'entre elles est la représentation de votre réseau d'ami. Imaginez que vous possédiez une liste de vos amis ainsi que des amis de vos amis (qui ne sont pas nécessairement vos amis). Cette liste peut-être représentée sous forme de graphe. Dans ce graphe :\\
    \begin{enumerate}
        \item  A quoi correspondent les arrêtes et les noeuds ? 
        \item Faut-il utiliser un graphe dirigé ?
    \end{enumerate}
    Supposez que vous disposiez d'un graphe des relations sociales. Décrivez comment retrouver les éléments suivants :
    \begin{enumerate}
        \item Votre ami qui a le plus d'ami
        \item Découvrir quels amis à vous se connaissent
        \item Listez vos amis qui pourraient vous présenter quelqu'un que vous ne connaissez pas (ami d'ami qui n'est pas votre ami)
    \end{enumerate}
    Vous voudriez désormais ajouter une nouvelle personne sur ce graphe, mais cette dernière n'est ni votre ami, ni l'ami d'un de vos amis. Quel sera son degré dans le graphe ?\\
    
    Retrouvez les éléments 1) à 3) dans le graphe ci-dessous :\\
    \includegraphics[]{Network1.PNG}
    \begin{conseil}
        Réfléchissez en terme d'arrêtes, de sommets, de degrés et de cycles.
    \end{conseil}

    \begin{solution}
        Dans le graphes des relations sociales, les arrêtes correspondent à un lien d'amitié et les noeuds représentent les personnes. Il n'est pas nécessaire d'utiliser un graphe dirigé si l'on considère qu'une relation d'amitié est toujours réciproque.\\
        
        Pour trouvez les éléments 1) à 3), il faut raisonner de la façon suivante :
        \begin{enumerate}
            \item Trouver le sommet relié à vous qui à le degré le plus élevé. Sommet B dans le graphe.
            \item Premièrement, les 2 personnes doivent être mes amis donc reliées à moi, mais de  plus elles doivent être reliés entre elles. Par conséquent, cela correspond à un cycle dans le graphe. Il y a autant d'amis qui se connaissent que de cycle dans le graphe. Ami A et B dans le graphe.
            \item Il ne doit pas y avoir d'arrête me reliant avec l'ami de mon ami. Sommet B dans le graphe.
        \end{enumerate}
        
        Si l'on ajoute un personne qui est ni un ami, ni l'ami d'un ami, alors aucune arrête n'est reliée avec ce sommet. Par conséquent, ce sommet a un degré 0.
    \end{solution} 
\end{Exercice}
\begin{Exercice}[5 minutes] \textbf{Reconnaître des réseaux semblables}\\
    Supposez maintenant que vous travaillez chez facebook, qui a récemment acheté Whatsapp et que vous disposiez des 2 graphes représentant respectivement Facebook et Whatsapp. Pouvez-vous déterminer à qui correspondent les gens de facebook sur whatsapp ?\\
    
    \includegraphics[]{Week 7/Network2.PNG}\\
    \includegraphics[]{Week 7/Network3.PNG}
    
    
    \begin{conseil}
    Quelques hints pour vous aider à résoudre cet exercice :
    \begin{enumerate}
        \item Identifiez les sommets des 2 graphes avec des caractéristiques semblables.
        \item Il est possible que les sommets ne soient pas tous identifiables.
    \end{enumerate}
        
    \end{conseil}
    \begin{solution}\\
        \begin{enumerate}
            \item B correspond à 2 (seul sommet de degré 4)
            \item A correspond à 1 (seul sommet de degré 1 relié au sommet de degré 4)
            \item C correspond à 3 (seul sommet de degré 3)
            \item E correspond à 5 (seul sommet de degré 1 relié au sommet de degré 2)
        \end{enumerate}
        Les sommets D et F et 4 et 6 ne peuvent pas être dissociés. On ne peut donc pas savoir qui correspond à qui.\\
    \end{solution}
\end{Exercice}

\newpage


\section{Algorithme de Dijkstra : Python}
Nous allons nous intéresser a l'algorithme de Dijkstra qui permet de calculer le chemin le plus court entre 2 sommets d'un graphe. Cet algorithme est par exemple utilisé par les systèmes GPS. Les questions de cet exercice sont à remplir sur le fichier Exercice3.py. \\
\begin{Exercice}[10 minutes]\textbf{Un petit échauffement}\\

Avant de rentrer dans le vif du sujet, nous allons devoir passer par quelques étapes préliminaires afin que vous puissiez codez l'algorithme par vous même. Considérez le graphe suivant :\\
\includegraphics[]{Week 7/Dijkstra.PNG}

Ouvrez le fichier Exercice3.py, et prenez connaissance du code, votre objectif sera de le compléter. Premièrement, représentez le graphe sous la forme d'une adjascency matrix.
\end{Exercice}
\begin{conseil}
    Représentez la matrice avec les colonnes et les lignes par ordre alphabétique.
\end{conseil}
\begin{solution}
    \lstinputlisting[language=pyton]{Question7_solution.py}
\end{solution}
\newpage
\begin{Exercice}[10 minutes]\textbf{Des outils utiles}\\
    La première étape étant complétée, nous allons maintenant nous intéresser à comment récupérer des informations concernant notre graphe. Voici une liste non-exhaustive des opérations que l'on peut effectuer :\\
    \begin{enumerate}
        \item get\_node() permet d'accéder à un noeud. Par exemple, en faisant graphe.get\_node('A'), j'obtiens des informations concernant le noeud A
        \item Lorsque l'on accède à un noeud par la fonction get\_node(), on peut ensuite accéder à la liste de toutes les arrêtes qui s'y connecte par l'attribut relationships. On peut par la suite distinguer les arrêts partant du noeud et les arrêtes y arrivant à l'aide des attributs relationship.\_from et relationship.to.
    \end{enumerate}
    
    L'exemple de code ci-dessous devrait vous aider à mieux comprendre :
    \lstinputlisting[language = python]{Question8_exemple.py}\\
    
    Pour vous assurez que vous avez bien compris cette partie avant de commencer, complétez la fonction linked du fichier Exercice3.py de sorte à ce que la fonction permette d'afficher tout les noeuds \textbf{partant} d'un sommet donné et d'afficher le poids de l'arrête reliant ces 2 sommets.\\
    
    
    \begin{conseil}
        Quelques hints pour écrire ce programme :
        \begin{enumerate}
            \item Vous devrez retirez les sommets reliés par une arrête avec un poids de 99999
            \item Utilisez une boucle for pour parcourir les arrêtes
        \end{enumerate}
    Votre output devrait être A 18, F 11.
    \end{conseil}
    \begin{solution}
        \lstinputlisting[language = python]{Question8_solution.py}
    \end{solution}
\end{Exercice}
\newpage
\begin{Exercice}[15 minutes] \textbf{Algorithme de Dijkstra (Difficile !)}\\
    L'algorithme de Dijkstra permet de calculer le chemin le plus court entre 2 sommets d'un graphe. L'algorithme de Dijkstra que l'on va utiliser se construit de façon récursive. On initialise l'algorithme en partant au point de départ. Puis l'on va se déplacer vers tout les sommets atteignable depuis notre point de départ et appliquer l'algorithme de dijkstra à ces voisins. Ainsi de suite jusqu'à ce que l'on atteigne le sommet de destination. Pour éviter de créer une boucle infinie à cause des cycles, nous appliquerons uniquement Dijkstra aux voisins qui n'ont pas encore été visité.\\
    
    Complétez la fonction dijkstra du fichier Exercice3.py.
    \begin{conseil}
        Quelques hint pour l'implémentation de l'algorithme :
        \begin{enumerate}
            \item Le chemin le plus court entre le sommet A et H devrait être ABCDFGH.
            \item Lorsque vous appliquerez Dijkstra aux voisins d'un sommet, pour choisir le chemin optimal vous devrez additionez la distance entre le sommet de départ et le poids du chemin fourni par Dijkstra.
            \item Attention, lorsque vous devez déterminer quels voisins sont atteignables, ceux dont le poids est de 99999 ne sont pas atteignables.
            \item L'algorithme doit retourner un tuble contenant la distance totale du trajet et le trajet.
        \end{enumerate}
    \end{conseil}
    \begin{solution}
        \lstinputlisting[language = python]{Question9_solution.py}
    \end{solution}
\end{Exercice}

\end{document}
