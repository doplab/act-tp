\documentclass[a4paper]{article}
\usepackage{times}
\usepackage[utf8]{inputenc}
\usepackage{selinput}
\usepackage{upquote}
\usepackage[margin=2cm, rmargin=4cm, tmargin=3cm]{geometry}
\usepackage{tcolorbox}
\usepackage{xspace}
\usepackage[french]{babel}
\usepackage{url}
\usepackage{hyperref}
\usepackage{fontawesome5}
\usepackage{marginnote}
\usepackage{ulem}
\usepackage{tcolorbox}
\usepackage{graphicx}
%\usepackage[top=Bcm, bottom=Hcm, outer=Ccm, inner=Acm, heightrounded, marginparwidth=Ecm, marginparsep=Dcm]{geometry}


\newtcolorbox{Example}[1]{colback=white,left=20pt,colframe=slideblue,fonttitle=\bfseries,title=#1}
\newtcolorbox{Solutions}[1]{colback=white,left=20pt,colframe=green,fonttitle=\bfseries,title=#1}
\newtcolorbox{Conseils}[1]{colback=white,left=20pt,colframe=slideblue,fonttitle=\bfseries,title=#1}
\newtcolorbox{Warning}[1]{colback=white,left=20pt,colframe=warning,fonttitle=\bfseries,title=#1}

\setlength\parindent{0pt}

  %Exercice environment
  \newcounter{exercice}
  \newenvironment{Exercice}[1][]
  {
  \par
  \stepcounter{exercice}\textbf{Question \arabic{exercice}:} (\faClock \enskip \textit{#1})
  }
  {\bigskip}
  

% Title
\newcommand{\titre}{\begin{center}
  \section*{Algorithmes et Pensée Computationnelle}
\end{center}}
\newcommand{\cours}[1]
{\begin{center} 
  \textit{#1}\\
\end{center}
  }


\newcommand{\exemple}[1]{\newline~\textbf{Exemple :} #1}
%\newcommand{\attention}[1]{\newline\faExclamationTriangle~\textbf{Attention :} #1}

% Documentation url (escape \# in the TP document)
\newcommand{\documentation}[1]{\faBookOpen~Documentation : \href{#1}{#1}}

% Clef API
\newcommand{\apikey}[1]{\faKey~Clé API : \lstinline{#1}}
\newcommand{\apiendpoint}[1]{\faGlobe~Url de base de l'API \href{#1}{#1}}

%Listing Python style
\usepackage{color}
\definecolor{slideblue}{RGB}{33,131,189}
\definecolor{green}{RGB}{0,190,100}
\definecolor{blue}{RGB}{121,142,213}
\definecolor{grey}{RGB}{120,120,120}
\definecolor{warning}{RGB}{235,186,1}

\usepackage{listings}
\lstdefinelanguage{texte}{
    keywordstyle=\color{black},
    numbers=none,
    frame=none,
    literate=
           {é}{{\'e}}1
           {è}{{\`e}}1
           {ê}{{\^e}}1
           {à}{{\`a}}1
           {â}{{\^a}}1
           {ù}{{\`u}}1
           {ü}{{\"u}}1
           {î}{{\^i}}1
           {ï}{{\"i}}1
           {ë}{{\"e}}1
           {Ç}{{\,C}}1
           {ç}{{\,c}}1,
    columns=fullflexible,keepspaces,
	breaklines=true,
	breakatwhitespace=true,
}
\lstset{
    language=Python,
	basicstyle=\bfseries\footnotesize,
	breaklines=true,
	breakatwhitespace=true,
	commentstyle=\color{grey},
	stringstyle=\color{slideblue},
  keywordstyle=\color{slideblue},
	morekeywords={with, as, True, False, Float, join, None, main, argparse, self, sort, __eq__, __add__, __ne__, __radd__, __del__, __ge__, __gt__, split, os, endswith, is_file, scandir, @classmethod},
	deletekeywords={id},
	showspaces=false,
	showstringspaces=false,
	columns=fullflexible,keepspaces,
	literate=
           {é}{{\'e}}1
           {è}{{\`e}}1
           {ê}{{\^e}}1
           {à}{{\`a}}1
           {â}{{\^a}}1
           {ù}{{\`u}}1
           {ü}{{\"u}}1
           {î}{{\^i}}1
           {ï}{{\"i}}1
           {ë}{{\"e}}1
           {Ç}{{\,C}}1
           {ç}{{\,c}}1,
    numbers=left,
}

\newtcbox{\mybox}{nobeforeafter,colframe=white,colback=slideblue,boxrule=0.5pt,arc=1.5pt, boxsep=0pt,left=2pt,right=2pt,top=2pt,bottom=2pt,tcbox raise base}
\newcommand{\projet}{\mybox{\textcolor{white}{\small projet}}\xspace}
\newcommand{\optionnel}{\mybox{\textcolor{white}{\small Optionnel}}\xspace}
\newcommand{\advanced}{\mybox{\textcolor{white}{\small Pour aller plus loin}}\xspace}
\newcommand{\auto}{\mybox{\textcolor{white}{\small Auto-évaluation}}\xspace}


\usepackage{environ}
\newif\ifShowSolution
\NewEnviron{solution}{
  \ifShowSolution
	\begin{Solutions}{\faTerminal \enskip Solution}
		\BODY
	\end{Solutions}
  \fi}


  \usepackage{environ}
  \newif\ifShowConseil
  \NewEnviron{conseil}{
    \ifShowConseil
    \begin{Conseils}{\faLightbulb \quad Conseil}
      \BODY
    \end{Conseils}

    \fi}

    \usepackage{environ}
  \newif\ifShowWarning
  \NewEnviron{attention}{
    \ifShowWarning
    \begin{Warning}{\faExclamationTriangle \quad Attention}
      \BODY
    \end{Warning}

    \fi}
  

%\newcommand{\Conseil}[1]{\ifShowIndice\ \newline\faLightbulb[regular]~#1\fi}



\usepackage{array}
\newcolumntype{C}[1]{>{\centering\let\newline\\\arraybackslash\hspace{0pt}}m{#1}}

\begin{document}

% Change the following values to true to show the solutions or/and the hints
\ShowSolutionfalse
\ShowConseiltrue
\titre
\cours{Semaine 6}
\cours{Algorithmes de recherche}


\section{Contenu des exercices}

\begin{enumerate}
    \item Recherche séquentielle
    \item Recherche binaire
    \item Arbres de recherche binaire\\
\end{enumerate}

Le but de cette séance est de se familiariser avec les algorithmes de recherches.\\

\section{Recherche séquentielle (ou recherche linéaire)}

Une recherche séquentielle scanne un élément après l'autre, de gauche à droite, pour trouver un élément x dans un tuple, une liste ou un dictionnaire.\\

Le but est de comparer x à chaque élément de la liste et de retourner l'index de l'élément correspondant.\\

Sa complexité dans le pire des cas est de O(n) et dans le meilleur des cas de O(1).\\

\begin{Exercice}[5 minutes] langage : \textbf{Python}\\

A partir des éléments ci-dessous, écrivez une fonction qui cherche \lstinline{x} dans la liste.\\La fonction doit retourner l'index de l'élément correspondant de la liste si \lstinline{x} est dans la liste et "-1" si \lstinline{x} n'est pas dans la liste (avec \lstinline{x = 100})\\

    \begin{lstlisting}
        def seq_search(list, x):
            #complètez ici
            
        liste = [3,55,6,8,3,5,56,33,6,5,3,2,99,53,532,75,21,963,100,445,56,45,12,56,24]
        x = 100
        
        seq_search(liste, x)
        
        %%time #Permet d'afficher le temps d'exécution de l'algorithme 
        print(seq_search(liste, x))
    \end{lstlisting}

Dans la pratique, l'algorithme de recherche séquentielle n'est pas souvent utilisé en vue de sa complexité élevée et des alternatives de recherche plus efficaces comme la recherche binaire.\\\\

    \begin{conseil}
        TODO: Créez votre fonction de recherche linéaire en utilisant une boucle \lstinline{for} ou une boucle \lstinline{while}.\\
        Aide à la compréhension de l'algorithme séquentielle : \href{https://runestone.academy/runestone/books/published/pythonds/SortSearch/TheSequentialSearch.html\#lst-seqsearchpython}{Cliquez ici}
    \end{conseil}
    
    \begin{solution}
        \lstinputlisting[language=python]{"resources/seq_search.py"} 
    \end{solution}
\textbf{Solution}
    \begin{lstlisting}
            def seq_search(list, x):
                #complètez ici
                for i in range(len(list)): # i est considéré comme un index
                    if list[i] == x:
                        print("X present in list at index: ", i)
                        return i
                        break
                else: #only used when loop not terminated by a break statement
                #if i == len(list)-1 and list[i] != x:  #also possible
                        print("X not present in list")
                        return -1
            
            liste = [3,55,6,8,3,5,56,33,6,5,3,2,99,53,532,75,21,963,100,445,56,45,12,56,24]
            x = 100
            seq_search(liste, x)
            
            %%time 
            print(seq_search(liste, x))
    \end{lstlisting}  
\end{Exercice}

\begin{Exercice}[5 minutes] language : \textbf{Python}\\

Considérez une liste d’entiers non triée \lstinline{L} ainsi qu’un entier \lstinline{e}. Écrivez un programme qui retourne l'élément de la liste \lstinline{L} le plus proche de \lstinline{e} en utilisant une recherche séquentielle.\\\\
Exemple :\\
L = [1, 2, 5, 8, 12, 16, 24, 56, 58, 63] \\
e = 41\\
Résultat attendu : 56\\

    \begin{lstlisting}
        def find_closest_seq(list,n): #calcul diff avec tout element et n
        diff = None #Initialisation de la variable
        resultat = None
        #complètez ici
            
        L = [1, 2, 5, 8, 12, 16, 24, 56, 58, 63] 
        e = 41

        find_closest_seq(L,e)
        
        %%time 
        print(find_closest_seq(L,e))
    \end{lstlisting}

    \begin{conseil}
        Aide à la compréhension de de l'algorithme séquentielle : \href{https://runestone.academy/runestone/books/published/pythonds/SortSearch/TheSequentialSearch.html\#lst-seqsearchpython}{Cliquez ici}
    \end{conseil}
    \begin{solution}
        Complétez la fonction \lstinline{find_closest_seq} et exécutez le code. 
    \end{solution}
\end{Exercice}

\begin{Exercice}[5 minutes] language : \textbf{Python}\\

    Considérez une liste d’entiers triés \lstinline{L} ainsi qu’un entier \lstinline{e}. Écrivez un programme qui retourne l'index de l'élément \lstinline{e} de la liste \lstinline{L} en utilisant une recherche séquentielle. Si \lstinline{e} n’est pas dans \lstinline{L}, retournez -1.\\\\
    
    Exemple:\\
    L = [1231321,3213125,3284016,4729273,5492710]\\
    e = 3284016\\
    Résultat attendu: 2\\
  
    \begin{lstlisting}
        def linear_search(L,e):
            for index, elem in enumerate(L):
            #complètez ici 
            
        L = [1231321,3213125,3284016,4729273,5492710]
        e = 3284016
        linear_search(L,e)
        
        %%time
        print(linear_search(L,e))
    \end{lstlisting}

    \begin{conseil}
        Aide à la compréhension de de l'algorithme séquentielle : \href{https://runestone.academy/runestone/books/published/pythonds/SortSearch/TheSequentialSearch.html\#lst-seqsearchpython}{Cliquez ici}
    \end{conseil}
    \begin{solution}
        Complétez la fonction \lstinline{linear_search} et exécutez le code. 
    \end{solution}

\end{Exercice}

\newpage
\section{Recherche binaire}
Le but de la recherche binaire est de trouver l'élément x plus rapidement. Pour cela, il est nécessaire d'utiliser \textbf{une liste d'éléments triée}.\\

La complexité de l'algorithme de recherche binaire est O(log n), très performante. Cependant, il ne faut pas oublier le coût lié à l'obtention d'une liste triée à partir d'une liste non triée.\\

L'algorithme de recherche binaire divise l'intervalle de recherche par deux à chaque coup jusqu'à ce qu'il trouve l'élément x ou que l'intervalle soit vide.\\

Ainsi, si x est plus petit que l'élément du milieu, l'algorithme va choisir la moitié de gauche comme intervalle de recherche et ainsi de suite. Si x est plus grand que l'élément du milieu la recherche va se faire dans la moitié droite de l'intervalle.\\
   
\begin{Exercice}[10-15 minutes] language : \textbf{Python}\\

A partir de la liste d'éléments \textbf{triée} ci-dessous, écrivez premièrement \textbf{une fonction récursive} puis \textbf{une fonction itérative} qui cherche x dans la liste. La fonction doit retourner l'index de l'élément correspondant de la liste si x est dans la liste et "-1" dans le cas contraire.\\
Ici, \lstinline{x = 5}.\\

    \begin{lstlisting}
        # Version récursive
        
        def rec_bin_search(list,s,r,x):
            #complètez ici 
        
        liste=[1,3,4,5,7,8,9,15]
        s = 0
        r = len(liste)
        x = 5
        rec_bin_search(liste,s, r, x)
     
        %%time
        print(rec_bin_search(liste,s, r, x))
    \end{lstlisting}


    \begin{conseil}
        TODO : Utilisez les variables à votre disposition et trouvez un moyen d'identifier le milieu de l'intervalle.\\\\
        Rappel : Une fonction récursive est une fonction qui fait appelle à elle-même.
    \end{conseil}
    
    
    \begin{lstlisting}
        # Version itérative
        
        def it_bin_search(list,s,r,x):
          #complètez ici
          
        liste = [1,3,4,5,7,8,9,15]
        s = 0
        r = len(liste)-1
        x = 7
        it_bin_search(liste,s, r, x)
        
        %%time
        print(it_bin_search(liste,s, r, x))
    \end{lstlisting}

    La recherche binaire regarde les comparaisons d'ordre, alors que la recherche séquentielle regarde les comparaisons d'égalité pour trouver x.\\

    \begin{conseil}
        TODO : Utilisez une boucle \lstinline{while}.\\\\
        Aide à la compréhension de l'algorithme de recherche binaire ( itérative) : \href{https://runestone.academy/runestone/books/published/pythonds/SortSearch/TheBinarySearch.html}{Cliquez ici}
    \end{conseil}
   
    \begin{solution}
        Complétez la fonction récursive \lstinline{rec_bin_search} et la fonction itérative \lstinline{it_bin_search}. 
    \end{solution}


\end{Exercice}


\begin{Exercice}[10 minutes] language : \textbf{Python}\\

Considérez une liste d’entiers non triés \lstinline{L} ainsi qu’un entier \lstinline{e}. Écrivez un programme qui retourne l'élément de la liste \lstinline{L} le plus proche de \lstinline{e} en utilisant une recherche binaire.\\

On vous donne une liste d’entiers non triés \lstinline{L} ainsi qu’un entier \lstinline{e}. Écrivez un programme retournant la valeur dans \lstinline{L} la plus proche de \lstinline{e} en utilisant une recherche binaire (binary search).\\

Résultat attendu : 56\\

\end{Exercice}



\newpage
\section{Arbre de recherche binaire}



\end{document}