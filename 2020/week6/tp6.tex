\documentclass[a4paper]{article}
\usepackage{times}
\usepackage[utf8]{inputenc}
\usepackage{selinput}
\usepackage{upquote}
\usepackage[margin=2cm, rmargin=4cm, tmargin=3cm]{geometry}
\usepackage{tcolorbox}
\usepackage{xspace}
\usepackage[french]{babel}
\usepackage{url}
\usepackage{hyperref}
\usepackage{fontawesome5}
\usepackage{marginnote}
\usepackage{ulem}
\usepackage{tcolorbox}
\usepackage{graphicx}
%\usepackage[top=Bcm, bottom=Hcm, outer=Ccm, inner=Acm, heightrounded, marginparwidth=Ecm, marginparsep=Dcm]{geometry}


\newtcolorbox{Example}[1]{colback=white,left=20pt,colframe=slideblue,fonttitle=\bfseries,title=#1}
\newtcolorbox{Solutions}[1]{colback=white,left=20pt,colframe=green,fonttitle=\bfseries,title=#1}
\newtcolorbox{Conseils}[1]{colback=white,left=20pt,colframe=slideblue,fonttitle=\bfseries,title=#1}
\newtcolorbox{Warning}[1]{colback=white,left=20pt,colframe=warning,fonttitle=\bfseries,title=#1}

\setlength\parindent{0pt}

  %Exercice environment
  \newcounter{exercice}
  \newenvironment{Exercice}[1][]
  {
  \par
  \stepcounter{exercice}\textbf{Question \arabic{exercice}:} (\faClock \enskip \textit{#1})
  }
  {\bigskip}
  

% Title
\newcommand{\titre}{\begin{center}
  \section*{Algorithmes et Pensée Computationnelle}
\end{center}}
\newcommand{\cours}[1]
{\begin{center} 
  \textit{#1}\\
\end{center}
  }


\newcommand{\exemple}[1]{\newline~\textbf{Exemple :} #1}
%\newcommand{\attention}[1]{\newline\faExclamationTriangle~\textbf{Attention :} #1}

% Documentation url (escape \# in the TP document)
\newcommand{\documentation}[1]{\faBookOpen~Documentation : \href{#1}{#1}}

% Clef API
\newcommand{\apikey}[1]{\faKey~Clé API : \lstinline{#1}}
\newcommand{\apiendpoint}[1]{\faGlobe~Url de base de l'API \href{#1}{#1}}

%Listing Python style
\usepackage{color}
\definecolor{slideblue}{RGB}{33,131,189}
\definecolor{green}{RGB}{0,190,100}
\definecolor{blue}{RGB}{121,142,213}
\definecolor{grey}{RGB}{120,120,120}
\definecolor{warning}{RGB}{235,186,1}

\usepackage{listings}
\lstdefinelanguage{texte}{
    keywordstyle=\color{black},
    numbers=none,
    frame=none,
    literate=
           {é}{{\'e}}1
           {è}{{\`e}}1
           {ê}{{\^e}}1
           {à}{{\`a}}1
           {â}{{\^a}}1
           {ù}{{\`u}}1
           {ü}{{\"u}}1
           {î}{{\^i}}1
           {ï}{{\"i}}1
           {ë}{{\"e}}1
           {Ç}{{\,C}}1
           {ç}{{\,c}}1,
    columns=fullflexible,keepspaces,
	breaklines=true,
	breakatwhitespace=true,
}
\lstset{
    language=Python,
	basicstyle=\bfseries\footnotesize,
	breaklines=true,
	breakatwhitespace=true,
	commentstyle=\color{grey},
	stringstyle=\color{slideblue},
  keywordstyle=\color{slideblue},
	morekeywords={with, as, True, False, Float, join, None, main, argparse, self, sort, __eq__, __add__, __ne__, __radd__, __del__, __ge__, __gt__, split, os, endswith, is_file, scandir, @classmethod},
	deletekeywords={id},
	showspaces=false,
	showstringspaces=false,
	columns=fullflexible,keepspaces,
	literate=
           {é}{{\'e}}1
           {è}{{\`e}}1
           {ê}{{\^e}}1
           {à}{{\`a}}1
           {â}{{\^a}}1
           {ù}{{\`u}}1
           {ü}{{\"u}}1
           {î}{{\^i}}1
           {ï}{{\"i}}1
           {ë}{{\"e}}1
           {Ç}{{\,C}}1
           {ç}{{\,c}}1,
    numbers=left,
}

\newtcbox{\mybox}{nobeforeafter,colframe=white,colback=slideblue,boxrule=0.5pt,arc=1.5pt, boxsep=0pt,left=2pt,right=2pt,top=2pt,bottom=2pt,tcbox raise base}
\newcommand{\projet}{\mybox{\textcolor{white}{\small projet}}\xspace}
\newcommand{\optionnel}{\mybox{\textcolor{white}{\small Optionnel}}\xspace}
\newcommand{\advanced}{\mybox{\textcolor{white}{\small Pour aller plus loin}}\xspace}
\newcommand{\auto}{\mybox{\textcolor{white}{\small Auto-évaluation}}\xspace}


\usepackage{environ}
\newif\ifShowSolution
\NewEnviron{solution}{
  \ifShowSolution
	\begin{Solutions}{\faTerminal \enskip Solution}
		\BODY
	\end{Solutions}
  \fi}


  \usepackage{environ}
  \newif\ifShowConseil
  \NewEnviron{conseil}{
    \ifShowConseil
    \begin{Conseils}{\faLightbulb \quad Conseil}
      \BODY
    \end{Conseils}

    \fi}

    \usepackage{environ}
  \newif\ifShowWarning
  \NewEnviron{attention}{
    \ifShowWarning
    \begin{Warning}{\faExclamationTriangle \quad Attention}
      \BODY
    \end{Warning}

    \fi}
  

%\newcommand{\Conseil}[1]{\ifShowIndice\ \newline\faLightbulb[regular]~#1\fi}



\usepackage{minted}
\usepackage{array}
\newcolumntype{C}[1]{>{\centering\let\newline\\\arraybackslash\hspace{0pt}}m{#1}}

\begin{document}

% Change the following values to true to show the solutions or/and the hints
\ShowSolutiontrue
\ShowConseiltrue
\titre
\cours{Semaine 6}
\cours{Algorithmes de recherche}



\begin{enumerate}
    \item Recherche séquentielle
    \item Recherche binaire
    \item Arbres de recherche binaire\\
\end{enumerate}

Le but de cette séance est de se familiariser avec les algorithmes de recherches.\\

\section{Recherche séquentielle (ou recherche linéaire)}

\subsection{Définition}

Une recherche \textbf{linéaire} (ou \textbf{séquentielle}) est une méthode permettant de trouver un élément dans une liste, un tableau ou un dictionnaire. Elle vérifie un à un chaque élément de gauche à droite jusqu'à ce qu'une correspondance soit trouvée ou que toute la liste ait été recherchée.\\

Si l'élément recherché est trouvé, l'algorithme \textbf{renvoie l'index}, c'est-à-dire la position, de l'élément dans la liste donnée. \\

Sa complexité dans le pire des cas est de \textbf{O(n)}, la longueur de la liste, et dans le meilleur des cas de \textbf{O(1)}, lorsque l'élément se trouve en première position. Dans la pratique, l'algorithme de recherche séquentielle n'est pas souvent utilisé en vue de sa complexité élevée et des alternatives de recherche plus efficaces comme la recherche binaire.\\\

\subsection{Exercices}

\begin{Exercice}[5 minutes] language : \textbf{Python}\\

A partir des éléments ci-dessous, écrivez une fonction qui cherche \lstinline{x} dans la liste.\\La fonction doit retourner l'index de l'élément correspondant de la liste si \lstinline{x} est dans la liste et "-1" si \lstinline{x} n'est pas dans la liste (avec \lstinline{x = 100})\\

\begin{minted}{python}
    def seq_search(list, x):
        #complètez ici
        
    liste = [3,55,6,8,3,5,56,33,6,5,3,2,99,53,532,75,21,963,100,445,56,56,24]
    x = 100
    
    seq_search(liste, x)
        
    %%time #Permet d'afficher le temps d'exécution de l'algorithme 
    print(seq_search(liste, x))
    
\end{minted}

\begin{conseil}
    Définissez votre fonction de recherche linéaire en utilisant une boucle \lstinline{for} ou une boucle \lstinline{while}.\\\\
    Attention: la fonction doit retourné l'index de la valeur et non pas la valeur. Pour cela, veillez à utiliser \lstinline{range(len(list))} avec la boucle \lstinline{for} et une incrémentation \lstinline{"i = i+1"} avec la boucle \lstinline{while}.\\\\
    La fonction \lstinline{print()} vous permet d'afficher l'index lorsqu'il a été trouvé et un autre message le cas échéant. 
\end{conseil}
    
    
\textbf{Solution question 1}

\begin{minted}{python}
    #Définition de la fonction
    def seq_search(list, x):
        for i in range(len(list)): #i représente l'index 
            if list[i] == x:
                print("X est présent dans la liste à l'index :", i)
                return i
                
            else: #Utilisez "else" s'il n'y a pas de "break"
                print("X n'est pas présent dans la liste")
                return -1
    
    #Déclaration de la liste et de la variable x 
    liste = [3,55,6,8,3,5,56,33,6,5,3,2,99,53,532,75,21,963,100,445,56,45,12,56,24]
    x = 100
    
    #Exécution de l'algorithme
    seq_search(liste, x)
    
    %%time #Permet d'afficher le temps de calcul
    print(seq_search(liste, x)
    
\end{minted}

\end{Exercice}

\begin{Exercice}[10 minutes] language : \textbf{Python}\\

Considérez une liste d’entiers non triée \lstinline{L} ainsi qu’un entier \lstinline{e}. Écrivez un programme qui retourne l'élément de la liste \lstinline{L} dont la valeur est la plus proche de \lstinline{e} en utilisant une recherche séquentielle.\\\\
Exemple :\\
L = [16, 2, 25, 8, 12, 31, 2, 56, 58, 63]\\
e = 50\\
Résultat attendu : 56\\

\begin{minted}{python}
    #Definition de la fonction ayant pour argument une liste et un nombre 
    def find_closest_seq(list,n):
        diff = None #Initialisation de la variable pour la différence
        resultat = None #Initialisation de la variable pour le résultat
    
        #Complètez ici
        
    #Déclaration de la liste et de la variable e
    L = [16, 2, 25, 8, 12, 31, 2, 56, 58, 63] 
    e = 50

    #Exécution de la fonction
    find_closest_seq(L,e)
    
    %%time 
    print(find_closest_seq(L,e))

\end{minted}

    \begin{conseil}
        Complétez la fonction \lstinline{find_closest_seq} et exécutez le code.\\\\
        Veillez à utiliser les valeurs absolues pour comparer les différences, la plus petite pouvant être positive ou négative. La fonction \lstinline{abs()} retourne la valeur absolue. Exemple : \lstinline{abs(3-10) retourne 7}.\\\\
        Étant donné que la liste est \textbf{non triée}, l'algorithme doit obligatoirement la parcourir intégralement.\\\\
        L'algorithme doit calculer la différence entre \lstinline{e} et chaque élément de la liste \lstinline{L} en gardant toujours la plus petite différence trouvée. À la fin, il retourne l'élément de la liste correspondant à la plus petite différence.  
    \end{conseil}
    
    \textbf{Solution question 2}
    
    \begin{minted}{python}
    
    #Definition de la fonction ayant pour argument une liste et un nombre 
    def find_closest_seq(list,n):
        diff = None #Initialisation de la variable pour la différence
        resultat = None #Initialisation de la variable pour le résultat
        
        #Solution
        for i in list:
            if not diff or abs(i-n) < diff: #old diff
                diff = abs(i-n) #new diff # 
                resultat = i
                
        return resultat, diff
    
    #Déclaration de la liste et de la variable e
    L = [16, 2, 25, 8, 12, 31, 2, 56, 58, 63]  
    e = 50
    
    #Exécution de la fonction
    find_closest_seq(L,e)
    
    %%time 
    print(find_closest_seq(L,e))
    
    \end{minted}
    
\end{Exercice}

\begin{Exercice}[5 minutes] language : \textbf{Python}\\

    Considérez une \textbf{liste d’entiers triés} \lstinline{L} ainsi qu’un entier \lstinline{e}. Écrivez un programme qui retourne l'index de l'élément \lstinline{e} de la liste \lstinline{L} en utilisant une recherche séquentielle. Si \lstinline{e} n’est pas dans \lstinline{L}, retournez -1.\\\\
    
    Exemple:\\
    L = [1231321,3213125,3284016,4729273,5492710]\\
    e = 3284016\\
    Résultat attendu: 2\\
  
\begin{minted}{python}
    def linear_search(L,e):
        for i in L: #Ici, i correspond à la valeur et non l'index.
        #complètez ici 
        
    L = [1231321,3213125,3284016,4729273,5492710]
    e = 3284016
    linear_search(L,e)
    
    %%time
    print(linear_search(L,e))
\end{minted}

    \begin{conseil}
        Une liste triée permet une recherche plus efficace à l'aide d'un algorithme plus simple.\\\\
        Veillez à retourner l'index de la valeur dans la liste. La fonction \lstinline{index()} retourne l'index d'un élément au sein d'une liste.\\Exemple et syntaxe: \lstinline{lst.index(i)} va indiqué la position de l'élément \lstinline{i} dans la liste \lstinline{lst}.
    \end{conseil}

    \textbf{Solution question 3}

    \begin{minted}{python}
    def linear_search(L,e):
        for i in L:
            #Solution
            if i == e:
                return L.index(i)
        return -1
        
    L = [1231321,3213125,3284016,4729273,5492710]
    e = 3284016
    linear_search(L,e)
    
    %%time
    print(linear_search(L,e))
    \end{minted}
\end{Exercice}

\newpage
\section{Recherche binaire}

\subsection{Définition}

Le but de la recherche binaire est de trouver l'élément \lstinline{x} plus rapidement. Pour cela, il est nécessaire d'utiliser \textbf{une liste d'éléments triée}.\\

La complexité de l'algorithme de recherche binaire est \lstinline{O(log n)}. Cependant, il ne faut pas oublier le coût lié à l'obtention d'une liste triée à partir d'une liste non triée.\\

L'algorithme de recherche binaire divise l'intervalle de recherche par deux à chaque coup jusqu'à ce qu'il trouve l'élément \lstinline{x} ou que l'intervalle soit vide.\\

Ainsi, si \lstinline{x} est plus petit que l'élément du milieu, l'algorithme va choisir la moitié de gauche comme intervalle de recherche et ainsi de suite. Si \lstinline{x} est plus grand que l'élément du milieu la recherche va se faire dans la moitié droite de l'intervalle.\\

La recherche binaire regarde les comparaisons d'ordre, alors que la recherche séquentielle regarde les comparaisons d'égalité pour trouver x.\\

\subsection{La récursivité}

Une fonction récursive est une fonction qui s'appelle elle-même pendant son exécution. Vous trouverez ci-dessous un exemple de fonction récursive utilisée pour le calcul factoriel.

\begin{minted}{python}
    def factoriel(n):
        if n == 1:
            return n
        else:
            return n * factoriel(n - 1)
            
    factorielle(4)
    #La fonction calcul 4 factoriel et retourne le résultat 
    #Rappel : 4! = 4 x 3 x 2 x 1 = 24
\end{minted}

Les fonctions récursives sont courantes en informatique car elles permettent aux programmeurs d'écrire des programmes efficaces en utilisant une quantité minimale de code. L'inconvénient est qu'elles peuvent provoquer des boucles infinies et d'autres résultats inattendus si elles ne sont pas écrites correctement. Si la fonction n'inclut pas les cas appropriés pour arrêter l'exécution, la récursivité se répétera à l'infini, provoquant le plantage du programme ou, pire encore, l'arrêt de tout le système informatique.\\
   
\subsection{Exercices}

\begin{Exercice}[10-15 minutes] language : \textbf{Python}\\

A partir de la liste d'éléments \textbf{triée} ci-dessous, écrivez premièrement \textbf{une fonction récursive} puis \textbf{une fonction itérative} qui cherche \lstinline{x} dans la liste. La fonction doit retourner \textbf{l'index} de l'élément correspondant de la liste si \lstinline{x} est dans la liste et \lstinline{-1} dans le cas contraire.\\
Ici, \lstinline{x = 5}.\\

\begin{minted}{python}
    #Version récursive
    
    def rec_bin_search(list,s,r,x):
        #complètez ici 
    
    liste=[1,3,4,5,7,8,9,15]
    s = 0
    r = len(liste)
    x = 5
    rec_bin_search(liste,s, r, x)
 
    %%time
    print(rec_bin_search(liste,s, r, x))



    #Version itérative
    
    def it_bin_search(list,s,r,x):
      #complètez ici
      
    liste = [1,3,4,5,7,8,9,15]
    s = 0
    r = len(liste)-1
    x = 7
    it_bin_search(liste,s, r, x)
    
    %%time
    print(it_bin_search(liste,s, r, x))
\end{minted}

    \begin{conseil}
        
        Détails sur les arguments de la fonction:\\
        \lstinline{list} : La liste dans laquelle nous effectuons la recherche\\
        \lstinline{s} : Le premier élément de la liste (index [0])\\
        \lstinline{r} : Le dernier élément de la liste (index [longueur de la liste -1])\\
        \lstinline{x} : La valeur recherchée.\\\\

        Ainsi, à chaque itération, vos fonction vont modifier les valeurs de base données en argument pour resserrer l'intervalle jusqu'à trouver la valeur recherchée.\\        
        Pour définir le milieu d'un intervalle qui contient un nombre pair ou impair d'éléments, utilisez la fonction \lstinline{int()}. Exemple:\\
        liste1 = [1,2,3,4,5]
        s = 0 \quad r = len(liste1)
        Calcul du milieu de l'intervalle:\\
        - \lstinline{(s+r)/2) = (0+5)/2 = 2.5 #Pas de correspondance}\\
        - \lstinline{int((s+r)/2) = int((0+5)/2) = int(2.5) = 2 #Arrondi vers le bas}\\\\
        
        Pour la version itérative, il est conseillé d'utiliser une boucle \lstinline{while}.

    \end{conseil}

    \textbf{Solution question 4}

    Complétez la fonction récursive \lstinline{rec_bin_search} et la fonction itérative \lstinline{it_bin_search}. 
    
    


\end{Exercice}


\begin{Exercice}[10 minutes] language : \textbf{Python}\\

Considérez une liste d’entiers non triés \lstinline{L} ainsi qu’un entier \lstinline{e}. Écrivez un programme qui retourne l'élément de la liste \lstinline{L} le plus proche de \lstinline{e} en utilisant une recherche binaire.\\

On vous donne une liste d’entiers non triés \lstinline{L} ainsi qu’un entier \lstinline{e}. Écrivez un programme retournant la valeur dans \lstinline{L} la plus proche de \lstinline{e} en utilisant une recherche binaire (binary search).\\

Résultat attendu : 56\\

\end{Exercice}



\newpage
\section{Arbre de recherche binaire}



\end{document}