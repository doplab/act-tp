\documentclass[a4paper]{article}
\usepackage{times}
\usepackage[utf8]{inputenc}
\usepackage{selinput}
\usepackage{upquote}
\usepackage[margin=2cm, rmargin=4cm, tmargin=3cm]{geometry}
\usepackage{tcolorbox}
\usepackage{xspace}
\usepackage[french]{babel}
\usepackage{url}
\usepackage{hyperref}
\usepackage{fontawesome5}
\usepackage{marginnote}
\usepackage{ulem}
\usepackage{tcolorbox}
\usepackage{graphicx}
%\usepackage[top=Bcm, bottom=Hcm, outer=Ccm, inner=Acm, heightrounded, marginparwidth=Ecm, marginparsep=Dcm]{geometry}


\newtcolorbox{Example}[1]{colback=white,left=20pt,colframe=slideblue,fonttitle=\bfseries,title=#1}
\newtcolorbox{Solutions}[1]{colback=white,left=20pt,colframe=green,fonttitle=\bfseries,title=#1}
\newtcolorbox{Conseils}[1]{colback=white,left=20pt,colframe=slideblue,fonttitle=\bfseries,title=#1}
\newtcolorbox{Warning}[1]{colback=white,left=20pt,colframe=warning,fonttitle=\bfseries,title=#1}

\setlength\parindent{0pt}

  %Exercice environment
  \newcounter{exercice}
  \newenvironment{Exercice}[1][]
  {
  \par
  \stepcounter{exercice}\textbf{Question \arabic{exercice}:} (\faClock \enskip \textit{#1})
  }
  {\bigskip}
  

% Title
\newcommand{\titre}{\begin{center}
  \section*{Algorithmes et Pensée Computationnelle}
\end{center}}
\newcommand{\cours}[1]
{\begin{center} 
  \textit{#1}\\
\end{center}
  }


\newcommand{\exemple}[1]{\newline~\textbf{Exemple :} #1}
%\newcommand{\attention}[1]{\newline\faExclamationTriangle~\textbf{Attention :} #1}

% Documentation url (escape \# in the TP document)
\newcommand{\documentation}[1]{\faBookOpen~Documentation : \href{#1}{#1}}

% Clef API
\newcommand{\apikey}[1]{\faKey~Clé API : \lstinline{#1}}
\newcommand{\apiendpoint}[1]{\faGlobe~Url de base de l'API \href{#1}{#1}}

%Listing Python style
\usepackage{color}
\definecolor{slideblue}{RGB}{33,131,189}
\definecolor{green}{RGB}{0,190,100}
\definecolor{blue}{RGB}{121,142,213}
\definecolor{grey}{RGB}{120,120,120}
\definecolor{warning}{RGB}{235,186,1}

\usepackage{listings}
\lstdefinelanguage{texte}{
    keywordstyle=\color{black},
    numbers=none,
    frame=none,
    literate=
           {é}{{\'e}}1
           {è}{{\`e}}1
           {ê}{{\^e}}1
           {à}{{\`a}}1
           {â}{{\^a}}1
           {ù}{{\`u}}1
           {ü}{{\"u}}1
           {î}{{\^i}}1
           {ï}{{\"i}}1
           {ë}{{\"e}}1
           {Ç}{{\,C}}1
           {ç}{{\,c}}1,
    columns=fullflexible,keepspaces,
	breaklines=true,
	breakatwhitespace=true,
}
\lstset{
    language=Python,
	basicstyle=\bfseries\footnotesize,
	breaklines=true,
	breakatwhitespace=true,
	commentstyle=\color{grey},
	stringstyle=\color{slideblue},
  keywordstyle=\color{slideblue},
	morekeywords={with, as, True, False, Float, join, None, main, argparse, self, sort, __eq__, __add__, __ne__, __radd__, __del__, __ge__, __gt__, split, os, endswith, is_file, scandir, @classmethod},
	deletekeywords={id},
	showspaces=false,
	showstringspaces=false,
	columns=fullflexible,keepspaces,
	literate=
           {é}{{\'e}}1
           {è}{{\`e}}1
           {ê}{{\^e}}1
           {à}{{\`a}}1
           {â}{{\^a}}1
           {ù}{{\`u}}1
           {ü}{{\"u}}1
           {î}{{\^i}}1
           {ï}{{\"i}}1
           {ë}{{\"e}}1
           {Ç}{{\,C}}1
           {ç}{{\,c}}1,
    numbers=left,
}

\newtcbox{\mybox}{nobeforeafter,colframe=white,colback=slideblue,boxrule=0.5pt,arc=1.5pt, boxsep=0pt,left=2pt,right=2pt,top=2pt,bottom=2pt,tcbox raise base}
\newcommand{\projet}{\mybox{\textcolor{white}{\small projet}}\xspace}
\newcommand{\optionnel}{\mybox{\textcolor{white}{\small Optionnel}}\xspace}
\newcommand{\advanced}{\mybox{\textcolor{white}{\small Pour aller plus loin}}\xspace}
\newcommand{\auto}{\mybox{\textcolor{white}{\small Auto-évaluation}}\xspace}


\usepackage{environ}
\newif\ifShowSolution
\NewEnviron{solution}{
  \ifShowSolution
	\begin{Solutions}{\faTerminal \enskip Solution}
		\BODY
	\end{Solutions}
  \fi}


  \usepackage{environ}
  \newif\ifShowConseil
  \NewEnviron{conseil}{
    \ifShowConseil
    \begin{Conseils}{\faLightbulb \quad Conseil}
      \BODY
    \end{Conseils}

    \fi}

    \usepackage{environ}
  \newif\ifShowWarning
  \NewEnviron{attention}{
    \ifShowWarning
    \begin{Warning}{\faExclamationTriangle \quad Attention}
      \BODY
    \end{Warning}

    \fi}
  

%\newcommand{\Conseil}[1]{\ifShowIndice\ \newline\faLightbulb[regular]~#1\fi}



\usepackage{array}
\newcolumntype{C}[1]{>{\centering\let\newline\\\arraybackslash\hspace{0pt}}m{#1}}

\begin{document}

% Change the following values to true to show the solutions or/and the hints
\ShowSolutiontrue
\ShowConseiltrue
\titre
\cours{Logiciels système}

Le but de cette séance est de comprendre le rôle d'un système d'exploitation, de logiciels système, d'interpréteurs, de compilateurs et des librairies.
Les exercices présentés au cours de cette séance se présentent sous deux sections. Une section composée de questions à choix multiples et une section permettant de découvrir et se familiariser avec votre futur environnement de travail.


\begin{section}{Systèmes d'exploitation}
    
Le système d'exploitation (OS) est le logiciel le plus important de l'ordinateur car il est responsable de tous les autres programmes et de leurs priorités d'exécution, il organise aussi les interactions et l'orchestration des différents composants physiques de la machine.
\\\\
L'OS contrôle l'accès au matériel physique et gère les processeurs, la mémoire, le stockage, la sécurité et les périphériques externes. Par exemple, l'OS identifie quel utilisateur est connecté (au moment d'entrer le mot de passe à l'allumage), il reconnaît quelles touches du clavier et de la souris sont pressées, il affiche les images à l'écran et il sauvegarde les fichiers dans les disques durs internes ou externes.
\\\\
La plupart du temps, de nombreuses applications sont exécutées au même moment sur le même ordinateur. Elles ont besoin d'utiliser les mêmes composants de l'ordinateur (CPU, RAM, Storage). Cependant, ces différents composants ne peuvent faire qu'une seule chose à la fois (même si ils les font très rapidement), c'est la raison pour laquelle l'OS orchestre chaque tâche de manière à ce que l'utilisateur ait l'impression que tout se déroule en même temps.
\\

    \begin{Exercice}[5 minutes]  \textbf{QCM}\\
    Lequel de ces logiciels n'est pas un système d'exploitation?
        \begin{enumerate}
            \item Microsoft Office 2012
            \item Microsoft Windows 98
            \item Unix BSD
            \item Gentoo
            \item Aucune réponse correcte.
        \end{enumerate}
    \end{Exercice}

    \begin{Exercice}[5 minutes]
        Quel est le rôle du noyau du système d'exploitation \textit{Kernel}? Trouvez la \textbf{mauvaise} réponse.
        \begin{enumerate}
            \item Servir d'interface de communication entre la partie logicielle et matérielle
            \item Gérer les ressources physiques de l'ordinateur
            \item Fournir les mécanismes d'abstraction du matériel
            \item Interpréter les instructions et les convertir en binaire
            \item Toutes les réponses sont correctes.
        \end{enumerate}
    \end{Exercice}

    \begin{Exercice}[5 minutes]
        Que représente la valeur \textbf{CPU Time} donnée par le gestionnaire de tâches de votre système d'exploitation?
        \begin{enumerate}
            \item Temps d'utilisation active de l'ensemble des cores du processeur.
            \item Temps réel d'utilisation d'une partie des ressources du processeur.
            \item Temps d'utilisation de l'ordinateur depuis le dernier démarrage.
            \item Temps maximum dédié à un processus particulier.
            \item Nombre de secondes écoulées depuis l'EPOCH.
        \end{enumerate}
        \begin{Example}{\faLightbulb \quad Informations utiles}
            \begin{itemize}
                \item L'EPOCH représente la date initiale à partir de laquelle est mesuré le temps par les systèmes d'exploitation.
                \item Pour accéder au gestionnaire de tâches sous Windows, ouvrez le menu démarrer > Saisissez ``Gestionnaire de tâches''. Sous Mac, faites ``command+espace'' et saisissez ``Moniteur d'activité'.
            \end{itemize}
            
        \end{Example}
        \begin{solution}
            Temps d'utilisation active de l'ensemble des cores du processeur.
            Exemple: Imaginez qu'un processus actif utilise constamment 10\% de la puissance du processeur pendant 20 minutes. Le temps compté et affiché dans le gestionnaire de tâches sera de 2 minutes.
        \end{solution}
    \end{Exercice}

    \begin{Exercice}[5 minutes]
        Que fait la commande suivante: \textbf{ls} \textit{(Linux/MacOS)} / \textbf{dir} \textit{(Windows)}?
        \begin{enumerate}
            \item Liste l'ensemble des fichiers du disque.
            \item Affiche sous forme de liste l'ensemble des processus en cours.
            \item Retourne uniquement les dossiers du répertoire courant.
            \item Affiche tous les dossiers et fichiers du répertoire courant.
            \item Aucune réponse n'est correcte.
        \end{enumerate}
        \begin{solution}
            \textbf{ls} \textit{(Linux/MacOS)} ou \textbf{dir} \textit{(Windows)} permet d'afficher le contenu du répertoire courant.
        \end{solution}
        \begin{conseil}
            Pour avoir une description détaillée d'une commande, vous pouvez ajouter \lstinline{man} devant chaque commande sous Linux/MacOS ou ajouter \lstinline{-h, --help} ou \lstinline{/?} après chaque commande sous Windows.
        \end{conseil}
    \end{Exercice}
\end{section}

\begin{section}{Prise en main de l'environnement de travail}
    \subsection{Installation des outils \textit{(\faClock \enskip 30 minutes)}} 
        Suivre le guide suivant pour installer les outils qui seront utilisés lors des prochaines séances de TP. //TODO: Lien vers le document prerequisite.
\end{section}

\begin{section}{Interpréteurs et compilateurs}
    Les ordinateurs ne ``comprennent" pas les langages de programmation, il faut passer par un programme qui va convertir le code écrit par un humain en instructions que l'ordinateur comprend (à base de 0 et de 1).
\\
Les interprètes et les compilateurs servent à faire ce travail de traduction, ils transforment donc les langages de programmation qui sont faits pour être compris et écrits par des humains, vers des instructions compréhensibles pour des ordinateurs.
\\
La façon dont les interprètes et les compilateurs opèrent est différente et les deux approches apportent leur propre lot de bénéfices et d'inconvénients:\\
    \begin{tabular}{| C{0.45\textwidth} | C{0.45\textwidth} |} 
        \hline
        \textbf{Compilateurs} & \textbf{Interpréteurs}\\ [0.5ex]
        \hline
        Les COMPILATEURS traduisent à l'avance & Les INTERPRÈTES traduisent au fur et à mesure\\ [0.5ex] 
        \hline
        Programmes générés à l'avance (besoin d'anticiper tous les cas de figure) & Programmes générés au fur et à mesure  \\ 
        \hline
        Le programme est intégralement traduit à l'avance & Les instructions sont traduites à la volée une par une  \\
        \hline
        Le compilateur trouve les erreurs à la compilation & L'interprète relève les erreurs pendant l'exécution du programme  \\
        \hline
        Plus rapide & Plus lent \\
        \hline
        Code machine généré et optimisé à l'avance & Code machine généré à la volée et optimisé à chaque exécution  \\
        \hline
        Ex: C, C++, Java, Scala, Rust, etc. & Ex: Bash, Python, Javascript, etc. \\
        \hline
    \end{tabular}
    \\\\
    \begin{Exercice}[10 minutes]
        En utilisant l'invite de commande (Terminal), créer un fichier contenant l'instruction suivante:
        \begin{lstlisting}
            if __name__ == "__main__":
                print("Hello world")\end{lstlisting}
        Enregistrer le fichier sous \lstinline{hello.py}. Utiliser l'interpréteur de Python pour exécuter le programme que vous venez de créer.
    \end{Exercice}
    \begin{conseil}
        \begin{itemize}
            \item Assurez-vous d'avoir correctement installer les outils de développement avant d'exécuter le programme.
            \item Utiliser un éditeur de texte présent sur votre ordinateur pour écrire directement votre programme depuis le terminal sous MacOS ou Linux. Nous vous conseillons d'utiliser Nano en tapant \lstinline{nano hello.py}.
            \item Sous Windows, vous pouvez utiliser n'importe quel éditeur de texte. ``Notepad'' est installé par défaut sous Windows.
        \end{itemize}
    \end{conseil}    

\end{section}

\begin{section}{Librairies}
    \textbf{Une librairie} est un ensemble de fonctions qui ont pour but d'être utilisées par d'autres programmes, mais une librairie seule ne suffit pas pour faire un programme. Par exemple, que ce soit Google Chrome, Word, ou Instagram, presque tous les programmes ont besoin d'utiliser des listes d'objets. Au lieu de réécrire l'ensemble des fonctions qui permettent de créer des listes et d'interragir avec, ces différents programmes utilisent tous une librairie écrite par un tiers.
\\\\
En Python, de nombreuses librairies sont disponibles de base (list, set, dict, str, etc.) et de nombreuses autres peuvent être utilisées en utilisant le mot-clé \lstinline{import}.
\\\\
Dans l'exemple suivant, nous importons la librairie math qui propose de nombreuses fonctions mathématiques et nous essayons la fonction factorielle pour calculer \lstinline{5!}.
\begin{Example}{\faTerminal Exemple}
    \begin{lstlisting}
    import math

    fact = math.factorial(5)
    print(fact)\end{lstlisting}
\end{Example}

Voici la documentation de la Librairie math en python : https://docs.python.org/3/library/math.html \\

Quand vous ne comprenez pas comment fonctionnent certaines fonctions issues d'une librairie, lire la documentation (la doc) est toujours une des premières chose à faire. Toutes les fonctions disponibles dans la librairie en question sont dévelopées. Trouvez la fonction en question, et vous trouverez son utilité, les paramètres à lui passer et ce qu'elle va retourner. Si vous cherchez à faire quelquechose qui a certainement déjà été implémenté par quelqu'un, pensez à regarder s'il n'existe pas une librairie contenant une fonction déjà existante au lieu de la réimplémenter. \\

\begin{conseil}
	https://docs.python.org/3/library/math.html
\end{conseil} 


\begin{Exercice}[5 minutes]
	A quoi sert la fonction gcd(a,b) ? 
\end{Exercice} \\

\begin{solution}
	Elle sert à calculer le plut grand diviseur commun entre a et b.
\end{solution} 

\begin{Exercice}[5 minutes]
	Comment faire si vous voulez convertir un angle de radians à degrès ? 
\end{Exercice} \\

\begin{solution}
	Il faut utiliser la fonction degrees(x)
\end{solution} 

\begin{Exercice}[5 minutes]
	A quoi sert la fonction fsum() et quels sont les paramètres à lui donner ? 
\end{Exercice} \\

\begin{solution}
	Elle sert à aditionner les éléments d'une liste de façon plus précise que sum().
\end{solution} 

\begin{Exercice}[5 minutes]
	Comment faire si je veux calculer la distance euclidienne entre 2 points ? 
\end{Exercice} \\

\begin{solution}
	Il faut utiliser la fonction dist(p,q). Attention, ici p et q sont les deux des listes de coordonnées !
\end{solution} 

\begin{Exercice}[5 minutes]
	Est-ce que la librairie math contient autre chose que des fonctions ? Si oui, quoi ? 
\end{Exercice} \\

\begin{solution}
	Non elle contient également des constantes, on peut citer pi par exemple.
\end{solution} 

\begin{Exercice}[10 minutes]
	Cherchez sur le net d'autres libraries disponibles pour python. Trouvez en 3 et lisez leur documentation en diagonale pour vous faire une idée de leur utilité.
\end{Exercice}    

\begin{solution}
	On peut citer Numpy, Scikit, TensorFlow qui sont utilisées pour faire du machine-learning. \\
	
	On peut aussi citer Matplotlib qui permet de créer des graphiques, Pandas qui permet de faire de l'analyse de données,SQLAlchemy qui permet de gérer des bases de données. \\
\end{solution} 
\end{section}




\end{document}
