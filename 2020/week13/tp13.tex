\documentclass[a4paper]{article}
\usepackage{times}
\usepackage[utf8]{inputenc}
\usepackage{selinput}
\usepackage{upquote}
\usepackage[margin=2cm, rmargin=4cm, tmargin=3cm]{geometry}
\usepackage{tcolorbox}
\usepackage{xspace}
\usepackage[french]{babel}
\usepackage{url}
\usepackage{hyperref}
\usepackage{fontawesome5}
\usepackage{marginnote}
\usepackage{ulem}
\usepackage{tcolorbox}
\usepackage{graphicx}
%\usepackage[top=Bcm, bottom=Hcm, outer=Ccm, inner=Acm, heightrounded, marginparwidth=Ecm, marginparsep=Dcm]{geometry}


\newtcolorbox{Example}[1]{colback=white,left=20pt,colframe=slideblue,fonttitle=\bfseries,title=#1}
\newtcolorbox{Solutions}[1]{colback=white,left=20pt,colframe=green,fonttitle=\bfseries,title=#1}
\newtcolorbox{Conseils}[1]{colback=white,left=20pt,colframe=slideblue,fonttitle=\bfseries,title=#1}
\newtcolorbox{Warning}[1]{colback=white,left=20pt,colframe=warning,fonttitle=\bfseries,title=#1}

\setlength\parindent{0pt}

  %Exercice environment
  \newcounter{exercice}
  \newenvironment{Exercice}[1][]
  {
  \par
  \stepcounter{exercice}\textbf{Question \arabic{exercice}:} (\faClock \enskip \textit{#1})
  }
  {\bigskip}
  

% Title
\newcommand{\titre}{\begin{center}
  \section*{Algorithmes et Pensée Computationnelle}
\end{center}}
\newcommand{\cours}[1]
{\begin{center} 
  \textit{#1}\\
\end{center}
  }


\newcommand{\exemple}[1]{\newline~\textbf{Exemple :} #1}
%\newcommand{\attention}[1]{\newline\faExclamationTriangle~\textbf{Attention :} #1}

% Documentation url (escape \# in the TP document)
\newcommand{\documentation}[1]{\faBookOpen~Documentation : \href{#1}{#1}}

% Clef API
\newcommand{\apikey}[1]{\faKey~Clé API : \lstinline{#1}}
\newcommand{\apiendpoint}[1]{\faGlobe~Url de base de l'API \href{#1}{#1}}

%Listing Python style
\usepackage{color}
\definecolor{slideblue}{RGB}{33,131,189}
\definecolor{green}{RGB}{0,190,100}
\definecolor{blue}{RGB}{121,142,213}
\definecolor{grey}{RGB}{120,120,120}
\definecolor{warning}{RGB}{235,186,1}

\usepackage{listings}
\lstdefinelanguage{texte}{
    keywordstyle=\color{black},
    numbers=none,
    frame=none,
    literate=
           {é}{{\'e}}1
           {è}{{\`e}}1
           {ê}{{\^e}}1
           {à}{{\`a}}1
           {â}{{\^a}}1
           {ù}{{\`u}}1
           {ü}{{\"u}}1
           {î}{{\^i}}1
           {ï}{{\"i}}1
           {ë}{{\"e}}1
           {Ç}{{\,C}}1
           {ç}{{\,c}}1,
    columns=fullflexible,keepspaces,
	breaklines=true,
	breakatwhitespace=true,
}
\lstset{
    language=Python,
	basicstyle=\bfseries\footnotesize,
	breaklines=true,
	breakatwhitespace=true,
	commentstyle=\color{grey},
	stringstyle=\color{slideblue},
  keywordstyle=\color{slideblue},
	morekeywords={with, as, True, False, Float, join, None, main, argparse, self, sort, __eq__, __add__, __ne__, __radd__, __del__, __ge__, __gt__, split, os, endswith, is_file, scandir, @classmethod},
	deletekeywords={id},
	showspaces=false,
	showstringspaces=false,
	columns=fullflexible,keepspaces,
	literate=
           {é}{{\'e}}1
           {è}{{\`e}}1
           {ê}{{\^e}}1
           {à}{{\`a}}1
           {â}{{\^a}}1
           {ù}{{\`u}}1
           {ü}{{\"u}}1
           {î}{{\^i}}1
           {ï}{{\"i}}1
           {ë}{{\"e}}1
           {Ç}{{\,C}}1
           {ç}{{\,c}}1,
    numbers=left,
}

\newtcbox{\mybox}{nobeforeafter,colframe=white,colback=slideblue,boxrule=0.5pt,arc=1.5pt, boxsep=0pt,left=2pt,right=2pt,top=2pt,bottom=2pt,tcbox raise base}
\newcommand{\projet}{\mybox{\textcolor{white}{\small projet}}\xspace}
\newcommand{\optionnel}{\mybox{\textcolor{white}{\small Optionnel}}\xspace}
\newcommand{\advanced}{\mybox{\textcolor{white}{\small Pour aller plus loin}}\xspace}
\newcommand{\auto}{\mybox{\textcolor{white}{\small Auto-évaluation}}\xspace}


\usepackage{environ}
\newif\ifShowSolution
\NewEnviron{solution}{
  \ifShowSolution
	\begin{Solutions}{\faTerminal \enskip Solution}
		\BODY
	\end{Solutions}
  \fi}


  \usepackage{environ}
  \newif\ifShowConseil
  \NewEnviron{conseil}{
    \ifShowConseil
    \begin{Conseils}{\faLightbulb \quad Conseil}
      \BODY
    \end{Conseils}

    \fi}

    \usepackage{environ}
  \newif\ifShowWarning
  \NewEnviron{attention}{
    \ifShowWarning
    \begin{Warning}{\faExclamationTriangle \quad Attention}
      \BODY
    \end{Warning}

    \fi}
  

%\newcommand{\Conseil}[1]{\ifShowIndice\ \newline\faLightbulb[regular]~#1\fi}


\usepackage{array}
\newcolumntype{C}[1]{>{\centering\let\newline\\\arraybackslash\hspace{0pt}}m{#1}}

\begin{document}
% Change the following values to true to show the solutions or/and the hints
\ShowSolutiontrue
\ShowConseiltrue
\titre
\cours{Programmation orientée objet: Héritage et Polymorphisme}
% TODO: Objectifs à atteindre (Alpha)
% TODO (All): Utiliser \lstinline{} sur tous les mots-clés, noms de classes, d'attributs, méthodes,...

Le code présenté dans les énoncés se trouve sur Moodle, dans le dossier \lstinline{Ressources}.
\section{Rappel: Surcharge des opérateurs - Python}
% Python
% TODO: Alpha 
% Utiliser une classe Fraction et effectuer des opérations de base sur des Fractions.
Dans cette section, vous manipulerez des fractions sous forme d'objets. Vous ferez des opérations de base sur ce nouveau type d'objets.

\begin{Exercice}[5 minutes]
    Dans un projet que vous aurez au préalable préparé, créez un fichier appelé \lstinline{surcharge.py}.% TODO: Nom fichier à définir.
    À l'intérieur de ce fichier, créer une classe \lstinline{Fraction} qui aura comme attributs un numérateur et un dénominateur. 
\end{Exercice}

\begin{Exercice}[5 minutes]
Définir un constructeur à votre classe. Assignez des valeurs par défaut à vos attributs.

    \begin{conseil}
        Les valeurs par défaut seront assignées à votre objet au cas où il est instancié sans valeurs. Ainsi en faisant \lstinline{f = Fraction()}, on obtiendra un objet \lstinline{Fraction} ayant pour valeurs un numérateur et un dénominateur à 1 soit $\frac{1}{1}$.
    \end{conseil}
\end{Exercice}

\begin{Exercice}[5 minutes]

\end{Exercice}


\section{Notions d'héritage - Java}
% TODO: Remplacer "sous-classe" par classe fille.
% TODO: Remplacer dessinateur par illustrateur.
% TODO: Utiliser \lstinline{} sur tous les mots-clés dont les noms d'attributs, classes, méthodes,...
% TODO: Au lieu de définir une méthode printInfo(), ne serait-ce pas plus judicieux de rédéfinir la méthode toString()?

Le but de cette partie est de pratiquer et d'assimiler les notions liées à l'héritage. Pour cela. nous allons nous inspirer de l'exemple présenté dans le cours. 

Nous allons créer une classe \lstinline{Livre()} qui contiendra deux sous-classes, \lstinline{Livre_Audio()} et \lstinline{Livre_Illustre()}. Les sous-classes hériteront des attributs et méthodes de la classe mère. 

\begin{Exercice}[10 minutes] Création de classe et sous-classes\\

Créez la classe mère \lstinline{Livre()} avec les caractéristiques suivantes:
\begin{itemize}
	\item une variable privée \lstinline{titre}
	\item une variable privée \lstinline{auteur}
	\item une variable privée \lstinline{annee}
	\item une variable privée \lstinline{note} (initialisée à \lstinline{-1})
	\item le constructeur public prenant en argument les trois premières variables ci-dessus
	\item une méthode \lstinline{printInfo()} qui affiche le titre, l'auteur, l'année et la note d'un ouvrage
	\item une méthode \lstinline{setNote()} qui permet de définir la variable \lstinline{note}
\end{itemize}

Créez les classes filles avec les caractéristiques suivantes:\\
\lstinline{class Livre_Audio extends Livre}
\begin{itemize}
	\item un attribut \lstinline{narrateur}

\end{itemize}
\lstinline{class Livre_Illustre extends Livre}
\begin{itemize}
	\item un attribut \lstinline{dessinateur}

\end{itemize}

\lstinputlisting{ressources/heritage.java}

\begin{conseil}
En java, lors de la déclaration d'une classe, le mot clef \lstinline{extends} permet d'indiquer qu'il s'agit d'une sous-classe de la classe indiquée. 

Le mot clef \lstinline{super} permet à la sous classe d'hérité d'éléments de la classe mère. \lstinline{super} peut être utilisé dans le constructeur de la sous-classe selon l'example suivant: \lstinline{super(variable_mère_1, variable_mère_2, variable_mère_3, etc.);}. Ainsi, il n'est pas nécessaire de redefinir toutes les variables d'une sous-classe !

L'instruction \lstinline{super} doit toujours être la première instruction dans le constructeur d'une sous-classe. 
\end{conseil}

\begin{solution}
	\lstinputlisting{Solutions/heritage-solution-1.java}
\end{solution}

\end{Exercice}

\begin{Exercice}[5 minutes] Méthode et héritage \\

Maintenant que vous avez créé la classe et les sous classes correspondantes, vous pouvez créer un objet \lstinline{Livre} à l'aide du constructeur de la sous-classe \lstinline{Livre_Audio}. 
% TODO: Phrase à reformuler. S'agit-il de valeurs par défaut du constructeur ou des valeurs de l'instance?
Si vous manquez d'inspiration vous pouvez indiquer les valeurs suivantes : titre: "Hamlet", auteur: "Shakespeare", année: "1609" et le narrateur "William.\\

Une fois l'objet créé, attribuez lui une note à l'aide de la méthode définie précédemment.\\ 

Finalement, utilisez la méthode \lstinline{printInfo()} pour afficher les informations du livre.\\

La méthode étant définie dans la classe mère, elle n'a pas connaissance de la variable \lstinline{narrateur} définie dans la sous-classe. Redéfinissez la méthode dans la sous-classe pour y inclure l'information sur le narrateur. \\

\begin{conseil}
\textbf{Attention}, on vous demande de créer un objet \lstinline{Livre} et non pas \lstinline{Livre_Audio}.

Le mot clef \lstinline{super} peut être utilisé dans la redéfinition d'une méthode selon l'example suivant: \lstinline{super.nom_de_la_methode();}. Cette instruction permet  d'inclure tout ce qui est défini dans la ``méthode mère'' et vous pouvez la complétez selon les caractéristiques de votre sous-classe.

%TODO: Phrase à reformuler. super représente l'objet parent. Il faudra faire le parallèle entre super et this (qui représente l'instance en cours).
L'instruction \lstinline{super} doit toujours être la première instruction dans le redéfinition d'une méthode dans une sous-classe. 
\end{conseil}

\begin{solution}
	\lstinputlisting{Solutions/heritage-solution-2.java}
	\lstinputlisting{Solutions/heritage-solution-3.java}
\end{solution}

\end{Exercice}

\section{Polymorphisme - Java}
% TODO: Utiliser \lstinline{} sur tous les mots-clés dont les noms d'attributs, classes, méthodes,... Par exemple sur Soigneur-

Dans cette partie, vous serez amenés à créer 2 nouvelles sous-classes de la classe mère \lstinline{Fighter}. La première classe représentera un \lstinline{Soigneur}, qui, lorsqu'il attaquera quelqu'un, le soignera au lieu de le blesser. La deuxième classe représentera un combattant spécialisé dans l'attaque \lstinline{Attaquant}, qui aura la capacité d'attaquer un certain nombre de fois (ce nombre sera défini au moment où vous l'instancierez). Pensez à télécharger la dernière version de la classe \lstinline{Fighter} dans le dossier ressources.\\

Voici le squelette du code que vous trouverez également dans le dossier ressources du moodle : \\

\lstinputlisting{ressources/Combattant_squelette.java}

\begin{Exercice}[5 minutes] Sous-classe \lstinline{Soigneur} \\

Commencez par déclarer une nouvelle sous-classe \lstinline{Soigneur}. Cette sous-classe prendra un nouvel attribut \lstinline{private}, \lstinline{int}, nommé \lstinline{résurrection}, qui vaudra 1 lors de l'instanciation. \\

Déclarez le \lstinline{constructeur} de cette classe ainsi que les \lstinline{getter} et \lstinline{setter} permettant d'interagir avec ce nouvel attribut (\lstinline{résurrection}). \\

\begin{conseil}
Pensez à utiliser le constructeur de votre classe mère \lstinline{Fighter}
\end{conseil}

\begin{solution}
	\lstinputlisting{Solutions/Soigneur.java}
\end{solution}

\end{Exercice}

\begin{Exercice}[10 minutes] Méthode \lstinline{résurrection(Fighter other)} de la sous-classe \lstinline{Soigneur} \\

Commencez par déclarer une nouvelle méthode nommée \lstinline{résurrection(Fighter other)}. \\
Cette méthode permettra de faire revenir un \lstinline{Fighter} à la vie, mais le \lstinline{Soigneur} ne pourra le faire qu'une seule fois. \\

Commencez par contrôler que l'instance depuis laquelle la méthode est appelée soit toujours en vie. Si ce n'est pas le cas, indiquez : \lstinline{nom_instance} est mort et ne peut plus rien faire.  \\ 

Contrôlez ensuite que l'instance \lstinline{other} soit vraiment morte. Si ce n'est pas le cas, indiquez le via : \lstinline{nom_other} est toujours en vie. \\

Pour finir, contrôlez que l'attribut \lstinline{résurrection} de l'instance depuis laquelle la méthode est appelée est égale à 1. Si ce n'est pas le cas, indiquez : \lstinline{nom_instance} ne peut plus ressusciter personne.\\

Si tous ces éléments sont réunis, faites revenir le \lstinline{Fighter} \lstinline{other} à la vie en lui remettant 10 points de vie et en l'ajoutant à la liste \lstinline{instances} de la classe \lstinline{Fighter}. Pensez également à mettre l'attribut \lstinline{résurrection} de l'instance appelée à   
0 afin de l'empêcher de réutiliser ce pouvoir, à appeler la méthode \lstinline{checkHealth()}, et à indiquer : \lstinline{nom_other} est revenu à la vie ! \\

\begin{conseil}
Utilisez un branchement conditionnel pour les contrôles. \\

Une nouvelle méthode nommée \lstinline{addInstances(Fighter other)} a été créée dans la classe \lstinline{Fighter}. Regardez à quoi elle sert et utilisez la. \\

Pour les indications en fonction des différentes conditions, imprimmez simplement la phrase en question. \\
\end{conseil}

\begin{solution}
	\lstinputlisting{Solutions/resurrection.java}
\end{solution}

\end{Exercice}

\begin{Exercice}[10 minutes] Méthode attack de la sous-classe \lstinline{Soigneur} \\

Réécrivez la méthode \lstinline{attack} de la sous-classe \lstinline{Soigneur} afin d'ajouter des points de vie à \lstinline{other} au lieu de lui en retirer. \\

Le seul argument nécessaire pour cette méthode sera le \lstinline{Fighter} \lstinline{other}. \\

Commencez par contrôler que le \lstinline{Soigneur} depuis lequel la méthode est appelée est encore en vie. Si ce n'est pas le cas, indiquez : \lstinline{nom_instance} est mort et ne peut plus rien faire.

Contrôlez ensuite si \lstinline{other} est toujours en vie. Si ce n'est pas le cas indiquez : \lstinline{nom_other} est déjà mort, ressuscitez le afin de pouvoir le soigner. Contrôlez également qu'il ait moins de 10 points de vie. Si ce n'est pas le cas, indiquez le via : \lstinline{nom_other} a déjà le maximum de points de vie. \\

Si toutes ces conditions sont réunies, ajoutez la valeur de l'attaque de l'instance qui appelle la méthode aux points de vie de \lstinline{other}, puis appelez la méthode de classe \lstinline{checkHealth()}.

\begin{conseil}
Pensez à utiliser du branchement conditionnel pour les contrôles. \\

Le nombre de points de vie à ajouter est simplement égal à l'attaque de l'instance depuis laquelle la méthode est appelée. Ajoutez la valeur de cet attribut \lstinline{attack} au \lstinline{Fighter} \lstinline{other}

\end{conseil}

\begin{solution}
	\lstinputlisting{Solutions/Soigneur_attack.java}
\end{solution}

\end{Exercice}

\begin{Exercice}[5 minutes] Sous-classe \lstinline{Attaquant} \\

Commencez par déclarer une nouvelle sous-classe \lstinline{Attaquant}. Cette sous-classe prendra un nouvel attribut \lstinline{private}, \lstinline{int}, nommé \lstinline{multiplicateur}, qui sera passé en argument du \lstinline{constructeur} de la sous-classe. \\

Déclarez le \lstinline{constructeur} de cette classe ainsi que les \lstinline{getter} et \lstinline{setter} permettant d'interagir avec ce nouvel attribut \lstinline{multiplicateur}. \\

\begin{conseil}
Pensez à utiliser le \lstinline{constructeur} de votre classe mère \lstinline{Combattant}.
\end{conseil}

\begin{solution}
	\lstinputlisting{Solutions/Attaquant.java}
\end{solution}

\end{Exercice}

\begin{Exercice}[10 minutes] Méthode \lstinline{attack} de la sous-classe \lstinline{Attaquant} \\

Réécrivez la méthode \lstinline{attack} de la sous-classe \lstinline{Attaquant} afin d'effectuer plusieurs attaques sur \lstinline{other} en fonction de l'attribut \lstinline{multiplicateur}. \\

Y'a t-il besoin de contrôler si l'instance depuis laquelle la méthode est appelée est encore en vie ? \\

Indiquez systématiquement le numéro de l'attaque, puis effectuez l'attaque. Répétez le procédé jusqu'à ce que le numéro de l'attaque soit égal à celui de \lstinline{multiplicateur_instance}.

\begin{conseil}
Aidez vous de la méthode \lstinline{attack} de la classe mère \lstinline{Combattant}.\\

Comment peut-on effectuer plusieurs fois une même séquence d'action en programmation ? \\

\end{conseil}

\begin{solution}
	\lstinputlisting{Solutions/Attaquant_attack.java}
\end{solution}

\end{Exercice}

Si tout est correct, en utilisant ce \lstinline{main} : \\

\lstinputlisting{ressources/Combattant_main.java} 

Vous devriez obtenir : \\

\lstinputlisting{Solutions/Combattant_main_solution.java} 

\section{Héritage en Python \optionnel}
% Python
% TODO: Attention, les décorateurs ne sont pas présentés en cours!
% Utiliser \lstinline{} sur tous les mots-clés (nom de classes, d'attributs, méthodes,...)
% Faire des exercices en français. Remplacer les noms de méthodes et attributs par des mots français qui puissent parler aux étudiants.

\begin{Exercice}[10 minutes]\textbf{Classe Point (Suite)}

Dans la série dernière, vous avez rencontré un exemple de classe en Python. Celui-là représente un point de 2 dimensions, x et y, ainsi que des calculs basiques des points 2D. Pour cet exercice, vous allez implémenter une classe des points de 3 dimensions en utilisant de l'héritage sur la classe \textbf{Point} qu'on a implémentée! À travers cet exercice, nous voudrions également vous présenter une syntaxe en Python qui concerne l'utilisation des 'décorateurs'.\\


Avant de commencer, nous voudrions attirer votre attentions sur les points suivants de la classe mère Point:
\begin{itemize}
    \item Tout d'abord, nous avons récrit la classe de la semaine dernière pour utiliser les 'decorators' en Python. Par exemple, vous trouverez une explication \href{https://python-reference.readthedocs.io/en/latest/docs/property/setter.html}{\textcolor{cyan}{ici}} pour le décorateur \textbf{setter}. Lisez bien les documentations et le code suivant pour comprendre l'usage de ces décorateurs.
    \item Nous avons changé le nom de la méthode \textbf{distance()} pour \textbf{euclidean\_distance()} pour la distinguer des autres types de distance. 
    \item Traiter la classe \textbf{Point} comme mère et faire l'hériter depuis la classe \textbf{Point3D} n'est pas la meilleure structure d'un programme Python, mais on la garde pour le moment afin de vous montrer comment les méthodes de classe mère peuvent être manipulées dans une classe fille.
\end{itemize}

Voici la classe \textbf{Point} qui était légèrement modifiée:
\lstinputlisting{ressources/Point.py} 

Ecrivez une classe qui hérite \textbf{Point}. Nommez-la \textbf{Point3D}. Après avoir rajouté la 3ème dimension comme attribut, implémentez les opérations ci-dessous:

\begin{itemize}
    \item Rajoutez une méthode qui renvoie une représentation vectorielle du point. Vous pouvez utiliser la \textbf{liste} en Python.
    \item Recalculez la distance euclidean et le milieu pour le point 3D.
	\item (Optionnel) Si vous voulez vous familiariser encore plus avec les méthodes de classe en Python, implémentez deux autres calculs de distance: Manhattan et Minkowski (détails : \url{https://www.analyticsvidhya.com/blog/2020/02/4-types-of-distance-metrics-in-machine-learning/} - % TODO: Remplacer par les articles Wikipédia correspondants
	)
\end{itemize}
\lstinputlisting{ressources/Point3D.py} 

\begin{conseil}
Recherchez sur Internet des documentations sur les décorateurs Python, notamment sur 'property', 'setter' et 'getter'. 

Que fait \textbf{super.\_\_init\_\_()}?

Vous voudrez aussi vous demander si la représentation vectorielle d'un point pourrait être une propriété au lieu d'une méthode. 
\end{conseil}
\begin{solution}
\lstinputlisting{Solutions/Point3D.py} 
\end{solution}
\end{Exercice} 

\begin{Exercice}[15 minutes]\textbf{Un exemple appliqué}

Dans les établissements universitaires, on rencontre souvent des problèmes lors du calcul de salaires du personnel. Sans penser aux recherches effectuées par certains professeurs, on va essayer de calculer les salaires de ceux qui sont reconnus comme 'Professor' à l'université et ceux qui y donnent des cours à temps partiel ('Part-time Lecturer').

La classe mère dans ce cas est nommée \textbf{Lecturer}, qui possède une propriété - le salaire annuel moyen. On voudrais que la méthode qui calcule cette quantité renvoie 60 000 (dollars américains) si l'enseignant a moins de 10 ans d'expériences, et 100 000 sinon. Si l'enseignant travaille à temps partiel, la méthode devrait renvoyer une chaîne qui dit 'Salary for part-time lecturers unknown'.

Ensuite, on veut calculer la paye mensuelle pour chaque type d'employé. Pour les 'Professors', la paye devrait être calculée sur la base de deux sources de revenu: un salaire mensuel et une commission pour chaque comité où ils participent. 

D'autre part, pour les 'Part-time Lecturers', la paye est calculée sur une base horaire i.e taux horaire $\times$ nombres d'heures de travail (par mois).
\lstinputlisting{ressources/Lecturer.py} 

\begin{conseil}
Où devrait-on mettre \textbf{super.\_\_init\_\_()} dans cet exemple ?

De nouveau, réfléchissez bien si quelques méthodes peuvent être traitées comme 'propriétés'.
\end{conseil}

\begin{solution}
\lstinputlisting{Solutions/Lecturer.py} 
\end{solution}
\end{Exercice}

\end{document}
