\documentclass[a4paper]{article}
\usepackage{times}
\usepackage[utf8]{inputenc}
\usepackage{selinput}
\usepackage{upquote}
\usepackage[margin=2cm, rmargin=4cm, tmargin=3cm]{geometry}
\usepackage{tcolorbox}
\usepackage{xspace}
\usepackage[french]{babel}
\usepackage{url}
\usepackage{hyperref}
\usepackage{fontawesome5}
\usepackage{marginnote}
\usepackage{ulem}
\usepackage{tcolorbox}
\usepackage{graphicx}
\usepackage{verbatimbox}
\usepackage{amsmath}
\usepackage{hyperref}
%\usepackage[top=Bcm, bottom=Hcm, outer=Ccm, inner=Acm, heightrounded, marginparwidth=Ecm, marginparsep=Dcm]{geometry}


\newtcolorbox{Example}[1]{colback=white,left=20pt,colframe=slideblue,fonttitle=\bfseries,title=#1}
\newtcolorbox{Solutions}[1]{colback=white,left=20pt,colframe=green,fonttitle=\bfseries,title=#1}
\newtcolorbox{Conseils}[1]{colback=white,left=20pt,colframe=slideblue,fonttitle=\bfseries,title=#1}
\newtcolorbox{Warning}[1]{colback=white,left=20pt,colframe=warning,fonttitle=\bfseries,title=#1}

\setlength\parindent{0pt}

  %Exercice environment
  \newcounter{exercice}
  \newenvironment{Exercice}[1][]
  {
  \par
  \stepcounter{exercice}\textbf{Question \arabic{exercice}:} (\faClock \enskip \textit{#1})
  }
  {\bigskip}
  

% Title
\newcommand{\titre}{\begin{center}
  \section*{Algorithmes et Pensée Computationnelle}
\end{center}}
\newcommand{\cours}[1]
{\begin{center} 
  \textit{#1}\\
\end{center}
  }


\newcommand{\exemple}[1]{\newline~\textbf{Exemple :} #1}
%\newcommand{\attention}[1]{\newline\faExclamationTriangle~\textbf{Attention :} #1}

% Documentation url (escape \# in the TP document)
\newcommand{\documentation}[1]{\faBookOpen~Documentation : \href{#1}{#1}}

% Clef API
\newcommand{\apikey}[1]{\faKey~Clé API : \lstinline{#1}}
\newcommand{\apiendpoint}[1]{\faGlobe~Url de base de l'API \href{#1}{#1}}

%Listing Python style
\usepackage{color}
\definecolor{slideblue}{RGB}{33,131,189}
\definecolor{green}{RGB}{0,190,100}
\definecolor{blue}{RGB}{121,142,213}
\definecolor{grey}{RGB}{120,120,120}
\definecolor{warning}{RGB}{235,186,1}

\usepackage{listings}
\lstdefinelanguage{texte}{
    keywordstyle=\color{black},
    numbers=none,
    frame=none,
    literate=
           {é}{{\'e}}1
           {è}{{\`e}}1
           {ê}{{\^e}}1
           {à}{{\`a}}1
           {â}{{\^a}}1
           {ù}{{\`u}}1
           {ü}{{\"u}}1
           {î}{{\^i}}1
           {ï}{{\"i}}1
           {ë}{{\"e}}1
           {Ç}{{\,C}}1
           {ç}{{\,c}}1,
    columns=fullflexible,keepspaces,
	breaklines=true,
	breakatwhitespace=true,
}
\lstset{
    language=Python,
	basicstyle=\bfseries\footnotesize,
	breaklines=true,
	breakatwhitespace=true,
	commentstyle=\color{grey},
	stringstyle=\color{slideblue},
  keywordstyle=\color{slideblue},
	morekeywords={with, as, True, False, Float, join, None, main, argparse, self, sort, __eq__, __add__, __ne__, __radd__, __del__, __ge__, __gt__, split, os, endswith, is_file, scandir, @classmethod},
	deletekeywords={id},
	showspaces=false,
	showstringspaces=false,
	columns=fullflexible,keepspaces,
	literate=
           {é}{{\'e}}1
           {è}{{\`e}}1
           {ê}{{\^e}}1
           {à}{{\`a}}1
           {â}{{\^a}}1
           {ù}{{\`u}}1
           {ü}{{\"u}}1
           {î}{{\^i}}1
           {ï}{{\"i}}1
           {ë}{{\"e}}1
           {Ç}{{\,C}}1
           {ç}{{\,c}}1,
    numbers=left,
}

\newtcbox{\mybox}{nobeforeafter,colframe=white,colback=slideblue,boxrule=0.5pt,arc=1.5pt, boxsep=0pt,left=2pt,right=2pt,top=2pt,bottom=2pt,tcbox raise base}
\newcommand{\projet}{\mybox{\textcolor{white}{\small projet}}\xspace}
\newcommand{\optionnel}{\mybox{\textcolor{white}{\small Optionnel}}\xspace}
\newcommand{\auto}{\mybox{\textcolor{white}{\small Auto-évaluation}}\xspace}


\usepackage{environ}
\newif\ifShowSolution
\NewEnviron{solution}{
  \ifShowSolution
	\begin{Solutions}{\faTerminal \enskip Solution}
		\BODY
	\end{Solutions}
  \fi}


  \usepackage{environ}
  \newif\ifShowConseil
  \NewEnviron{conseil}{
    \ifShowConseil
    \begin{Conseils}{\faLightbulb \quad Conseil}
      \BODY
    \end{Conseils}

    \fi}

    \usepackage{environ}
  \newif\ifShowWarning
  \NewEnviron{attention}{
    \ifShowWarning
    \begin{Warning}{\faExclamationTriangle \quad Attention}
      \BODY
    \end{Warning}

    \fi}
  

%\newcommand{\Conseil}[1]{\ifShowIndice\ \newline\faLightbulb[regular]~#1\fi}



\usepackage{array}
\newcolumntype{C}[1]{>{\centering\let\newline\\\arraybackslash\hspace{0pt}}m{#1}}

\usepackage{minted}

\NewDocumentCommand{\codeword}{v}{%
\texttt{\textcolor{blue}{#1}}%
}

\begin{document}

% Change the following values to true to show the solutions or/and the hints
\ShowSolutiontrue
\ShowConseiltrue
\titre
\cours{Algorithmes de tri et Complexité}

Le but de cette séance est d'aborder divers concepts de langages de programmation. En effet, la série d'exercices porte sur:
\begin{enumerate}
    \item la complexité des algorithmes,
    \item la récursion et
    \item les algorithmes de tri
\end{enumerate}

Les exercices sont disponibles en Python et en Java.


\section{Complexité (30 minutes)}

Pour chaque algorithme ci-dessous, indiquez en une phrase, ce que font ces algorithmes et calculez leur complexité temporelle avec la notation $O( )$. Le code est écrit en Python et en Java. \\

\begin{Exercice}[10 minutes] \textbf{Complexité} \\
    \textbf{Python :}
    \begin{minted}[fontsize=\footnotesize, autogobble, breaklines]{python}
    # Entrée: n un nombre entier
    def algo1(n):
        s = 0
        for i in range(10*n):
            s += i
        return s
    \end{minted}
    
    \textbf{Java :}
    \begin{minted}[fontsize=\footnotesize, autogobble, breaklines]{java}
        public static int algo1(int n) {
            int s = 0;
            for (int i=0; i < 10*n; i++){
                s += i;
            }
            return s;
        }
    \end{minted}
    
    \begin{conseil}
    Rappelez vous que la notation $O()$ sert à exprimer la complexité d'algorithmes dans le \textbf{pire scénario}. Les règles suivantes vous seront utiles. Pour $n$ étant la taille de vos données, on a que :
    \begin{enumerate}
        \item Les constantes sont ignorées : $O(2n) = 2*O(n) = O(n)$ 
        \item Les termes dominés sont ignorés : $O(2n^2+5n+50)$ = $O(n^2)$
    \end{enumerate}
    \end{conseil}
    \begin{solution}
        L'algorithme est composé d'une boucle qui incrémente une variable \lstinline{s}. Il effectue $10*n$ l'opération et par conséquent a une complexité de $O(n)$.
    \end{solution}
\end{Exercice}

\begin{Exercice}[10 minutes] \textbf{Complexité} \\
    \textbf{Python :}
    \begin{minted}[fontsize=\footnotesize, autogobble, breaklines]{python}
        # Entrée: liste de nombre entiers et M un nombre entier
        def algo2(L, M):
            i = 0
            while i < len(L) and L[i] <= M:
                i += 1
            s = i - 1
            return s
    \end{minted}
    
    \textbf{Java :}
    \begin{minted}[fontsize=\footnotesize, autogobble, breaklines]{java}
        public static int algo2(int[] L, int M) {
            int i = 0;
            while (i < L.length && L[i] <= M){
                i += 1;
            }
            int s = i - 1;
            return s;
        }
    \end{minted}
    \begin{solution}
    L'algorithme est composé d'une boucle \lstinline{while} qui va parcourir une liste \lstinline{L} jusqu'à trouver une valeur qui soit supérieur à \lstinline{M}. Ainsi, dans le pire scénario, l'algorithme parcourt toute la liste, et a donc une complexité de $O(n)$, $n$ étant la taille de la liste.
    \end{solution}
\end{Exercice}
\begin{Exercice}[10 minutes] \textbf{Complexité} \\

        \textbf{Python :}
        \begin{minted}[fontsize=\footnotesize, autogobble, breaklines]{python}
        #Entrée: L et M sont 2 listes de nombre entiers
        def algo3(L, M):
            n = len(L)
            m = len(M)
            for i in range(n):
                L[i] = L[i]*2
            for j in range(m):
                M[j] = M[j]%2
        \end{minted}
        
        \textbf{Java :}
        \begin{minted}[fontsize=\footnotesize, autogobble, breaklines]{java}
            public static void algo3(int[] L, int[] M) {
                int n = L.length;
                int m = M.length;
                for (int i=0; i < n; i++){
                    L[i] = L[i]*2;
                }
                for (int j=0; j < m; j++){
                    M[j] = M[j]%2;
                }
            }
        \end{minted}
        \begin{solution}
        L'algorithme est composé de 2 boucles, une qui parcourt \lstinline{L} et l'autre qui parcourt \lstinline{M}. Ainsi pour $n$ et $m$ étant les tailles respectives de \lstinline{L} et de \lstinline{M}, on a que la complexité est $O(n) + O(m) = O(\max\{n,m\})$. Ainsi la taille la plus grande domine la taille la plus petite.
        \end{solution}
\end{Exercice}
\begin{Exercice}[10 minutes] \textbf{Complexité (Optionnel)} \\        
        \textbf{Python :}
        \begin{minted}[fontsize=\footnotesize, autogobble, breaklines]{python}
        # Entrée: n un nombre entier
        def algo4(n):
            m = 0
            for i in range(n):
                for j in range(i):
                    m += i+j
            return m
        \end{minted}
        
        \textbf{Java :}
        \begin{minted}[fontsize=\footnotesize, autogobble, breaklines]{java}
            public static int algo4(int n) {
                int m = 0;
                for (int i=0; i < n; i++){
                    for (int j=0; j < i; j++){
                        m += i+j;
                    }
                }
                return m;
            }
        \end{minted}
    \begin{solution} 
    L'algorithme est composé de 2 boucles \textbf{imbriquées}. Cela veut dire que nous parcourons la liste un maximum de $n \times n$ fois, $n$ étant la taille de la liste. La complexité de l'algorithme est ainsi $O(n^2)$.
    \end{solution}
    
\end{Exercice}
    
        
\section{Récursion (15 minutes)}

Le but principal de la récursion est de résoudre un gros problème en le divisant en plusieurs petites parties à résoudre.

\begin{Exercice} [5 minutes] \textbf{Somme des chiffres} \\

Écrivez un algorithme récursif en Python ou en Java qui prend un nombre et retourne la somme des chiffres dont il est composé. Par exemple, la somme des chiffres de 126 est : 1+2+6 = 9.

\begin{conseil}

Pour obtenir les chiffres qui composent un nombre, utilisez l'opérateur \lstinline{\%}. \\
Pour obtenir le nombre 12 à partir du nombre 126, il vous suffit de faire la division entière par 10. En Python, on utilise l'opérateur \lstinline{\\\\} : 126 \lstinline{\\\\} 10 = 12. En Java, la division entre deux variables de type \lstinline{int} est entière, et vous n'aurez ainsi qu'à utiliser l'opérateur de division normal \lstinline{\\} : 126 \lstinline{\\} 10 = 12

\end{conseil}

\textbf{Solutions :}

\begin{itemize}
    \item \textbf{Python :}
        \begin{minted}[fontsize=\footnotesize, autogobble, breaklines]{python}
            def sum_digits(number):
              if number == 0:
                return 0
              else:
                return (number%10) + sumdigits(number//10)
        \end{minted}
        
    \item \textbf{Java :}
        \begin{minted}[fontsize=\footnotesize, autogobble, breaklines]{java}
                public static int sum_digits(int number) {
                    if(number == 0){
                        return 0;
                    } else{
                        return (number%10) + sumdigits(number/10);
                    }
                }
                
        \end{minted}

\end{itemize}
\end{Exercice}

\begin{Exercice} [10 minutes] \textbf{Fibonacci} \\

La suite de Fibonacci est définie récursivement par :
\begin{itemize}
    \item si n est 0 ou 1 : fibo(0) = fibo(1) = 1
    \item si n au moins égal à 2, alors ; fibo (n) = fibo(n - 1) + fibo(n - 2)
\end{itemize}

Voici son implémentation en Java :

\begin{minted}[fontsize=\footnotesize, autogobble, breaklines]{java}
        public static int fibonacci(int n) {
            if(n == 0 | n == 1){
                return n;
            } else{
                return fibonacci(n-1) + fibonacci(n-2);
            }
        }
        
\end{minted}

Calculez la compléxité de l'algorithme.

\begin{conseil}
Aidez-vous d'un exemple (\lstinline{fibonacci(3)}, \lstinline{fibonacci(4)},...) \\
Pour formaliser la formule de complexité, on peut poser que $T(n)$ énumère le nombre d'opérations requises pour calculer \lstinline{fibonacci(n)}. Ainsi, $T(n) = T(n-1) + T(n-2) + c$, $c$ étant une constante. Vous pouvez alors énumérer le nombre d'opérations pour \lstinline{fibonacci(3)}, \lstinline{fibonacci(4)}... et esssayer de trouver la complexité en terme de $O()$.
\end{conseil}
\ \\

\begin{Solutions}
    La complexité de cet algorithme est $O(2^n)$.
\end{Solutions}
\end{Exercice}

\section{Algorithmes de Tri (60 minutes)}

\begin{Exercice} [20 minutes] \textbf{Tri à bulles (Insertion Sort)} \\
Le tri à bulles consiste à parcourir une liste et à comparer ses éléments. Le tri est effectué en permutant les éléments de telle sorte que les éléments les plus grands soient placés à la fin de la liste. 

Concrètement, si un premier nombre $x$ est plus grand qu'un deuxième nombre $y$ et que l'on souhaite trier l'ensemble par ordre croissant, alors $x$ et $y$ sont mal placés et il faut les inverser. Si, au contraire, $x$ est plus petit que $y$, alors on ne fait rien et l'on compare $y$ à $z$, l'élément suivant.

Soit la liste \lstinline{l} suivante, trier les éléments de la liste suivante en utilisant un tri à bulles. Combien d'itération effectuez-vous?

\begin{itemize}
        \item \textbf{Python :}
                \begin{minted}[fontsize=\footnotesize, autogobble, breaklines]{python}
                    def tri_bulle(l):
                        for i in range(len(l)):
                            #TODO: Code à compléter
                    
                    if __name__ == "__main__":
                        l = [1, 2, 4, 3, 1]
                        tri_bulle(l)
                        print(liste_triee)
                \end{minted}
        \item \textbf{Java :}
                \begin{minted}[fontsize=\footnotesize, autogobble, breaklines]{java}
                    public class Main {
                        public static void tri_bulle(int[] l) {
                            for (int i = 0; i < l.length - 1; i++){
                                //TODO: Code à compléter 
                            }
                        }
                        
                        public static void printArray(int l[]){ 
                            int n = l.length; 
                            for (int i = 0; i < n; ++i) 
                                System.out.print(arr[i] + " "); 
                      
                            System.out.println(); 
                        } 
              
                        
                        public static void main(String[] args){
                            int[] l = {1, 2, 4, 3, 1};
                            tri_bulle(l);
                            printArray(l);
                        }
                    }
                \end{minted}
    \end{itemize}
    
    \begin{conseil}
    En Java, utilisez une variable temporaire \lstinline{temp} afin de faire l'échange de valeur entre deux cases dans une liste.
    \end{conseil}
    
    \ \\
    
    \textbf{Solutions :}
    \textbf{Python :}
                \begin{minted}[fontsize=\footnotesize, autogobble, breaklines]{python}
                    def tri_bulle(l):
                        for i in range(1, len(l)):
                            # Les i derniers élements sont dans leur bonne position 
                            for j in range(0, n-i-1): 
                      
                                # parcourir la liste de 0 à n-i-1 
                                # Echanger si l'élément trouvé est supérieur 
                                # au prochain élement
                                if l[j] > l[j+1] : 
                                    l[j], l[j+1] = l[j+1], l[j] 
                    
                    if __name__ == "__main__":
                        l = [1, 2, 4, 3, 1]
                        tri_bulle(l)
                        print(liste_triee)
                \end{minted}
        \textbf{Java :}
                \begin{minted}[fontsize=\footnotesize, autogobble, breaklines]{java}
                    public class Main {
                        public static void tri_bulle(int[] l) {
                            for (int i = 0; i < l.length - 1; i++){
                                for (int j = 0; j < n-i-1; j++) {
                                    if (l[j] > l[j+1]) { 
                                        // échange l[j+1] et l[i] 
                                        int temp = l[j]; 
                                        l[j] = l[j+1]; 
                                        l[j+1] = temp; 
                                    } 
                            }
                        }
                        
                        public static void printArray(int l[]){ 
                            int n = l.length; 
                            for (int i = 0; i < n; ++i) 
                                System.out.print(arr[i] + " "); 
                      
                            System.out.println(); 
                        } 
              
                        
                        public static void main(String[] args){
                            int[] l = {1, 2, 4, 3, 1};
                            tri_bulle(l);
                            printArray(l);
                        }
                    }
                \end{minted}
        
        L'algorithme a une complexité de $O(n^2)$ car il contient deux boucles qui parcourent la liste.
    
\end{Exercice}

\begin{Exercice} [20 minutes] \textbf{Tri par insertion (Insertion Sort)} \\

    Dans l'algorithme, on parcourt le tableau à trier du début à la fin. Au moment où on considère le i-ème élément, les éléments qui le précèdent sont déjà triés. Pour faire l'analogie avec l'exemple du jeu de cartes, lorsqu'on est à la i-ème étape du parcours, le i-ème élément est la carte saisie, les éléments précédents sont la main triée et les éléments suivants correspondent aux cartes encore en désordre sur la table. 
    
    L'objectif d'une étape est d'insérer le i-ème élément à sa place parmi ceux qui précèdent. Il faut pour cela trouver où l'élément doit être inséré en le comparant aux autres, puis décaler les éléments afin de pouvoir effectuer l'insertion. En pratique, ces deux actions sont fréquemment effectuées en une passe, qui consiste à faire « remonter » l'élément au fur et à mesure jusqu'à rencontrer un élément plus petit. 
    
    Compléter le code suivant pour trier la liste \lstinline{l} définie ci-dessous en utilisant un tri par insertion. Combien d'itérations effectuez-vous?
    \begin{itemize}
        \item \textbf{Python :}
                \begin{minted}[fontsize=\footnotesize, autogobble, breaklines]{python}
                    def tri_insertion(l):
                        for i in range(1, len(l)):
                            #TODO: Code à compléter
                    
                    if __name__ == "__main__":
                        l = [2, 43, 1, 3, 43]
                        tri_insertion(l)
                        print(l)
                \end{minted}
        \item \textbf{Java :}
                \begin{minted}[fontsize=\footnotesize, autogobble, breaklines]{java}
                    public class Main {
                        public static void tri_insertion(int[] l) {
                            for (int i = 1; i < l.length; i++){
                                //TODO: Code à compléter 
                            }
                        }
                        
                        public static void printArray(int l[]){ 
                            int n = l.length; 
                            for (int i = 0; i < n; ++i) 
                                System.out.print(arr[i] + " "); 
                      
                            System.out.println(); 
                        } 
              
                        
                        public static void main(String[] args){
                            int[] l = {2, 43, 1, 3, 43};
                            tri_insertion(l);
                            printArray(l);
                        }
                    }
                \end{minted}
    \end{itemize}
    
    \begin{conseil}
        Référez vous à la figure du dessous pour un exemple de tri par insertion.
    \end{conseil}
    
    \begin{figure}[h!]
        \centering
        \includegraphics[width=10cm]{ressources/tri_insertion.png}
    \end{figure}

    \textbf{Solutions :}
    \begin{itemize}
        \item \textbf{Python :}
                \begin{minted}[fontsize=\footnotesize, autogobble, breaklines]{python}
                    def tri_insertion(l):
                        for i in range(1, len(l)):
                            key = l[i] 
                            j = i-1
                            
                            while j >= 0 and key < l[j] : 
                                    l[j + 1] = l[j] 
                                    j -= 1
                            l[j + 1] = key 
                    
                    if __name__ == "__main__":
                        l = [2, 43, 1, 3, 43]
                        tri_insertion(l)
                        print(l)
                \end{minted}
        \item \textbf{Java :}
                \begin{minted}[fontsize=\footnotesize, autogobble, breaklines]{java}
                    public class Main {
                        public static void tri_insertion(int[] l) {
                            for (int i = 1; i < l.length; i++){
                                int key = l[i]; 
                                int j = i - 1; 
                      
                                while (j >= 0 && l[j] > key) { 
                                    l[j + 1] = l[j]; 
                                    j = j - 1; 
                                } 
                                l[j + 1] = key; 
                            }
                        }
                        
                        public static void printArray(int l[]){ 
                            int n = l.length; 
                            for (int i = 0; i < n; ++i) 
                                System.out.print(l[i] + " "); 
                      
                            System.out.println(); 
                        } 
              
                        
                        public static void main(String[] args){
                            int[] l = {2, 43, 1, 3, 43};
                            tri_insertion(l);
                            printArray(l);
                        }
                    }
                \end{minted}
    \end{itemize}
    
    La complexité de l'algorithme est de $O(n^2)$ car nous utilisons 2 boucles imbriquées, qui dans le pire des cas, parcourent la liste deux fois.
    
\end{Exercice}

\begin{Exercice} [30 minutes] \textbf{Tri fusion (Merge Sort)} \\
    À partir de deux listes triées, on peut facilement construire une liste triée comportant les éléments issus de ces deux listes (leur \textit{fusion}). Le principe de l'algorithme de tri fusion repose sur cette observation : le plus petit élément de la liste à construire est soit le plus petit élément de la première liste, soit le plus petit élément de la deuxième liste. Ainsi, on peut construire la liste élément par élément en retirant tantôt le premier élément de la première liste, tantôt le premier élément de la deuxième liste (en fait, le plus petit des deux, à supposer qu'aucune des deux listes ne soit vide, sinon la réponse est immédiate). 
    
    Les pas de l'algorithme sont comme suit:
    \begin{enumerate}
        \item Si le tableau n'a qu'un élément, il est déjà trié.
        \item Sinon, séparer le tableau en deux parties à peu près égales.
        \item Trier récursivement les deux parties avec l'algorithme du tri fusion.
        \item Fusionner les deux tableaux triés en un seul tableau trié.
    \end{enumerate}
    
    Soit la liste \lstinline{l} suivante, trier les éléments de la liste suivante en utilisant un tri à bulles. Combien d'itération effectuez-vous?
    
    \begin{itemize}
        \item \textbf{Python :}
                \begin{minted}[fontsize=\footnotesize, autogobble, breaklines]{python}
                    def merge(partie_gauche, partie_droite):
                        #TODO: Code à compléter
                    def tri_fusion(l):
                        #TODO: Code à compléter
                    
                    if __name__ == "__main__":
                        l = [38, 27, 43, 3, 9, 82, 10]
                        print(tri_fusion(l))
                \end{minted}
        \item \textbf{Java :}
                \begin{minted}[fontsize=\footnotesize, autogobble, breaklines]{java}
                    public class Main {
                        // Fusionne 2 sous-listes de arr[]. 
                        // Première sous-liste est arr[l..m] 
                        // Deuxième sous-liste est arr[m+1..r] 
                        public static void merge(int arr[], int l, int m, int r) {
                            //TODO: Code à compléter 
                        }
                        
                        // Fonction principale qui trie arr[l..r] en utilisant 
                        // merge() 
                        public static void tri_fusion(int arr[], int l, int r){
                            //TODO: Code à compléter 
                        }
                        
                        public static void printArray(int l[]){ 
                            int n = l.length; 
                            for (int i = 0; i < n; ++i) 
                                System.out.print(arr[i] + " "); 
                      
                            System.out.println(); 
                        } 
              
                        
                        public static void main(String[] args){
                            int[] l = {38, 27, 43, 3, 9, 82, 10};
                            tri_fusion(l);
                            printArray(l);
                        }
                    }
                \end{minted}
    \end{itemize}
    
    \begin{conseil}
    \begin{itemize}
        \item L'algorithme est récursif. 
        \item Revenez à la visualisation de l'algorithme dans les diapositives pour comprendre comment marche concrètement le tri fusion. 
    \end{itemize}
    
    \end{conseil}
    
    \textbf{Solutions :} \\
        \textbf{Python :}
            
            \begin{minted}[fontsize=\footnotesize, autogobble, breaklines]{python}
                def merge(partie_gauche, partie_droite):
                    # créer la liste qui sera retournée à la fin
                    liste_fusionnee = []   
                    
                    # définir un compteur pour l'index de la liste de gauche
                    compteur_gauche = 0   
                    # pareil pour la liste de droite
                    compteur_droite = 0   
                    
                    longueur_gauche = len(partie_gauche)  
                    longueur_droite = len(partie_droite) 
                    
                    # continuer jusqu'à ce que l'un des index (ou les deux) atteigne l'une des longueurs (ou les deux)
                    while compteur_gauche < longueur_gauche and compteur_droite < longueur_droite:
                        # comparer les éléments actuels, ajouter le plus petit à la liste fusionnée 
                        # et augmenter le compteur de cette liste
                        if partie_gauche[compteur_gauche] < partie_droite[compteur_droite]:
                            liste_fusionnee.append(partie_gauche[compteur_gauche])
                            compteur_gauche += 1
                        else:
                            liste_fusionnee.append(partie_droite[compteur_droite])
                            compteur_droite += 1
                    
                    # s'il y a encore des éléments dans les listes, il faut les ajouter à la liste fusionnée
                    liste_fusionnee += partie_gauche[compteur_gauche:longueur_gauche]
                    liste_fusionnee += partie_droite[compteur_droite:longueur_droite]
                    
                    return liste_fusionnee   # retourner la liste fusionnée
                
                def tri_fusion(l):
                    # complèter la fonction
                    longueur = len(l)   # calculer la longueur de la liste
                    # s'il n'y a pas plus d'un élément, retourner la liste
                    if longueur == 1 or longueur == 0:
                        return l
                    # sinon, diviser la liste en deux
                    elif longueur > 1:
                        # convertir la variable en nombre entier (l'index ne peut pas être un nombre à virgule)
                        index_milieu = int(longueur / 2)   
                        # la partie gauche va du 1er élément à celui du milieu
                        partie_gauche = l[0:index_milieu] 
                        # la partie droite va du milieu à la fin de la liste
                        partie_droite = l[index_milieu:longueur]   
                        
                        # appeler la fonction tri_fusion à nouveau sur la partie gauche (récursivité)
                        partie_gauche_triee = tri_fusion(partie_gauche)
                         # même chose pour la partie droite
                        partie_droite_triee = tri_fusion(partie_droite)  
                        
                        liste_fusionnee = merge(partie_gauche_triee, partie_droite_triee)   # enfin, joindre les 2 parties
                        
                        # retourner le résultat
                        return liste_fusionnee   
                
                if __name__ == "__main__":
                        l = [38, 27, 43, 3, 9, 82, 10]
                        print(tri_fusion(l))
    
            \end{minted}
            \textbf{Java :}
            \begin{minted}[fontsize=\footnotesize, autogobble, breaklines]{java}
                public class Main {
                    // Fusionne 2 sous-listes de arr[]. 
                    // Première sous-liste est arr[l..m] 
                    // Deuxième sous-liste est arr[m+1..r] 
                    public static void merge(int arr[], int l, int m, int r) {
                        // Trouver la taille des deux sous-listes à fusionner
                        int n1 = m - l + 1; 
                        int n2 = r - m; 
                  
                        /* Créer des listes temporaires */
                        int L[] = new int[n1]; 
                        int R[] = new int[n2]; 
                  
                        /*Copier les données dans les sous-listes temporaires */
                        for (int i = 0; i < n1; ++i) 
                            L[i] = arr[l + i]; 
                        for (int j = 0; j < n2; ++j) 
                            R[j] = arr[m + 1 + j]; 
                  
                        /* Fusionner les sous-listes temporaires */
                  
                        // Indexes initiaux de la première et seconde sous-liste
                        int i = 0, j = 0; 
                  
                        // Index initial de la sous-liste fusionnée
                        int k = l; 
                        while (i < n1 && j < n2) { 
                            if (L[i] <= R[j]) { 
                                arr[k] = L[i]; 
                                i++; 
                            } 
                            else { 
                                arr[k] = R[j]; 
                                j++; 
                            } 
                            k++; 
                        } 
                  
                        /* Copier les élements restants de L[] */
                        while (i < n1) { 
                            arr[k] = L[i]; 
                            i++; 
                            k++; 
                        } 
                  
                        /* Copier les élements restants de R[] */
                        while (j < n2) { 
                            arr[k] = R[j]; 
                            j++; 
                            k++; 
                        }  
                    }
                    
                    // Fonction principale qui trie arr[l..r] en utilisant 
                    // merge() 
                    public static void tri_fusion(int arr[], int l, int r){
                        if (l < r) { 
                            // Trouver le milieu de la liste
                            int m = (l + r) / 2; 
                  
                            // Trier la première et la deuxième parties de la liste
                            tri_fusion(arr, l, m); 
                            tri_fusion(arr, m + 1, r); 
                  
                            // Fusionner les deux parties
                            merge(arr, l, m, r); 
                        } 
                    }
                    
                    public static void printArray(int l[]){ 
                        int n = l.length; 
                        for (int i = 0; i < n; ++i) 
                            System.out.print(arr[i] + " "); 
                  
                        System.out.println(); 
                    } 
          
                    
                    public static void main(String[] args){
                        int[] l = {38, 27, 43, 3, 9, 82, 10};
                        tri_fusion(l);
                        printArray(l);
                    }
                }
        \end{minted}
        
        Le tri fusion est un algorithme récursif. Ainsi, nous pouvons exprimer la complexité temporelle via une relation de récurence : $T(n) = 2T(n/2) + O(n)$. En effet, l'algorithme comporte 3 étapes :
        \begin{enumerate}
            \item la "Divide Step", qui divise les listes en deux sous-listes, et cela prend un temps constant
            \item la "Conquer Step", qui trie récursivement les sous-listes de taille $n/2$ chacune, et cette étape est représentée par le terme $2T(n/2)$ dans l'équation.
            \item l'étape où l'on fusionne les listes, qui prend $O(n)$.
        \end{enumerate}
        La solution à cette équation est $O(n \log n)$.
\end{Exercice}


\end{document}