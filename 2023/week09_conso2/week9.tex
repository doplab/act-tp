\documentclass[a4paper]{article}
\usepackage{times}
\usepackage[utf8]{inputenc}
\usepackage{selinput}
\usepackage{upquote}
\usepackage[margin=2cm, rmargin=4cm, tmargin=3cm]{geometry}
\usepackage{tcolorbox}
\usepackage{xspace}
\usepackage[french]{babel}
\usepackage{url}
\usepackage{hyperref}
\usepackage{fontawesome5}
\usepackage{marginnote}
\usepackage{ulem}
\usepackage{tcolorbox}
\usepackage{graphicx}
%\usepackage[top=Bcm, bottom=Hcm, outer=Ccm, inner=Acm, heightrounded, marginparwidth=Ecm, marginparsep=Dcm]{geometry}


\newtcolorbox{Example}[1]{colback=white,left=20pt,colframe=slideblue,fonttitle=\bfseries,title=#1}
\newtcolorbox{Solutions}[1]{colback=white,left=20pt,colframe=green,fonttitle=\bfseries,title=#1}
\newtcolorbox{Conseils}[1]{colback=white,left=20pt,colframe=slideblue,fonttitle=\bfseries,title=#1}
\newtcolorbox{Warning}[1]{colback=white,left=20pt,colframe=warning,fonttitle=\bfseries,title=#1}

\setlength\parindent{0pt}

  %Exercice environment
  \newcounter{exercice}
  \newenvironment{Exercice}[1][]
  {
  \par
  \stepcounter{exercice}\textbf{Question \arabic{exercice}:} (\faClock \enskip \textit{#1})
  }
  {\bigskip}
  

% Title
\newcommand{\titre}{\begin{center}
  \section*{Algorithmes et Pensée Computationnelle}
\end{center}}
\newcommand{\cours}[1]
{\begin{center} 
  \textit{#1}\\
\end{center}
  }


\newcommand{\exemple}[1]{\newline~\textbf{Exemple :} #1}
%\newcommand{\attention}[1]{\newline\faExclamationTriangle~\textbf{Attention :} #1}

% Documentation url (escape \# in the TP document)
\newcommand{\documentation}[1]{\faBookOpen~Documentation : \href{#1}{#1}}

% Clef API
\newcommand{\apikey}[1]{\faKey~Clé API : \lstinline{#1}}
\newcommand{\apiendpoint}[1]{\faGlobe~Url de base de l'API \href{#1}{#1}}

%Listing Python style
\usepackage{color}
\definecolor{slideblue}{RGB}{33,131,189}
\definecolor{green}{RGB}{0,190,100}
\definecolor{blue}{RGB}{121,142,213}
\definecolor{grey}{RGB}{120,120,120}
\definecolor{warning}{RGB}{235,186,1}

\usepackage{listings}
\lstdefinelanguage{texte}{
    keywordstyle=\color{black},
    numbers=none,
    frame=none,
    literate=
           {é}{{\'e}}1
           {è}{{\`e}}1
           {ê}{{\^e}}1
           {à}{{\`a}}1
           {â}{{\^a}}1
           {ù}{{\`u}}1
           {ü}{{\"u}}1
           {î}{{\^i}}1
           {ï}{{\"i}}1
           {ë}{{\"e}}1
           {Ç}{{\,C}}1
           {ç}{{\,c}}1,
    columns=fullflexible,keepspaces,
	breaklines=true,
	breakatwhitespace=true,
}
\lstset{
    language=Python,
	basicstyle=\bfseries\footnotesize,
	breaklines=true,
	breakatwhitespace=true,
	commentstyle=\color{grey},
	stringstyle=\color{slideblue},
  keywordstyle=\color{slideblue},
	morekeywords={with, as, True, False, Float, join, None, main, argparse, self, sort, __eq__, __add__, __ne__, __radd__, __del__, __ge__, __gt__, split, os, endswith, is_file, scandir, @classmethod},
	deletekeywords={id},
	showspaces=false,
	showstringspaces=false,
	columns=fullflexible,keepspaces,
	literate=
           {é}{{\'e}}1
           {è}{{\`e}}1
           {ê}{{\^e}}1
           {à}{{\`a}}1
           {â}{{\^a}}1
           {ù}{{\`u}}1
           {ü}{{\"u}}1
           {î}{{\^i}}1
           {ï}{{\"i}}1
           {ë}{{\"e}}1
           {Ç}{{\,C}}1
           {ç}{{\,c}}1,
    numbers=left,
}

\newtcbox{\mybox}{nobeforeafter,colframe=white,colback=slideblue,boxrule=0.5pt,arc=1.5pt, boxsep=0pt,left=2pt,right=2pt,top=2pt,bottom=2pt,tcbox raise base}
\newcommand{\projet}{\mybox{\textcolor{white}{\small projet}}\xspace}
\newcommand{\optionnel}{\mybox{\textcolor{white}{\small Optionnel}}\xspace}
\newcommand{\advanced}{\mybox{\textcolor{white}{\small Pour aller plus loin}}\xspace}
\newcommand{\auto}{\mybox{\textcolor{white}{\small Auto-évaluation}}\xspace}


\usepackage{environ}
\newif\ifShowSolution
\NewEnviron{solution}{
  \ifShowSolution
	\begin{Solutions}{\faTerminal \enskip Solution}
		\BODY
	\end{Solutions}
  \fi}


  \usepackage{environ}
  \newif\ifShowConseil
  \NewEnviron{conseil}{
    \ifShowConseil
    \begin{Conseils}{\faLightbulb \quad Conseil}
      \BODY
    \end{Conseils}

    \fi}

    \usepackage{environ}
  \newif\ifShowWarning
  \NewEnviron{attention}{
    \ifShowWarning
    \begin{Warning}{\faExclamationTriangle \quad Attention}
      \BODY
    \end{Warning}

    \fi}
  

%\newcommand{\Conseil}[1]{\ifShowIndice\ \newline\faLightbulb[regular]~#1\fi}


\usepackage{array}
\usepackage{blindtext}
\usepackage{multicol}
\newcolumntype{C}[1]{>{\centering\let\newline\\\arraybackslash\hspace{0pt}}m{#1}}

\begin{document}
% Change the following values to true to show the solutions or/and the hints
\ShowSolutiontrue
\ShowConseiltrue
\ShowNotefalse
\titre
\cours{Consolidation POO + Classes abstraites et interfaces}

% Le but de cette séance est d'approfondir les notions de programmation orientée objet vues précédemment. 

Les exercices sont construits autour des concepts d'héritage, de classes abstraites et d'interfaces. Au terme de cette séance, vous devez être en mesure de différencier une classe abstraite d'une interface, savoir à quel moment utiliser l'un ou l'autre, factoriser votre code afin de le rendre mieux structuré et plus lisible.

Cette feuille d'exercices est divisée en 3 sections. La première portant sur des notions de bases en POO, la deuxième sur les classes abstraites et la dernière sur les interfaces. 

Les exercices de cette feuille d'exercices doivent être faits uniquement en \lstinline{Java}.

Le code présenté dans les énoncés se trouve sur Moodle, dans le dossier \lstinline{Code}.

\section{Consolidation - Programmation Orientée Objet}

\begin{Exercice}[10 minutes]{Encapsulation - Java}

L'encapsulation sert à cacher les détails d'implémentation. Elle sert uniquement à montrer que les informations essentielles aux utilisateurs.\\

En Java, il est recommandé de déclarer les attributs comme étant \lstinline{private} et de mettre à disposition des utilisateurs des méthodes publiques d'accès afin qu'ils puissent accéder ou modifier la valeur des attributs privés.\\

Les méthodes publiques d'accès comme \lstinline{getName} et \lstinline{setName} doivent être nommées avec soit \lstinline{get} soit \lstinline{set} suivi du nom de l'attribut avec la 1ère lettre en majuscule \href{https://www.oracle.com/java/technologies/javase/codeconventions-namingconventions.html}{(Java Naming convention)}. 

Le mot-clé \lstinline{this} fait référence à l'objet en question.\\

\begin{enumerate}
	\item Dans votre IDE, créez une classe \lstinline{Person},
	\item Ajoutez-y un attribut privé \lstinline{name},
	\item Créez un getter et un setter pour l'attribut \lstinline{name} en suivant la convention de nommage des méthodes.
	\item Dans votre \lstinline{main}, créez une instance de \lstinline{Person}, 
	\item En utilisant le setter défini précédemment, donnez un nom (\lstinline{name}) à votre instance.
	\item Affichez le nom de l'instance en utilisant le getter de l'attribut \lstinline{name}.
\end{enumerate}
	\begin{solution}  
		\lstinputlisting{solutions/Person.java} 
	\end{solution}
\end{Exercice}
	
	
\begin{Exercice}[10 minutes]{Héritage - Java}

	Comme vous le savez, il est possible que des classes-filles héritent des attributs ou méthodes de classes-mères.
	En Java, il faut utiliser le mot-clé \lstinline{extends} lorsqu'on défini une classe-fille. Ainsi, l'héritage permet la réutilisation des attributs et méthodes d'une classe existante.

	\begin{enumerate}
		\item Créez une classe-mère \lstinline{Vehicle} ayant un attribut protégé appelé \lstinline{brand}.
		\item Créez un constructeur pour la classe \lstinline{Vehicle} et assignez une valeur à l'attribut \lstinline{brand}.
		\item Définissez une méthode \lstinline{honk} qui affiche \lstinline{"Tuut, tuut!"}
		\item Créez une classe-fille \lstinline{Car} qui hérite de \lstinline{Vehicle} et ayant pour attribut \lstinline{modelName} avec pour valeur par défaut \lstinline{"Mustang"}.
	\end{enumerate}

	\begin{conseil}
		Dans le constructeur de la classe-fille, n'oubliez pas de faire appel au constructeur de la classe-mère en utilisant le mot-clé \lstinline{super}.
	\end{conseil}	
	
	\begin{solution}
		\lstinputlisting{solutions/Vehicle.java} 
	\end{solution}
\end{Exercice}

\begin{Exercice}[30 minutes]{Surcharge de méthodes, égalité et identité - Java}

En Java, créez une classe \lstinline{Fraction} ayant pour attributs privés un numérateur et un dénominateur. 
\begin{conseil}
	Vous pouvez vous inspirer des solutions d'exercices de la semaine 8.
\end{conseil}
Assignez des valeurs à vos attributs dans le constructeur que vous définirez. Dans le corps de votre classe, effectuez les opérations suivantes:
\begin{itemize}
	\item Définir une méthode nommée \lstinline{gcd} qui prend comme arguments les deux attributs définis précédemment et qui retourne leur plus grand diviseur commun. Pour cela, vous devez implémenter l'\href{https://fr.wikipedia.org/wiki/Algorithme_d%27Euclide#:~:text=le%20code%5D-,Principe}{algorithme d'Euclide} de façon récursive.
	\begin{conseil}
		L'algorithme d'Euclide fonctionne comme suit:
		\begin{itemize}
			\item Si A = 0 alors gcd(A, B) = B.
			\item Si B = 0 alors gcd(A, B) = A.
			\item Si aucune des conditions ci-dessus n'est remplie, faire appel à \lstinline{gcd} de façon récursive avec pour premier argument \lstinline{B} et pour deuxième \lstinline{R} avec \lstinline{R = A\%B}. \lstinline{R} étant le reste de la division entre \lstinline{A} et \lstinline{B}.
		\end{itemize}
	\end{conseil}
	\item Définir une méthode \lstinline{simplify} qui sera appelée dans le constructeur après l'initialisation des attributs. 
	\item Redéfinir la méthode \lstinline{toString} pour produire une représentation textuelle des instances de \lstinline{Fraction}.
	\item Redéfinir la méthode \lstinline{equals} pour tester l'égalité entre deux fractions. Cette méthode prendra comme argument un objet \lstinline{other} de type \lstinline{Object}.
	\begin{conseil}
		En Java, par défaut toutes les classes héritent d'une classe-mère nommée \lstinline{Object}
	\end{conseil}
	\item Dans la méthode \lstinline{equals}, vérifier que l'instance \lstinline{Fraction} (\lstinline{this}) n'est pas \textbf{identique} à l'objet \lstinline{other}. Si les deux objets sont identiques, renvoyer \lstinline{true}. Vérifier également que les deux objets sont de même type. Dans le cas contraire, renvoyer \lstinline{false}.
	\begin{conseil}
		En Java, pour vérifier que deux objets sont de même type, vous pouvez accéder à la classe de chacun des éléments en utilisant la méthode \lstinline{.getClass())} et les comparer.
	\end{conseil}
	\item Dans le \lstinline{main}, exécuter le code suivant (disponible sur Moodle): 
	\lstinputlisting{ressources/fraction_main.java}.
	\item Pourquoi les lignes 5 et 6 renvoient-elles des résultats différents?	
\end{itemize}

\begin{solution}
	\lstinputlisting{solutions/fraction.java}

	L'instruction \lstinline{f1 == f2} vérifie que les deux objets sont identiques. Dans notre cas, \lstinline{f1} et \lstinline{f2} sont deux objets différents. On peut le vérifier en comparant leur représentation numérique. Cela peut être fait en utilisant la méthode \lstinline{.hashcode()}. Vous pouvez essayer en exécutant la ligne suivante dans le~\lstinline{main}: \lstinline{System.out.println(f1.hashcode())}.

	L'instruction \lstinline{f1.equals(f2)} fait appel à la méthode \lstinline{equals()} de notre classe \lstinline{Fraction} et exécute les instructions que nous avons défini dans cette méthode.
\end{solution}





\end{Exercice}
	

\section{Classes abstraites}
% Ajouter deux exercices sur les classes abstraites

\begin{Exercice}[15 minutes]{Création d'une classe abstraite}

	Une classe abstraite est une classe dont l'implémentation n'est pas complète et qui n'est pas instanciable. Elle est déclarée en utilisant le mot-clé \lstinline{abstract}. Elle peut inclure des méthodes abstraites ou non. Bien que ne pouvant être instanciées, les classes abstraites servent de base à des sous-classes qui en sont dérivées.
	Lorsqu'une sous-classe est dérivée d'une classe abstraite, elle complète généralement l'implémentation de toutes les méthodes abstraites de la classe-mère. Si ce n'est pas le cas, la sous-classe doit également être déclarée comme abstraite.
	
	\lstinputlisting{ressources/exo1.java} 
	
	\begin{itemize}
		\item Créez une classe abstraite appelée \lstinline{Item}. Elle doit avoir 4 attributs d'instance et un attribut de classe, qui sont les suivants:
		\lstinputlisting{ressources/variables_exo1.java} 
		\begin{conseil}
			Les attributs d'instance sont créés lors de l'instanciation d'un objet (à l'aide du mot-clé \lstinline{new}) et détruits lors de la destruction de l'objet. Les attributs de classes (attributs statiques), quant à eux, sont créés lors de l'exécution du programme et détruits lors de l'arrêt du programme. En Java, les attributs de classes sont accessibles en utilisant le nom de la classe soit: \lstinline{ClassName.AttributeName}.
		\end{conseil}
		\item Créez un constructeur pour initialiser les attributs \lstinline{name, price, ingredients et id}. L'attribut \lstinline{id} incrémentera à chaque instanciation de la classe.
		\begin{conseil}
			Pensez à utilisez \lstinline{count} pour initialiser la valeur d'\lstinline{id}. Ainsi, dans le constructeur, \lstinline{id} sera égal à \lstinline{++count}.
		\end{conseil}
		\item Implémentez les \lstinline{getters} des attributs \lstinline{id, name, price et ingredients}.
		\item Implémentez les méthodes \lstinline{equals}(Object o) et \lstinline{toString}().
	
	\begin{conseil}
	La méthode \lstinline{equals} permet de comparer deux objets. Elle prend en entrée un objet de type \lstinline{Object} et doit retourner \lstinline{True} si l'objet instancié est égal à l'objet passé en paramètre.
	\end{conseil}
\end{itemize}
		\begin{solution}  
			\lstinputlisting{solutions/solution1.java} 
		\end{solution}
	\end{Exercice}
	
	
	\begin{Exercice}[15 minutes]{Classe abstraite et types d'attributs}
	\begin{itemize}
		\item Implémentez une classe abstraite \lstinline{Figure} contenant deux attributs protégés : \lstinline{largeur} et \lstinline{longueur} et deux méthodes abstraites : \lstinline{getAire}()et \lstinline{getPerimetre}().
		\item Créez deux classes \lstinline{Carre} et \lstinline{Rectangle} qui héritent de la classe \lstinline{Figure}. À l'intérieur de ces classes, implémentez les méthodes \lstinline{getAire()} et \lstinline{getPerimetre()}.
	\end{itemize}
	
	\begin{conseil}
		Un attribut protégé est accessible aussi bien dans la classe-mère que dans la(les) classe(s)-fille(s). On utilise le mot-clé \lstinline{protected} pour rendre des attributs protégés. \\
		Pour rappel, pour créer une classe fille, on utilise le mot-clé \lstinline{extends}. Par exemple: \lstinline{public class Carre extends Figure}.
	\end{conseil}
	
	
	\begin{solution}
	\lstinputlisting{solutions/Figure.java} 
	\end{solution}
	\end{Exercice}
	\newpage
\section{Interfaces}
\begin{Exercice}[15 minutes]{Interface et héritage (\faLink~~Liée à la question 4)}\\

En Java, une interface se déclare comme suit:
\lstinputlisting{ressources/interface_exemple1.java} 

Les méthodes déclarées dans une interface doivent être implémentées dans des sous-classes :

\lstinputlisting{ressources/interface_exemple2.java} 
\begin{itemize}
	\item Créez une interface \lstinline{Edible} contenant une méthode \lstinline{eatMe} qui ne retourne aucune valeur.
	\item Créez une interface \lstinline{Drinkable} contenant une méthode \lstinline{drinkMe} qui ne retourne aucune valeur.
	\item Créez une classe \lstinline{Food} qui hérite la classe \lstinline{Item} (définie à la question 4) et qui implémente l'interface \lstinline{Edible}. Ecrire le constructeur de \lstinline{Food} et la méthode \lstinline{eatMe} (dans la classe \lstinline{Food}).
\end{itemize}

\begin{conseil}
Vous pouvez reprendre la classe \lstinline{Item} de la question 4.\\
Dans la méthode \lstinline{eatMe()}, vous pouvez simplement afficher un message en utilisant un \lstinline{println}.
\end{conseil}

Certains aliments ne sont pas seulement \lstinline{Edible} (mangeable) mais aussi \lstinline{Drinkable} (buvable) comme les soupes par exemple.\\

4. Créez une classe \lstinline{Soup} qui hérite de \lstinline{Food} et implémente l'interface \lstinline{Drinkable}. Ensuite, écrivez à la fois un constructeur pour \lstinline{Soup} ainsi que la méthode \lstinline{drinkMe} (dans la classe \lstinline{Soup}).\\

Vous pouvez ensuite créer des instances de \lstinline{Soup} et \lstinline{Food} à l'aide des lignes suivantes pour tester les méthodes \lstinline{eatMe()} et \lstinline{drinkMe()}.
\lstinputlisting{ressources/fin_interface.java} 

\begin{solution}
\lstinputlisting{solutions/interface.java} 
\end{solution}
\end{Exercice}

\begin{comment}
\section{Exercices complémentaires}

\begin{Exercice}[10 minutes] \textbf{Programmation de base}\\
	Ecrivez un programme Python qui affiche tous les nombres impairs à partir de 1 jusqu’à un nombre \lstinline{n} défini par l’utilisateur. Ce nombre \lstinline{n} doit être supérieur à 1.
	Exemple : si n = 6, résultat attendu : 1, 3, 5
% Solutions nombres impaire - Maeva
	\begin{solution}
   		 \lstinputlisting{solutions/exercise3.py}
	\end{solution}
\end{Exercice}

% Ajouter des questions sur la récursivité

% Question sur la manipulation avancée de listes et dictionnaires



% Ajouter des questions sur la récursivité

% Question sur la manipulation avancée de listes et dictionnaires

\end{comment}


\end{document}