\documentclass[a4paper]{article}
\usepackage{times}
\usepackage[utf8]{inputenc}
\usepackage{selinput}
\usepackage{upquote}
\usepackage[margin=2cm, rmargin=4cm, tmargin=3cm]{geometry}
\usepackage{tcolorbox}
\usepackage{xspace}
\usepackage[french]{babel}
\usepackage{url}
\usepackage{hyperref}
\usepackage{fontawesome5}
\usepackage{marginnote}
\usepackage{ulem}
\usepackage{tcolorbox}
\usepackage{graphicx}
%\usepackage[top=Bcm, bottom=Hcm, outer=Ccm, inner=Acm, heightrounded, marginparwidth=Ecm, marginparsep=Dcm]{geometry}


\newtcolorbox{Example}[1]{colback=white,left=20pt,colframe=slideblue,fonttitle=\bfseries,title=#1}
\newtcolorbox{Solutions}[1]{colback=white,left=20pt,colframe=green,fonttitle=\bfseries,title=#1}
\newtcolorbox{Conseils}[1]{colback=white,left=20pt,colframe=slideblue,fonttitle=\bfseries,title=#1}
\newtcolorbox{Warning}[1]{colback=white,left=20pt,colframe=warning,fonttitle=\bfseries,title=#1}

\setlength\parindent{0pt}

  %Exercice environment
  \newcounter{exercice}
  \newenvironment{Exercice}[1][]
  {
  \par
  \stepcounter{exercice}\textbf{Question \arabic{exercice}:} (\faClock \enskip \textit{#1})
  }
  {\bigskip}
  

% Title
\newcommand{\titre}{\begin{center}
  \section*{Algorithmes et Pensée Computationnelle}
\end{center}}
\newcommand{\cours}[1]
{\begin{center} 
  \textit{#1}\\
\end{center}
  }


\newcommand{\exemple}[1]{\newline~\textbf{Exemple :} #1}
%\newcommand{\attention}[1]{\newline\faExclamationTriangle~\textbf{Attention :} #1}

% Documentation url (escape \# in the TP document)
\newcommand{\documentation}[1]{\faBookOpen~Documentation : \href{#1}{#1}}

% Clef API
\newcommand{\apikey}[1]{\faKey~Clé API : \lstinline{#1}}
\newcommand{\apiendpoint}[1]{\faGlobe~Url de base de l'API \href{#1}{#1}}

%Listing Python style
\usepackage{color}
\definecolor{slideblue}{RGB}{33,131,189}
\definecolor{green}{RGB}{0,190,100}
\definecolor{blue}{RGB}{121,142,213}
\definecolor{grey}{RGB}{120,120,120}
\definecolor{warning}{RGB}{235,186,1}

\usepackage{listings}
\lstdefinelanguage{texte}{
    keywordstyle=\color{black},
    numbers=none,
    frame=none,
    literate=
           {é}{{\'e}}1
           {è}{{\`e}}1
           {ê}{{\^e}}1
           {à}{{\`a}}1
           {â}{{\^a}}1
           {ù}{{\`u}}1
           {ü}{{\"u}}1
           {î}{{\^i}}1
           {ï}{{\"i}}1
           {ë}{{\"e}}1
           {Ç}{{\,C}}1
           {ç}{{\,c}}1,
    columns=fullflexible,keepspaces,
	breaklines=true,
	breakatwhitespace=true,
}
\lstset{
    language=Python,
	basicstyle=\bfseries\footnotesize,
	breaklines=true,
	breakatwhitespace=true,
	commentstyle=\color{grey},
	stringstyle=\color{slideblue},
  keywordstyle=\color{slideblue},
	morekeywords={with, as, True, False, Float, join, None, main, argparse, self, sort, __eq__, __add__, __ne__, __radd__, __del__, __ge__, __gt__, split, os, endswith, is_file, scandir, @classmethod},
	deletekeywords={id},
	showspaces=false,
	showstringspaces=false,
	columns=fullflexible,keepspaces,
	literate=
           {é}{{\'e}}1
           {è}{{\`e}}1
           {ê}{{\^e}}1
           {à}{{\`a}}1
           {â}{{\^a}}1
           {ù}{{\`u}}1
           {ü}{{\"u}}1
           {î}{{\^i}}1
           {ï}{{\"i}}1
           {ë}{{\"e}}1
           {Ç}{{\,C}}1
           {ç}{{\,c}}1,
    numbers=left,
}

\newtcbox{\mybox}{nobeforeafter,colframe=white,colback=slideblue,boxrule=0.5pt,arc=1.5pt, boxsep=0pt,left=2pt,right=2pt,top=2pt,bottom=2pt,tcbox raise base}
\newcommand{\projet}{\mybox{\textcolor{white}{\small projet}}\xspace}
\newcommand{\optionnel}{\mybox{\textcolor{white}{\small Optionnel}}\xspace}
\newcommand{\advanced}{\mybox{\textcolor{white}{\small Pour aller plus loin}}\xspace}
\newcommand{\auto}{\mybox{\textcolor{white}{\small Auto-évaluation}}\xspace}


\usepackage{environ}
\newif\ifShowSolution
\NewEnviron{solution}{
  \ifShowSolution
	\begin{Solutions}{\faTerminal \enskip Solution}
		\BODY
	\end{Solutions}
  \fi}


  \usepackage{environ}
  \newif\ifShowConseil
  \NewEnviron{conseil}{
    \ifShowConseil
    \begin{Conseils}{\faLightbulb \quad Conseil}
      \BODY
    \end{Conseils}

    \fi}

    \usepackage{environ}
  \newif\ifShowWarning
  \NewEnviron{attention}{
    \ifShowWarning
    \begin{Warning}{\faExclamationTriangle \quad Attention}
      \BODY
    \end{Warning}

    \fi}
  

%\newcommand{\Conseil}[1]{\ifShowIndice\ \newline\faLightbulb[regular]~#1\fi}



\usepackage{listings}
\usepackage{array}
\newcolumntype{C}[1]{>{\centering\let\newline\\\arraybackslash\hspace{0pt}}m{#1}}

\begin{document}

% Change the following values to true to show the solutions or/and the hints
\ShowSolutionfalse
\ShowConseiltrue
\titre
\cours{Programmation de base - Suite}


\section{Bases en programmation}
Le but de cette section est d'écrire vos premières lignes de code. Les notions abordées concerneront les variables, les fonctions, et les interactions avec l'utilisateur (input/output). Vous pouvez les écrire en Java ou en Python.\\
Le temps indiqué (\faClock) est à titre indicatif.\\

\begin{Exercice}[5 minutes] \textbf{Output (Java ou Python)}\\
   Créez une variable \textit{nom} (str) contenant votre nom, et une autre \textit{prenom} (str) contenant votre prénom puis affichez : "Bonjour, \textit{prenom nom}". \\
   
    \begin{conseil}
        Utilisez la fonction \lstinline{print()} de Python et \lstinline{System.out.println()} de Java. 
        
    \end{conseil}
    \begin{solution}
    
    \textbf{Python:} 
    \lstinputlisting{resources/solutions/Question1.py}
    
    \textbf{\\Java:} 
    \lstinputlisting{resources/solutions/Question1.java}  
       
        
    \end{solution}   
\end{Exercice}

\begin{Exercice}[5 minutes] \textbf{Input (Java ou Python)}\\
   En vous référant à l'exercice précédent (\textbf{Output (Java ou Python)}), demandez à l'utilisateur d'entrer son nom et son prénom via la fonction \lstinline{input()} au lieu d'initialiser vous-même les variables. \\
   
    \begin{conseil}
       Utilisez la fonction \lstinline{input()} en Python, la classe \lstinline{Scanner()} en Java (n'oubliez pas d'ajouter \lstinline{import java.util.Scanner;}) tout au début de votre code. N'hésitez pas à vous référer à la diapositive 31 du cours 3.
        
    \end{conseil}
    \begin{solution}
    
    \textbf{Python:} 
    \lstinputlisting{resources/solutions/Question2.py}
    
    \textbf{\\Java:}
    \lstinputlisting{resources/solutions/Question2.java}  
       
        
    \end{solution}   
\end{Exercice}

\begin{Exercice}[5 minutes] \textbf{Format d'impression (Python uniquement)}\\
   Créez et assignez des valeurs à 2 variables \textit{prenom} (str) et \textit{age} (int), puis affichez: "Je m'appelle \textit{prenom} et j'ai \textit{age} ans". Gérez le format de l'impression via l'opérateur +, puis en utilisant la fonction \lstinline{format()}. \\
   
    \begin{conseil}
       N'hésitez pas à consulter ce lien pour plus de détails concernant l'utilisation de la fonction format(): \url{
        https://docs.python.org/fr/3.5/library/stdtypes.html\#str.format}
    \end{conseil}
    \begin{solution}
     
    \lstinputlisting{resources/solutions/Question3.py}
           
    \end{solution}   
\end{Exercice}

\begin{Exercice}[3 minutes] \textbf{Type (Python uniquement)}\\
   Déclarez deux variables \textit{nom} (String) et \textit{age} (int), puis affichez le type de chacune de ces deux variables.
   
    \begin{conseil}
       Vous pouvez contrôler le type de vos variables via la fonction \lstinline{type()}.
        
    \end{conseil}
    \begin{solution}
     
    \lstinputlisting{resources/solutions/Question4.py}
           
    \end{solution}   
\end{Exercice}

\begin{Exercice}[5 minutes] \textbf{Conversion des variables (Type casting) (Java ou Python)}\\
   Il est possible de convertir une variable d'un certain type vers un autre type. Il est par exemple possible de changer un \lstinline{int} en \lstinline{float} ou un \lstinline{float} en \lstinline{int}. Cette opération se nomme le \textit{Type Casting}. \\
   Déclarez une variable \textit{nombre\_entier} de type \lstinline{int}, puis une autre variable \textit{nombre\_decimal} de type \lstinline{float}. Affichez \textit{nombre\_entier} en le convertissant en \lstinline{float} et \textit{nombre\_decimal} en le convertissant en \lstinline{int}. \\
   
    \begin{conseil}
       Utilisez la fonction \lstinline{int(float)} et \lstinline{float(int)} en Python / Utilisez \lstinline{(int) float} et (float) int en Java.
        
    \end{conseil}
    \begin{solution}
    
    \textbf{Python:}
    
    \lstinputlisting{resources/solutions/Question5.py}
    
    \textbf{\\Java:}
    \lstinputlisting{resources/solutions/Question5.java}
           
    \end{solution}   
\end{Exercice}
\begin{Exercice}[5 minutes] \textbf{Conversion des variables (Type casting) (Java ou Python) \optionnel}\\
   Qu'afficheront les programmes suivants ? \\
   
   \textbf{Python:}
   \lstinputlisting{resources/solutions/Question6.py}
   
   \textbf{\\Java:}
   \lstinputlisting{resources/solutions/Question6.java}
    
   
    \begin{conseil}
      	Attention, ces fonctions ne changent pas le type des variables, elles ne font que les convertir.
        
    \end{conseil}
    \begin{solution}
     
    \textbf{Python:}\\
    3\\
    3.0
    
    \textbf{\\Java:}\\
    3\\
    3.14 \\
           
    \end{solution}   
\end{Exercice}
\begin{Exercice}[3 minutes] \textbf{Calculs (multiplication) (Java ou Python)}\\

   Créez 2 variables \textit{facteur\_1} (= 11) et \textit{facteur\_2} (= 3). Multipliez la première variable par la deuxième et stockez le résultat dans une nouvelle variable \textit{produit}. Vous pouvez afficher les différentes variables pour voir leurs valeurs. Vous pouvez répéter l'exercice avec l'addition et la soustraction. \\
   
    \begin{conseil}
      	L'opérateur de multiplication est le *, celui d'addition est le + et celui de soustraction est le -.
        
    \end{conseil}
    \begin{solution}
    
    \textbf{Python:}
    \lstinputlisting{resources/solutions/Question7.py}
    
    \textbf{\\Java:}
    \lstinputlisting{resources/solutions/Question7.java}
           
    \end{solution}   
\end{Exercice}

\begin{Exercice}[10 minutes] \textbf{Calculs (division) (Java ou Python)}\\
    Créez 2 variables \textit{nb\_bonbons} avec pour valeur 11 et \textit{nb\_personnes} avec pour valeur 3. Divisez la première variable par la deuxième et stockez le résultat dans une nouvelle variable \textit{bonbons\_personnes}. Pour finir, calculez le nombre de bonbons restants via l'opérateur \% (modulo) et stockez le résultat dans une nouvelle variable \textit{reste}. Vous pouvez afficher les différentes variables pour voir leurs valeurs. \\
    
     \begin{conseil}
           Attention, en Python il existe 2 opérateurs de division, / effectue une division classique, tandis que // effectue une division entière. En Java, si vous travaillez uniquement avec des int, / effectuera une division entière tandis que si vous travaillez avec au moins un float, / effectuera une division classique. Vous pouvez aussi formater le type du résultat lorsque vous créez une variable.
         
     \end{conseil}
     \begin{solution}
     
     \textbf{Python:}
     
     \lstinputlisting{resources/solutions/Question8.py}
     
     \textbf{\\Java:}
     \lstinputlisting{resources/solutions/Question8.java}
            
     \end{solution}   
 \end{Exercice}
 
\begin{Exercice}[5 minutes] \textbf{Calculs (incrémentation / décrémentation) (Java ou Python)} \optionnel\\
   Gardez vos variables de l'exercice précédent (\textbf{Calculs (division) (Java ou Python)}), augmentez la valeur de \textit{nb\_bonbons} de 1, et diminuez celle de \textit{nb\_personnes} de 1.  \\
   
    \begin{conseil}
      	Vous pouvez utiliser les opérateurs += et -= en Python, et les opérateurs ++ et \textit{- -} en Java.
        
    \end{conseil}
    \begin{solution}
    
    \textbf{Python:}
    \lstinputlisting{resources/solutions/Question9.py}
    
    \textbf{\\Java:}
    \lstinputlisting{resources/solutions/Question9.java}
           
    \end{solution}   
\end{Exercice}

\begin{Exercice}[5 minutes] \textbf{Manipulation des chaînes de caractères (indexation) (Java ou Python)}\\
   Créez une variable \textit{mon\_mot} de type chaîne de caractères avec pour valeur \textbf{``Hard But Cool !!''}. Créez ensuite une variable \textit{premiere} contenant la première lettre de \textit{mon\_mot} en utilisant l'indexation. Créez enfin une variable \textit{derniere} contenant la dernière lettre de \textit{mon\_mot} en utilisant l'indexation. Affichez les résultats. Qu'obtenez-vous? \\
   
    \begin{conseil}
      	Pour Python, utilisez \lstinline{[]}, et pour Java, utilisez la fonction \lstinline{substring()} ainsi que la fonction \lstinline{length()} qui permet d'obtenir la taille d'un élément.
        
    \end{conseil}
    \begin{solution}
    
    \textbf{Python:}
    \lstinputlisting{resources/solutions/Question10.py}
    
    \textbf{\\Java:}
    \lstinputlisting{resources/solutions/Question10.java}
           
    \end{solution}   
\end{Exercice}

\begin{Exercice}[5 minutes] \textbf{Manipulation des chaînes de caractères (indexation 2) (Java ou Python)}\\
   Gardez votre variable, \textit{mon\_mot} et créez une variable \textit{lettre\_5} contenant la cinquième lettre de \textit{mon\_mot} en utilisant l'indexation. Créez ensuite une variable \textit{lettre\_9\_13} contenant les lettres 9, 10, 11, 12, 13 de \textit{mon\_mot}. Afficher les résultats et voyez ce que vous obtenez.  \\
   
    \begin{conseil}
      	Attention, ici les espaces comptent comme des lettres ! \\

		Pour Python, utilisez \lstinline{[:]}, et pour Java, utilisez la fonction \lstinline{substring()}.
        
    \end{conseil}
    \begin{solution}
    
    \textbf{Python:}    
    \lstinputlisting{resources/solutions/Question11.py}
    
    \textbf{\\Java:}
    \lstinputlisting{resources/solutions/Question11.java}
           
    \end{solution}   
\end{Exercice}
\begin{Exercice}[10 minutes] \textbf{Manipulation des chaînes de caractères (Java ou Python) \optionnel}\\
   Il est possible d'obtenir la longueur d'une chaîne de caractère (ou d'une liste ou d'un dictionnaire) en utilisant la fonction \lstinline{len()}. Gardez votre variable \textit{mon\_mot} et créez une nouvelle variable nommée \textit{ln\_mon\_mot} contenant le nombre de caractère de la variable \textit{mon\_mot}, puis une nouvelle variable \textit{moitie} contenant la première moitié de la variable \textit{mon\_mot} (utilisez la variable que vous venez de créer). Affichez le résultat.   \\
   
    \begin{conseil}
      	La fonction présentée dans l'énoncé de la question n'est valable que pour python. L'équivalent pour Java est la fonction \lstinline{length()}.
        
    \end{conseil}
    \begin{solution}
    
    \textbf{Python}:
    \lstinputlisting{resources/solutions/Question12.py}
    
    \textbf{\\Java}:
    \lstinputlisting{resources/solutions/Question12.java}
    \end{solution}   
\end{Exercice}
\begin{Exercice}[5 minutes] \textbf{Les fonctions (fonctions basiques) (Java ou Python))}\\
  Définissez une fonction nommée \lstinline{ping()} qui, lorsqu'elle est appelée, affiche ``pong''. Appelez la plusieurs fois et observez le résultat.  \\
   
    \begin{conseil}
        \begin{itemize}
            \item Référez vous aux diapositives du cours pour la création et l'appel des fonctions. 
            \item Vous pourriez utiliser une boucle \lstinline{for} pour effectuer plusieurs appels à la fonction \lstinline{ping()}.
        \end{itemize}        
    \end{conseil}
    \begin{solution}
    
        \textbf{Python}:
        \lstinputlisting{resources/solutions/Question13.py}
        
        \textbf{\\Java}:
        \lstinputlisting{resources/solutions/Question13.java}
           
    \end{solution}   
\end{Exercice}

\begin{comment}
   \begin{Exercice}[5 minutes] \textbf{Les Fonctions (Fonction multiplication) (Java ou Python)}\\
   Définissez une fonction nommée \lstinline{multiplicateur()} qui prend deux arguments \textit{multiple\_1} et \textit{multiple\_2}, les multiplie et retourne le résultat. Stockez le résultat de \lstinline{multiplicateur(2,3)} dans une variable \textit{resultat} et affichez la.   \\
   
    \begin{conseil}
        \begin{itemize}
            \item Référez vous au cours pour la création et l'appel des fonctions.
            \item Pour retourner une valeur au lieu de l'imprimer, utilisez le mot clé \lstinline{return} (pour Python et Java).
        \end{itemize}        
    \end{conseil}
    \begin{solution}
        \textbf{Python}:
        \lstinputlisting{resources/Question23.py}
        
        \textbf{\\Java}:
        \lstinputlisting{resources/Question23.java}
    \end{solution}   
\end{Exercice} 
\end{comment}

\begin{Exercice}[5 minutes] \textbf{Les Fonctions (Fonctions Aire et Périmètre) (Java ou Python)}\\
    Définissez deux fonctions nommées \lstinline{aire()} et \lstinline{perimètre()} qui prennent un argument (\lstinline{rayon}) et renvoient respectivement l'aire et le périmètre d'un cercle. Stockez les résultats dans des variables \lstinline{aire} et \lstinline{perimetre} et affichez le contenu de ces variables.   \\
   
    \begin{conseil}
        \begin{itemize}
            \item Référez vous au cours pour la création et l'appel des fonctions.
            \item Pour retourner une valeur au lieu de l'imprimer, utilisez le mot clé \lstinline{return} (pour Python et Java).
            \item Pour rappel, le périmètre d'un cercle s'obtient en utilisant la formule $P = 2*\pi*r$ et l'aire s'obtient en utilisant la formule $A = r^2*\pi$.
        \end{itemize}        
    \end{conseil}
    \begin{solution}
        \textbf{Python}:
        \lstinputlisting{resources/solutions/Question14.py}
        
        \textbf{\\Java}:
        \lstinputlisting{resources/solutions/Question14.java}
    \end{solution}   
\end{Exercice} 




\newpage


\end{document}
