

\documentclass[a4paper]{article}
\usepackage{times}
\usepackage[utf8]{inputenc}
\usepackage{selinput}
\usepackage{upquote}
\usepackage[margin=2cm, rmargin=4cm, tmargin=3cm]{geometry}
\usepackage{tcolorbox}
\usepackage{xspace}
\usepackage[french]{babel}
\usepackage{url}
\usepackage{hyperref}
\usepackage{fontawesome5}
\usepackage{marginnote}
\usepackage{ulem}
\usepackage{tcolorbox}
\usepackage{graphicx}
%\usepackage[top=Bcm, bottom=Hcm, outer=Ccm, inner=Acm, heightrounded, marginparwidth=Ecm, marginparsep=Dcm]{geometry}


\newtcolorbox{Example}[1]{colback=white,left=20pt,colframe=slideblue,fonttitle=\bfseries,title=#1}
\newtcolorbox{Solutions}[1]{colback=white,left=20pt,colframe=green,fonttitle=\bfseries,title=#1}
\newtcolorbox{Conseils}[1]{colback=white,left=20pt,colframe=slideblue,fonttitle=\bfseries,title=#1}
\newtcolorbox{Warning}[1]{colback=white,left=20pt,colframe=warning,fonttitle=\bfseries,title=#1}

\setlength\parindent{0pt}

  %Exercice environment
  \newcounter{exercice}
  \newenvironment{Exercice}[1][]
  {
  \par
  \stepcounter{exercice}\textbf{Question \arabic{exercice}:} (\faClock \enskip \textit{#1})
  }
  {\bigskip}
  

% Title
\newcommand{\titre}{\begin{center}
  \section*{Algorithmes et Pensée Computationnelle}
\end{center}}
\newcommand{\cours}[1]
{\begin{center} 
  \textit{#1}\\
\end{center}
  }


\newcommand{\exemple}[1]{\newline~\textbf{Exemple :} #1}
%\newcommand{\attention}[1]{\newline\faExclamationTriangle~\textbf{Attention :} #1}

% Documentation url (escape \# in the TP document)
\newcommand{\documentation}[1]{\faBookOpen~Documentation : \href{#1}{#1}}

% Clef API
\newcommand{\apikey}[1]{\faKey~Clé API : \lstinline{#1}}
\newcommand{\apiendpoint}[1]{\faGlobe~Url de base de l'API \href{#1}{#1}}

%Listing Python style
\usepackage{color}
\definecolor{slideblue}{RGB}{33,131,189}
\definecolor{green}{RGB}{0,190,100}
\definecolor{blue}{RGB}{121,142,213}
\definecolor{grey}{RGB}{120,120,120}
\definecolor{warning}{RGB}{235,186,1}

\usepackage{listings}
\lstdefinelanguage{texte}{
    keywordstyle=\color{black},
    numbers=none,
    frame=none,
    literate=
           {é}{{\'e}}1
           {è}{{\`e}}1
           {ê}{{\^e}}1
           {à}{{\`a}}1
           {â}{{\^a}}1
           {ù}{{\`u}}1
           {ü}{{\"u}}1
           {î}{{\^i}}1
           {ï}{{\"i}}1
           {ë}{{\"e}}1
           {Ç}{{\,C}}1
           {ç}{{\,c}}1,
    columns=fullflexible,keepspaces,
	breaklines=true,
	breakatwhitespace=true,
}
\lstset{
    language=Python,
	basicstyle=\bfseries\footnotesize,
	breaklines=true,
	breakatwhitespace=true,
	commentstyle=\color{grey},
	stringstyle=\color{slideblue},
  keywordstyle=\color{slideblue},
	morekeywords={with, as, True, False, Float, join, None, main, argparse, self, sort, __eq__, __add__, __ne__, __radd__, __del__, __ge__, __gt__, split, os, endswith, is_file, scandir, @classmethod},
	deletekeywords={id},
	showspaces=false,
	showstringspaces=false,
	columns=fullflexible,keepspaces,
	literate=
           {é}{{\'e}}1
           {è}{{\`e}}1
           {ê}{{\^e}}1
           {à}{{\`a}}1
           {â}{{\^a}}1
           {ù}{{\`u}}1
           {ü}{{\"u}}1
           {î}{{\^i}}1
           {ï}{{\"i}}1
           {ë}{{\"e}}1
           {Ç}{{\,C}}1
           {ç}{{\,c}}1,
    numbers=left,
}

\newtcbox{\mybox}{nobeforeafter,colframe=white,colback=slideblue,boxrule=0.5pt,arc=1.5pt, boxsep=0pt,left=2pt,right=2pt,top=2pt,bottom=2pt,tcbox raise base}
\newcommand{\projet}{\mybox{\textcolor{white}{\small projet}}\xspace}
\newcommand{\optionnel}{\mybox{\textcolor{white}{\small Optionnel}}\xspace}
\newcommand{\advanced}{\mybox{\textcolor{white}{\small Pour aller plus loin}}\xspace}
\newcommand{\auto}{\mybox{\textcolor{white}{\small Auto-évaluation}}\xspace}


\usepackage{environ}
\newif\ifShowSolution
\NewEnviron{solution}{
  \ifShowSolution
	\begin{Solutions}{\faTerminal \enskip Solution}
		\BODY
	\end{Solutions}
  \fi}


  \usepackage{environ}
  \newif\ifShowConseil
  \NewEnviron{conseil}{
    \ifShowConseil
    \begin{Conseils}{\faLightbulb \quad Conseil}
      \BODY
    \end{Conseils}

    \fi}

    \usepackage{environ}
  \newif\ifShowWarning
  \NewEnviron{attention}{
    \ifShowWarning
    \begin{Warning}{\faExclamationTriangle \quad Attention}
      \BODY
    \end{Warning}

    \fi}
  

%\newcommand{\Conseil}[1]{\ifShowIndice\ \newline\faLightbulb[regular]~#1\fi}


\usepackage{array}
\newcolumntype{C}[1]{>{\centering\let\newline\\\arraybackslash\hspace{0pt}}m{#1}}

\begin{document}
% Change the following values to true to show the solutions or/and the hints
\ShowSolutionfalse
\ShowConseiltrue
\titre
\cours{Programmation orientée objet: Héritage et Polymorphisme - Exercices avancés}

Le but de cette séance est d'approfondir les notions de programmation orientée objet vues précédemment. Les exercices sont construits autour des concepts d'héritage, de surcharge d'opérateurs/méthodes et de polymorphisme. Au terme de cette séance, vous devez être en mesure de factoriser votre code afin de le rendre mieux structuré et plus lisible. 

Le code présenté dans les énoncés se trouve sur Moodle, dans le dossier \lstinline{Code}.

\section{Héritage en Python }
\begin{Exercice}[15 minutes]\textbf{Classe Point (Suite)}

Lors de la séance précédente, vous avez défini une classe \lstinline{Point} représentant un point de 2~dimensions, avec des coordonnées \lstinline{x} et \lstinline{y}, ainsi que des opérations basiques sur des points 2D. Pour cet exercice, vous allez implémenter une classe de points à 3 dimensions qui héritera de la classe \lstinline{Point} créée précédemment. 
%À travers cet exercice, nous voudrions également vous présenter une syntaxe en Python qui concerne l'utilisation des 'décorateurs'.\\


Voici la classe \lstinline{Point} qui a été légèrement modifiée (Vous trouverez le fichier sur Moodle, dans le dossier~\lstinline{Code}):
\lstinputlisting{ressources/Point.py} 

Écrivez une classe qui hérite de \lstinline{Point}. Nommez la \lstinline{Point3D}. Après avoir rajouté la 3ème dimension comme attribut, effectuez les opérations ci-dessous:

\begin{itemize}
	\item Rajoutez une méthode qui renvoie une représentation vectorielle du point. Vous pouvez utiliser une liste.
	\item Recalculez la distance euclidienne et le milieu pour le point 3D.
	\item \advanced Si vous voulez vous familiariser encore plus avec les méthodes de classe en Python, implémentez deux autres algorithmes de calculs de distance: les distances de \href{https://fr.wikipedia.org/wiki/Distance_de_Manhattan}{\textcolor{cyan}{Manhattan}} et \href{https://en.wikipedia.org/wiki/Minkowski_distance}{\textcolor{cyan}{Minkowski}}.
\end{itemize}
\lstinputlisting{ressources/Point3D.py} 

\begin{conseil}
%Recherchez sur Internet des documentations sur les décorateurs Python, notamment sur 'property', 'setter' et 'getter'. 

Que fait \lstinline{super().__init__()}?\\
Dans un espace à 3 dimensions, la formule pour calculer la distance entre un $p_1 = (x_1, y_1, z_1)$ et $p_2 = (x_2, y_2, z_2)$ est $\sqrt{(x_1 - x_2)^2 + (y_1 - y_2)^2 + (y_1 - y_2)^2}$.
\end{conseil}
\begin{solution}
\lstinputlisting{Solutions/Point3D.py} 
\end{solution}
\end{Exercice} 

\begin{Exercice}[15 minutes]\textbf{Un exemple appliqué}

Dans les établissements universitaires, on rencontre souvent des problèmes lors du calcul de salaires du personnel. Sans penser aux recherches effectuées par certains professeurs, on va essayer de calculer les salaires de ceux qui sont reconnus comme `Professeur' (ordinaire, titulaire, associé ou assistant) à l'université et ceux qui y donnent des cours à temps partiel (on va les considérer comme `Collaborateurs' dans cet exercice).

La classe mère dans ce cas est nommée \lstinline{Enseignant}, qui possède une propriété - le salaire annuel moyen. On voudrait que la méthode qui calcule cette quantité renvoie 60 000 (dollars américains) si l'enseignant a moins de 10 ans d'expérience, et 100 000 sinon. Si l'enseignant travaille à temps partiel, la méthode devrait renvoyer une chaîne qui dit `Le salaire annuel ne s'applique pas aux collaborateurs'.

Ensuite, on veut calculer la paye mensuelle pour chaque type d'employé. Pour les \lstinline{Professeurs}, la paye devrait être calculée sur la base de deux sources de revenu: un salaire mensuel et une commission pour chaque comité où ils participent. 

D'autre part, pour les \lstinline{Collaborateurs}, la paye est calculée sur une base horaire i.e \textit{taux horaire $\times$ nombres d'heures de travail (par mois)}.


Complétez le code ci-dessous:

\lstinputlisting{ressources/Enseignant.py} 

\begin{conseil}
Pensez à redéfinir les attributs de la classe mère en utilisant \lstinline{super().__init__()}.
\end{conseil}

\begin{solution}
\lstinputlisting{Solutions/Enseignant.py} 
\end{solution}
\end{Exercice}
%\end{comment}

\end{document}
