\documentclass[a4paper]{article}
\usepackage{times}
\usepackage[utf8]{inputenc}
\usepackage{selinput}
\usepackage{upquote}
\usepackage[margin=2cm, rmargin=4cm, tmargin=3cm]{geometry}
\usepackage{tcolorbox}
\usepackage{xspace}
\usepackage[french]{babel}
\usepackage{url}
\usepackage{hyperref}
\usepackage{fontawesome5}
\usepackage{marginnote}
\usepackage{ulem}
\usepackage{tcolorbox}
\usepackage{graphicx}
%\usepackage[top=Bcm, bottom=Hcm, outer=Ccm, inner=Acm, heightrounded, marginparwidth=Ecm, marginparsep=Dcm]{geometry}


\newtcolorbox{Example}[1]{colback=white,left=20pt,colframe=slideblue,fonttitle=\bfseries,title=#1}
\newtcolorbox{Solutions}[1]{colback=white,left=20pt,colframe=green,fonttitle=\bfseries,title=#1}
\newtcolorbox{Conseils}[1]{colback=white,left=20pt,colframe=slideblue,fonttitle=\bfseries,title=#1}
\newtcolorbox{Warning}[1]{colback=white,left=20pt,colframe=warning,fonttitle=\bfseries,title=#1}

\setlength\parindent{0pt}

  %Exercice environment
  \newcounter{exercice}
  \newenvironment{Exercice}[1][]
  {
  \par
  \stepcounter{exercice}\textbf{Question \arabic{exercice}:} (\faClock \enskip \textit{#1})
  }
  {\bigskip}
  

% Title
\newcommand{\titre}{\begin{center}
  \section*{Algorithmes et Pensée Computationnelle}
\end{center}}
\newcommand{\cours}[1]
{\begin{center} 
  \textit{#1}\\
\end{center}
  }


\newcommand{\exemple}[1]{\newline~\textbf{Exemple :} #1}
%\newcommand{\attention}[1]{\newline\faExclamationTriangle~\textbf{Attention :} #1}

% Documentation url (escape \# in the TP document)
\newcommand{\documentation}[1]{\faBookOpen~Documentation : \href{#1}{#1}}

% Clef API
\newcommand{\apikey}[1]{\faKey~Clé API : \lstinline{#1}}
\newcommand{\apiendpoint}[1]{\faGlobe~Url de base de l'API \href{#1}{#1}}

%Listing Python style
\usepackage{color}
\definecolor{slideblue}{RGB}{33,131,189}
\definecolor{green}{RGB}{0,190,100}
\definecolor{blue}{RGB}{121,142,213}
\definecolor{grey}{RGB}{120,120,120}
\definecolor{warning}{RGB}{235,186,1}

\usepackage{listings}
\lstdefinelanguage{texte}{
    keywordstyle=\color{black},
    numbers=none,
    frame=none,
    literate=
           {é}{{\'e}}1
           {è}{{\`e}}1
           {ê}{{\^e}}1
           {à}{{\`a}}1
           {â}{{\^a}}1
           {ù}{{\`u}}1
           {ü}{{\"u}}1
           {î}{{\^i}}1
           {ï}{{\"i}}1
           {ë}{{\"e}}1
           {Ç}{{\,C}}1
           {ç}{{\,c}}1,
    columns=fullflexible,keepspaces,
	breaklines=true,
	breakatwhitespace=true,
}
\lstset{
    language=Python,
	basicstyle=\bfseries\footnotesize,
	breaklines=true,
	breakatwhitespace=true,
	commentstyle=\color{grey},
	stringstyle=\color{slideblue},
  keywordstyle=\color{slideblue},
	morekeywords={with, as, True, False, Float, join, None, main, argparse, self, sort, __eq__, __add__, __ne__, __radd__, __del__, __ge__, __gt__, split, os, endswith, is_file, scandir, @classmethod},
	deletekeywords={id},
	showspaces=false,
	showstringspaces=false,
	columns=fullflexible,keepspaces,
	literate=
           {é}{{\'e}}1
           {è}{{\`e}}1
           {ê}{{\^e}}1
           {à}{{\`a}}1
           {â}{{\^a}}1
           {ù}{{\`u}}1
           {ü}{{\"u}}1
           {î}{{\^i}}1
           {ï}{{\"i}}1
           {ë}{{\"e}}1
           {Ç}{{\,C}}1
           {ç}{{\,c}}1,
    numbers=left,
}

\newtcbox{\mybox}{nobeforeafter,colframe=white,colback=slideblue,boxrule=0.5pt,arc=1.5pt, boxsep=0pt,left=2pt,right=2pt,top=2pt,bottom=2pt,tcbox raise base}
\newcommand{\projet}{\mybox{\textcolor{white}{\small projet}}\xspace}
\newcommand{\optionnel}{\mybox{\textcolor{white}{\small Optionnel}}\xspace}
\newcommand{\advanced}{\mybox{\textcolor{white}{\small Pour aller plus loin}}\xspace}
\newcommand{\auto}{\mybox{\textcolor{white}{\small Auto-évaluation}}\xspace}


\usepackage{environ}
\newif\ifShowSolution
\NewEnviron{solution}{
  \ifShowSolution
	\begin{Solutions}{\faTerminal \enskip Solution}
		\BODY
	\end{Solutions}
  \fi}


  \usepackage{environ}
  \newif\ifShowConseil
  \NewEnviron{conseil}{
    \ifShowConseil
    \begin{Conseils}{\faLightbulb \quad Conseil}
      \BODY
    \end{Conseils}

    \fi}

    \usepackage{environ}
  \newif\ifShowWarning
  \NewEnviron{attention}{
    \ifShowWarning
    \begin{Warning}{\faExclamationTriangle \quad Attention}
      \BODY
    \end{Warning}

    \fi}
  

%\newcommand{\Conseil}[1]{\ifShowIndice\ \newline\faLightbulb[regular]~#1\fi}


\usepackage{array}
\newcolumntype{C}[1]{>{\centering\let\newline\\\arraybackslash\hspace{0pt}}m{#1}}

\begin{document}
% Change the following values to true to show the solutions or/and the hints
\ShowSolutiontrue
\ShowConseiltrue
\ShowNotefalse

\titre
\cours{Architecture des ordinateurs - Exercices avancés}


\begin{comment}
\begin{enumerate}
    \item Recherche séquentielle
    \item Recherche binaire
    \item Arbres de recherche binaire\\
\end{enumerate}
\end{comment}

Le but de cette séance est de comprendre le fonctionnement d'un ordinateur. La série d'exercices sera axée autour de de conversions en base binaire, décimale ou hexadécimal, de calcul de base en suivant le modèle Von Neumann. \\
Cette feuille d'exercices avancés vous permettra d'approfondir vos connaissances des notions vues en cours.

Le code présenté dans les énoncés se trouve sur Moodle, dans le dossier \lstinline{Ressources}.\\\\


\begin{Exercice}[20 minutes] \textbf{Conversion et addition:}\\
    Effectuer les opérations suivantes:
    \begin{enumerate}
        \item 111101$_{(2)}$ + 110$_{(2)}$ = ...$_{(10)}$
        \item 111111$_{(2)}$ + 000001$_{(2)}$ = ...$_{(10)}$
        \item 127$_{(10)}$ + ABC$_{(16)}$ = ...$_{(10)}$
    \end{enumerate}
    \begin{conseil}
        Calculez à l'aide du tableau d'addition binaire ci-dessus ou une autre méthode que vous préférez.\\
        Additionnez les nombres lorsqu'ils sont dans la même base puis convertissez les dans la base souhaitée.\\

        Exemple de conversion : 1001000$_{(2)}$ = 72$_{(10)}$\\
\begin{tabular}
    {| C{0.15\textwidth} | C{0.07\textwidth} | C{0.07\textwidth} | C{0.07\textwidth} | C{0.07\textwidth} | C{0.07\textwidth} | C{0.07\textwidth} | C{0.07\textwidth} | C{0.06\textwidth} |} 
            \hline
            Base$_{(2)}$ & \textbf{1} & \textbf{0} & \textbf{0} & \textbf{1} & \textbf{0} & \textbf{0} & \textbf{0} & \textbf{}\\ [0.5ex]
            \hline
            Position$_{(2)}$ & $2^6$ & $2^5$ & $2^4$ & $2^3$ & $2^2$ & $2^1$ & $2^0$ & \\ [0.5ex] 
            \hline
             & $\downarrow$ & $\downarrow$ & $\downarrow$ & $\downarrow$ & $\downarrow$ & $\downarrow$ & $\downarrow$ & \\ [0.5ex] 
            \hline
            Équivalent$_{(10)}$ & 1 x $2^6$ & 0 x $2^5$ & 0 x $2^4$ & 1 x $2^3$ & 0 x $2^2$ & 0 x $2^1$ & 0 x $2^0$ & \\ [0.5ex]     
            \hline
            Valeurs$_{(10)}$ & 64 & 0 & 0 & 8 & 0 & 0 & 0 & = \textbf{72} \\ [0.5ex]
            \hline
\end{tabular}
\\\\\\
\textbf{Rappel:} Les valeurs en hexadécimale (base 16)\\
\begin{tabular}{| C{0.030\textwidth} | C{0.030\textwidth} | C{0.030\textwidth} | C{0.030\textwidth} | C{0.030\textwidth} | C{0.030\textwidth} | C{0.030\textwidth} | C{0.030\textwidth} | C{0.030\textwidth} | C{0.030\textwidth} | C{0.030\textwidth} | C{0.030\textwidth} | C{0.030\textwidth} | C{0.030\textwidth} | C{0.030\textwidth} | C{0.030\textwidth} |} 
            \hline
            \textbf{0} & \textbf{1} & \textbf{2} & \textbf{3} & \textbf{4} & \textbf{5} & \textbf{6} & \textbf{7} & \textbf{8} & \textbf{9} & \textbf{A} & \textbf{B} & \textbf{C} & \textbf{D} & \textbf{E} & \textbf{F}\\ [0.5ex]
            \hline
            0 & 1 & 2 & 3 & 4 & 5 & 6 & 7 & 8 & 9 & 10 & 11 & 12 & 13 & 14 & 15 \\ [0.5ex] 
            \hline
\end{tabular}
\\\\\\
Exemple de conversion : 3BF$_{(16)}$ = 959$_{(10)}$\\
    \begin{tabular}{| C{0.15\textwidth} | C{0.15\textwidth} | C{0.15\textwidth} | C{0.15\textwidth} | C{0.1\textwidth} |} 
            \hline
            Base$_{(16)}$ & \textbf{3} & \textbf{B} & \textbf{F} & \textbf{}\\ [0.5ex]
            \hline
             Valeurs$_{(16)}$ & 3 & 11 & 15 & \\ [0.5ex]
             \hline
             Position$_{(16)}$ & $16^2$ & $16^1$ & $16^0$ & \\ [0.5ex] 
            \hline
             & $\downarrow$ & $\downarrow$ & $\downarrow$ & \\ [0.5ex] 
            \hline
            Équivalent$_{(10)}$ & 3 x $16^2$ & 11 x $16^1$ & 15 x $16^0$ &\\ [0.5ex]     
            \hline
            Valeurs$_{(10)}$ & 768 & 176 & 15 & = \textbf{959} \\ [0.5ex]
            \hline
    \end{tabular}
\\\\\\
    Exemple de conversion : 123$_{(10)}$ = ...$_{(8)}$\\
        
        $8^0 = 1 < 123$\qquad
        $8^1 = 8 < 123$\qquad
        $8^2 = 64 < 123$\qquad
        $8^3 = 512 > 123$\\
        
        $123 / 8 = 15$ avec un reste de \textbf{3}\\
        $15 / 8 = 1$ avec un reste de \textbf{7}\\
        $1 / 8 = 0$ avec un reste de \textbf{1}\\\
        
        \textbf{123$_{(10)}$ = 173$_{(8)}$}\\


    \end{conseil}
    \begin{solution}
        \textbf{6.1}\\
        Calcul du résultat en base 2:\\
        111101$_{(2)}$ + 110$_{(2)}$ = 1000011$_{(2)}$\\\\
        Conversion en base décimale:\\
        1000011$_{(2)}$ = 67$_{(10)}$\\\\
        Réponse :\\
        111101$_{(2)}$ + 110$_{(2)}$ = \textbf{67$_{(10)}$}\\
        
        \begin{tabular}{| C{0.15\textwidth} | C{0.07\textwidth} | C{0.07\textwidth} | C{0.07\textwidth} | C{0.07\textwidth} | C{0.07\textwidth} | C{0.07\textwidth} | C{0.07\textwidth} | C{0.06\textwidth} |} 
            \hline
            Base$_{(2)}$ & \textbf{1} & \textbf{0} & \textbf{0} & \textbf{0} & \textbf{0} & \textbf{1} & \textbf{1} & \textbf{}\\ [0.5ex]
            \hline
             & $2^6$ & $2^5$ & $2^4$ & $2^3$ & $2^2$ & $2^1$ & $2^0$ & \\ [0.5ex] 
            \hline
             & $\downarrow$ & $\downarrow$ & $\downarrow$ & $\downarrow$ & $\downarrow$ & $\downarrow$ & $\downarrow$ & \\ [0.5ex] 
            \hline
            Équivalent$_{(10)}$ & 1 x $2^6$ & 0 x $2^5$ & 0 x $2^4$ & 0 x $2^3$ & 0 x $2^2$ & 1 x $2^1$ & 1 x $2^0$ & \\ [0.5ex]     
            \hline
            Valeurs$_{(10)}$ & 64 & 0 & 0 & 0 & 0 & 2 & 1 & = 67 \\ [0.5ex]
            \hline
        \end{tabular}
        \\\\\\
        \textbf{6.2}\\
        Calcul du résultat en base 2:\\
        111111$_{(2)}$ + 000001$_{(2)}$ = 1000000$_{(2)}$\\\\
        Conversion en base décimale:\\
        1000000$_{(2)}$ = 64$_{(10)}$\\\\
        Réponse :\\
        111111$_{(2)}$ + 000001$_{(2)}$ = \textbf{64$_{(10)}$}\\
        
        \textbf{6.3}\\
        ABC$_{(16)}$ = 2748$_{(10)}$ (Conversion en base 10)\\
        \begin{tabular}{| C{0.15\textwidth} | C{0.08\textwidth} | C{0.08\textwidth} | C{0.08\textwidth} | C{0.08\textwidth} |} 
            \hline
            Base$_{(16)}$ & \textbf{A} & \textbf{B} & \textbf{C} & \textbf{}\\ [0.5ex]
            \hline
             & 10 & 11 & 12 & \\ [0.5ex]
             \hline
             & $16^2$ & $16^1$ & $16^0$ & \\ [0.5ex] 
            \hline
             & $\downarrow$ & $\downarrow$ & $\downarrow$ & \\ [0.5ex] 
            \hline
            Équivalent$_{(10)}$ & 10 x $16^2$ & 11 x $16^1$ & 12 x $16^0$ &\\ [0.5ex]     
            \hline
            Valeurs$_{(10)}$ & 2560 & 176 & 12 & = 2748 \\ [0.5ex]
            \hline
        \end{tabular}

        127$_{(10)}$ + 2748$_{(10)}$ = 2875$_{(10)}$ = 2875\\
        En commençant par zéro, augmentez 8 à des puissances entières de plus en plus grandes jusqu'à ce que le résultat dépasse 2875.\\

        \begin{tabular}{| C{0.08\textwidth} | C{0.08\textwidth} | C{0.08\textwidth} | C{0.08\textwidth} | C{0.08\textwidth} | C{0.08\textwidth} |} 
            \hline
            Entier & 4 & 3 & 2 & 1 & 0\\ [0.5ex]
            \hline
             & $8^4$ & $8^3$ & $8^2$ & $8^1$ & $8^0$\\ [0.5ex]
            \hline
             & $\downarrow$ & $\downarrow$ & $\downarrow$ & $\downarrow$ & $\downarrow$\\ [0.5ex] 
            \hline
             Valeurs & 4096 & 512 & 64 & 8 & 1\\ [0.5ex] 
            \hline
        \end{tabular}
        \\\\

        Déterminez les puissances de 8 qui seront utilisées pour placer les chiffres dans la représentation en base 8.\\

        $8^3$   $8^2$   $8^1$   $8^0$ \\

    
    \end{solution}
    \newpage
    \begin{solution}
        Déterminez la valeur du premier chiffre en partant de la droite (correspondant à $8^0$) grâce au reste de la division entière.\\
        2875 / 8 = 359 avec un reste de \textbf{3}\\\\
        Divisez la partie numérique entière du quotient précédent, 359, par 8 et trouvez le reste. Le reste est le chiffre suivant (correspondant à $8^1$):\\
        359 / 8 = 44 avec un reste de \textbf{7}\\\\
        Ainsi de suite... la valeur pour $8^2$:\\
        44 / 8 = 5 avec un reste de \textbf{4}\\\\
        La valeur pour $8^3$:\\
        5 / 8 = 0 avec un reste de \textbf{5}\\

        2875$_{(10)}$ = \textbf{5473$_{(8)}$}    
    
    \end{solution}
\end{Exercice}

\begin{Exercice}[20 minutes] \textbf{Conversion et soustraction:}\\
    Effectuer les opérations suivantes:
    \begin{enumerate}
        \item 101010$_{(2)}$ - 010101$_{(2)}$ = ...$_{(10)}$
        \item 64$_{(10)}$ - 001000$_{(2)}$ = ...$_{(10)}$
        \item FFF$_{(16)}$ - 127$_{(10)}$ = ...$_{(2)}$
    \end{enumerate}
    \begin{conseil}
        Calculez à l'aide du tableau de soustraction binaire ci-dessus ou une autre méthode que vous préférez.
            
            Additionnez les nombres lorsqu'ils sont dans la même base puis convertissez les dans la base souhaitée.\\
            
            Exemple de conversion : 1001000$_{(2)}$ = 72$_{(10)}$\\
            \begin{tabular}
                {| C{0.15\textwidth} | C{0.07\textwidth} | C{0.07\textwidth} | C{0.07\textwidth} | C{0.07\textwidth} | C{0.07\textwidth} | C{0.07\textwidth} | C{0.07\textwidth} | C{0.06\textwidth} |} 
                \hline
                Base$_{(2)}$ & \textbf{1} & \textbf{0} & \textbf{0} & \textbf{1} & \textbf{0} & \textbf{0} & \textbf{0} & \textbf{}\\ [0.5ex]
                \hline
                Position$_{(2)}$ & $2^6$ & $2^5$ & $2^4$ & $2^3$ & $2^2$ & $2^1$ & $2^0$ & \\ [0.5ex] 
                \hline
                 & $\downarrow$ & $\downarrow$ & $\downarrow$ & $\downarrow$ & $\downarrow$ & $\downarrow$ & $\downarrow$ & \\ [0.5ex] 
                \hline
                Équivalent$_{(10)}$ & 1 x $2^6$ & 0 x $2^5$ & 0 x $2^4$ & 1 x $2^3$ & 0 x $2^2$ & 0 x $2^1$ & 0 x $2^0$ & \\ [0.5ex]     
                \hline
                Valeurs$_{(10)}$ & 64 & 0 & 0 & 8 & 0 & 0 & 0 & = \textbf{72} \\ [0.5ex]
                \hline
            \end{tabular}\\\\

            Rappel: Les valeurs en hexadécimale (base 16)\\
            \begin{tabular}{| C{0.030\textwidth} | C{0.030\textwidth} | C{0.030\textwidth} | C{0.030\textwidth} | C{0.030\textwidth} | C{0.030\textwidth} | C{0.030\textwidth} | C{0.030\textwidth} | C{0.030\textwidth} | C{0.030\textwidth} | C{0.030\textwidth} | C{0.030\textwidth} | C{0.030\textwidth} | C{0.030\textwidth} | C{0.030\textwidth} | C{0.030\textwidth} |} 
            \hline
            \textbf{0} & \textbf{1} & \textbf{2} & \textbf{3} & \textbf{4} & \textbf{5} & \textbf{6} & \textbf{7} & \textbf{8} & \textbf{9} & \textbf{A} & \textbf{B} & \textbf{C} & \textbf{D} & \textbf{E} & \textbf{F}\\ [0.5ex]
            \hline
            0 & 1 & 2 & 3 & 4 & 5 & 6 & 7 & 8 & 9 & 10 & 11 & 12 & 13 & 14 & 15 \\ [0.5ex] 
            \hline
        \end{tabular}\\\\
          Exemple de conversion : 123$_{(10)}$ = ...$_{(2)}$\\
        $2^0 = 1 < 123$\qquad
        $2^1 = 2 < 123$\qquad
        $2^2 = 4 < 123$\qquad
        $2^3 = 8 < 123$\qquad
        $2^4 = 16 < 123$\qquad
        $2^5 = 32 < 123$\qquad
        $2^6 = 64 < 123$\qquad
        $2^7 = 128 > 123$\\
        
        $123 / 2 = 61$ avec un reste de \textbf{1} (premier chiffre en partant de droite)\\
        $61 / 2 = 30$ avec un reste de \textbf{1}\\
        $30 / 2 = 15$ avec un reste de \textbf{0}\\
        $15 / 2 = 7$ avec un reste de \textbf{1}\\
        $7 / 2 = 3$ avec un reste de \textbf{1}\\
        $3 / 2 = 1$ avec un reste de \textbf{1}\\
        $1 / 2 = 0$ avec un reste de \textbf{1} (dernier chiffre en partant de droite)\\
        123$_{(10)}$ = \textbf{1111011$_{(2)}$}
    \end{conseil}
    \begin{solution} \textbf{7.1}\\
        Calcul du résultat en base 2:\\
        101010$_{(2)}$ - 010101$_{(2)}$ = 010101$_{(2)}$\\\\
        Conversion en base décimale:\\
        010101$_{(2)}$ = 21$_{(10)}$\\\\
        Réponse:\\
        101010$_{(2)}$ - 010101$_{(2)}$ = \textbf{21$_{(10)}$}\\\\
    
         \textbf{7.2}\\
        Conversion de 001000$_{(2)}$ en base 10 :\\
        001000$_{(2)}$ = $2^3$ = 8$_{(10)}$\\
        
        Calcul du résultat en base 10:\\
        64$_{(10)}$ - 8$_{(10)}$ = 56$_{(10)}$\\
        
        Réponse : \\
        64$_{(10)}$ - 001000$_{(2)}$ = \textbf{56$_{(10)}$}\\\\

        \textbf{7.3}\\
        Conversion de FFF$_{(16)}$ en base 10:\\
        FFF$_{(16)}$ = 4095$_{(10)}$\\\\
        Calcul en base décimale:\\
        4095$_{(10)}$ - 127$_{(10)}$ = 3968$_{(10)}$\\\\
        Convertir 3968$_{(10)}$ en base binaire :\\
        En commençant par zéro, augmentez 2 à des puissances entières de plus en plus grandes jusqu'à ce que le résultat dépasse 3968.\\
        Déterminez les puissances de 2 qui seront utilisées comme les places des chiffres dans la représentation en base 2 de 3968 :
        \begin{tabular}
            {| C{0.06\textwidth} | C{0.06\textwidth} | C{0.06\textwidth} | C{0.04\textwidth} | C{0.04\textwidth} | C{0.04\textwidth} | C{0.035\textwidth} | C{0.035\textwidth} | C{0.035\textwidth} | C{0.035\textwidth} | C{0.035\textwidth} | C{0.035\textwidth} | C{0.035\textwidth} | C{0.035\textwidth} |} 
            \hline
             $2^{12}$ & $2^{11}$ & $2^{10}$ & $2^9$ & $2^8$ & $2^7$ & $2^6$ & $2^5$ & $2^4$ & $2^{3}$ & $2^2$ & $2^1$ & $2^0$\\ [0.5ex]
            \hline
            $\downarrow$ & $\downarrow$ & $\downarrow$ & $\downarrow$ & $\downarrow$ & $\downarrow$ & $\downarrow$ & $\downarrow$ & $\downarrow$ & $\downarrow$ & $\downarrow$ & $\downarrow$ & $\downarrow$\\ [0.5ex] 
            \hline
            4096 & 2048 & 1024 & 512 & 256 & 128 & 64 & 32 & 16 & 8 & 4 & 2 & 1\\ [0.5ex]
            \hline
        \end{tabular}
        Déterminez la valeur du premier chiffre en partant de la droite (correspondant à $2^0$) grâce au reste de la division entière.\\
        3968 / 2 = 1984 avec un reste de \textbf{0}\\
        Divisez la partie numérique entière du quotient précédent, 1984, par 2 et trouvez le reste. Le reste est le chiffre suivant (correspondant à $2^1$):\\
        1984 / 2 = 992 avec un reste de \textbf{0}\\
        Ainsi de suite... Pour $2^2$\\
        992 / 2 = 496 avec un reste de \textbf{0}\\
        Pour $2^3$\\
        496 / 2 = 258 avec un reste de \textbf{0}\\
        Pour $2^4$\\
        258 / 2 = 124 avec un reste de \textbf{0}\\
        Pour $2^5$\\
        124 / 2 = 62 avec un reste de \textbf{0}\\
        Pour $2^6$\\
        62 / 2 = 31 avec un reste de \textbf{0}\\
        Pour $2^7$\\
        31 / 2 = 15 avec un reste de \textbf{1}\\
        Pour $2^8$\\
        

    \end{solution}

    \begin{solution} 
        15 / 2 = 7 avec un reste de \textbf{1}\\
        Pour $2^9$\\ 
        7 / 2 = 3 avec un reste de \textbf{1}\\
        Pour $2^10$\\
        3 / 2 = 1 avec un reste de \textbf{1}\\
        Pour $2^11$\\
        1 / 2 = 0 avec un reste de \textbf{1}\\\\

        3968$_{(10)}$ = 111110000000$_{(2)}$\\\\

        Réponse:\\
        FFF$_{(16)}$ - 127$_{(10)}$ = \textbf{111110000000$_{(2)}$}
    \end{solution}

    
\end{Exercice}

\end{document}
