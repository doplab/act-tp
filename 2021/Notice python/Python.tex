\documentclass[a4paper]{article}
\usepackage{times}
\usepackage[utf8]{inputenc}
\usepackage{selinput}
\usepackage{upquote}
\usepackage[margin=2cm, rmargin=4cm, tmargin=3cm]{geometry}
\usepackage{tcolorbox}
\usepackage{xspace}
\usepackage[french]{babel}
\usepackage{url}
\usepackage{hyperref}
\usepackage{fontawesome5}
\usepackage{marginnote}
\usepackage{ulem}
\usepackage{tcolorbox}
\usepackage{graphicx}
%\usepackage[top=Bcm, bottom=Hcm, outer=Ccm, inner=Acm, heightrounded, marginparwidth=Ecm, marginparsep=Dcm]{geometry}


\newtcolorbox{Example}[1]{colback=white,left=20pt,colframe=slideblue,fonttitle=\bfseries,title=#1}
\newtcolorbox{Solutions}[1]{colback=white,left=20pt,colframe=green,fonttitle=\bfseries,title=#1}
\newtcolorbox{Conseils}[1]{colback=white,left=20pt,colframe=slideblue,fonttitle=\bfseries,title=#1}
\newtcolorbox{Warning}[1]{colback=white,left=20pt,colframe=warning,fonttitle=\bfseries,title=#1}

\setlength\parindent{0pt}

  %Exercice environment
  \newcounter{exercice}
  \newenvironment{Exercice}[1][]
  {
  \par
  \stepcounter{exercice}\textbf{Question \arabic{exercice}:} (\faClock \enskip \textit{#1})
  }
  {\bigskip}
  

% Title
\newcommand{\titre}{\begin{center}
  \section*{Algorithmes et Pensée Computationnelle}
\end{center}}
\newcommand{\cours}[1]
{\begin{center} 
  \textit{#1}\\
\end{center}
  }


\newcommand{\exemple}[1]{\newline~\textbf{Exemple :} #1}
%\newcommand{\attention}[1]{\newline\faExclamationTriangle~\textbf{Attention :} #1}

% Documentation url (escape \# in the TP document)
\newcommand{\documentation}[1]{\faBookOpen~Documentation : \href{#1}{#1}}

% Clef API
\newcommand{\apikey}[1]{\faKey~Clé API : \lstinline{#1}}
\newcommand{\apiendpoint}[1]{\faGlobe~Url de base de l'API \href{#1}{#1}}

%Listing Python style
\usepackage{color}
\definecolor{slideblue}{RGB}{33,131,189}
\definecolor{green}{RGB}{0,190,100}
\definecolor{blue}{RGB}{121,142,213}
\definecolor{grey}{RGB}{120,120,120}
\definecolor{warning}{RGB}{235,186,1}

\usepackage{listings}
\lstdefinelanguage{texte}{
    keywordstyle=\color{black},
    numbers=none,
    frame=none,
    literate=
           {é}{{\'e}}1
           {è}{{\`e}}1
           {ê}{{\^e}}1
           {à}{{\`a}}1
           {â}{{\^a}}1
           {ù}{{\`u}}1
           {ü}{{\"u}}1
           {î}{{\^i}}1
           {ï}{{\"i}}1
           {ë}{{\"e}}1
           {Ç}{{\,C}}1
           {ç}{{\,c}}1,
    columns=fullflexible,keepspaces,
	breaklines=true,
	breakatwhitespace=true,
}
\lstset{
    language=Python,
	basicstyle=\bfseries\footnotesize,
	breaklines=true,
	breakatwhitespace=true,
	commentstyle=\color{grey},
	stringstyle=\color{slideblue},
  keywordstyle=\color{slideblue},
	morekeywords={with, as, True, False, Float, join, None, main, argparse, self, sort, __eq__, __add__, __ne__, __radd__, __del__, __ge__, __gt__, split, os, endswith, is_file, scandir, @classmethod},
	deletekeywords={id},
	showspaces=false,
	showstringspaces=false,
	columns=fullflexible,keepspaces,
	literate=
           {é}{{\'e}}1
           {è}{{\`e}}1
           {ê}{{\^e}}1
           {à}{{\`a}}1
           {â}{{\^a}}1
           {ù}{{\`u}}1
           {ü}{{\"u}}1
           {î}{{\^i}}1
           {ï}{{\"i}}1
           {ë}{{\"e}}1
           {Ç}{{\,C}}1
           {ç}{{\,c}}1,
    numbers=left,
}

\newtcbox{\mybox}{nobeforeafter,colframe=white,colback=slideblue,boxrule=0.5pt,arc=1.5pt, boxsep=0pt,left=2pt,right=2pt,top=2pt,bottom=2pt,tcbox raise base}
\newcommand{\projet}{\mybox{\textcolor{white}{\small projet}}\xspace}
\newcommand{\optionnel}{\mybox{\textcolor{white}{\small Optionnel}}\xspace}
\newcommand{\advanced}{\mybox{\textcolor{white}{\small Pour aller plus loin}}\xspace}
\newcommand{\auto}{\mybox{\textcolor{white}{\small Auto-évaluation}}\xspace}


\usepackage{environ}
\newif\ifShowSolution
\NewEnviron{solution}{
  \ifShowSolution
	\begin{Solutions}{\faTerminal \enskip Solution}
		\BODY
	\end{Solutions}
  \fi}


  \usepackage{environ}
  \newif\ifShowConseil
  \NewEnviron{conseil}{
    \ifShowConseil
    \begin{Conseils}{\faLightbulb \quad Conseil}
      \BODY
    \end{Conseils}

    \fi}

    \usepackage{environ}
  \newif\ifShowWarning
  \NewEnviron{attention}{
    \ifShowWarning
    \begin{Warning}{\faExclamationTriangle \quad Attention}
      \BODY
    \end{Warning}

    \fi}
  

%\newcommand{\Conseil}[1]{\ifShowIndice\ \newline\faLightbulb[regular]~#1\fi}


\usepackage{array}
\newcolumntype{C}[1]{>{\centering\let\newline\\\arraybackslash\hspace{0pt}}m{#1}}

\begin{document}
% Change the following values to true to show the solutions or/and the hints
\ShowSolutiontrue
\ShowConseiltrue
\ShowNotetrue

\titre
\cours{Introduction à Python}

\todo{Rappeler le but de cet exercice - S'inspirer de la diapositive "Learning objectives"}
\todo{Utiliser \lstinline{} pour la mise en forme de code}
\todo{Ajouter des captures d'écran afin de rendre le manuel plus intuitif}
\todo{Les box Conseils devraient être utilisées pour mettre des astuces devant aider l'étudiant et non la question à laquelle ils doivent répondre.}
\todo{Modifier l'ordre des sections. Commencer par les variables, ensuite l'indentation et enfin les fonctions.}




\section{Introduction}

\subsection{Création d'un script Python}

Pour commencer à programmer en Python, il est nécessaire de créer un fichier appelé script Python. Pour se faire, il suffit de créer un simple fichier vide et de changer l'extension du fichier. Il existe plusieurs manières d'effectuer cela. La manière la plus simple est de passer par le terminal de l'OS.

\begin{conseil}

Dans le terminal, entrer la commande \textbf{touch hello.txt} pour créer un fichier texte vide nommé \lstinline{hello.txt} changer l'extension du fichier directement depuis l'explorateur de fichier de .txt à .py.


\subsection{PyCharm}
\end{conseil}

Lors du cours, il vous a été demandé d'installer PyCharm, l'IDE Python le plus populaire de ces dernières années. Un IDE (Integrated Developement Environnement) est un programme qui combine différents outils de développement et qui facilite grandement le travail d'un programmeur.

\begin{conseil}
\todo{Ce texte devrait se trouver à l'extérieur de la boîte de conseils. Il s'agit d'une question. Prière de se référer aux exercices de l'année précédente.}
Lancez le programme PyCharm et ouvrez le fichier hello.py. Notez qu'il est aussi possible créer directement un fichier depuis PyCharm.
 
\end {conseil}

La fenêtre qui vient de s'ouvrir est similaire à un éditeur de texte basique. C'est ici que le code peut être entré et modifié.

\subsection{Les fonctions}

Dans les langages de programmation, on retrouve un très grand nombre de fonctions. Ces fonctions sont des blocs de code qui, lorsqu'ils sont invoqués avec certains paramètres, effectuent certaines actions.
Une des fonctions basique et plutôt importante en Python est la fonction \textbf{print()}.  Cette fonction permet d'afficher à l'écran le contenu de la parenthèse.

\begin{conseil}

Entrez la ligne de code \lstinline{print(``Hello World'')} puis exécutez le script en allant sur le barre de menu (tout en haut) et en cliquant sur \lstinline{Run > Run `hello'}. Dans le bas de l'écran, dans la rubrique \lstinline{Run}, vous verrez apparaître le message que vous vouliez afficher.
\end{conseil}

En plus des fonctions incluses dans les librairies Python, il est possible de créer des fonctions complètement personnalisées (généralement plusieurs fonctions sont réunies en une seule) au moyen de la fonction \lstinline{def nomdelafonction:} et de l'utiliser à n'importe quel moment en l'invoquant avec \textbf{nomdelafonction}.

\begin{solution}
\lstinputlisting{def.py}
\end{solution}

\subsection{Indentation}

Comme tous les autres langages de programmation, Python est sensible aux erreurs d'indentation \todo{Attention, tous les langages de programmation ne sont pas sensibles à une mauvaise indentation. Cette phrase peut porter à confusion}. Il sera donc important de bien comprendre comment celles-ci fonctionnent. Prenons par exemple une simple boucle \lstinline{for} qui sera traitée un peu plus loin dans ce cours.



\begin{conseil}
Entrez les deux scripts suivants dans PyCharm. Un des deux retourne une erreur d'indentation alors que l'autre fonctionne correctement.
\end{conseil}

\begin{solution}
\lstinputlisting{indent1.py}
\end{solution}

\begin{solution}
\lstinputlisting{indent2.py}
\end{solution}

L'indentation est importante dans la syntaxe d'un script Python. Une erreur d'indentation peut changer le fonctionnement d'un script ou tout simplement empêcher celui-ci de s'executer.

\subsection{Création de variable}
Les variables dans les langages de programmation sont similaires à des noms données à une valeur précise
Pour assigner une valeur à une variable en Python, il suffit de respecter la forme suivante \textbf{variable=valeur}.

\begin{conseil}
\begin{itemize}
	\item En Python, une variable est  toujours dynamique (il n'existe pas de variable statique).
	\item On peut assigner n'importe quelle suite de caractères non-reservée en tant que variable
	\item Il est aussi possible d'assigner des chaînes de caractères (strings en anglais) ou encore des valeurs booléennes à des variables
	\item En réassignant une nouvelle valeur à une variable déjà définie, la valeur de la variable va être écrasée et remplacée par la nouvelle valeur
	\item Il est possible d'additionner les variables du même type
	\item Plusieurs variables peuvent avoir la même valeur
\end{itemize}
\end{conseil}

Voici quelques exemples d'attribution de variables. Entrez ces lignes de code dans PyCharm et observez ce que le programme vous renvoie.

\begin{solution}
	\todo{Ceci ne devrait pas être mis dans la case de solutions. Pour rappel et de façon générale, dans la première version qui sera envoyée, nous masquerons les cases de solutions}
    \lstinputlisting{Variables.py}
    
\end{solution}


\end{document}
