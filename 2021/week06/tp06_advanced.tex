\documentclass[a4paper]{article}
\usepackage{times}
\usepackage[utf8]{inputenc}
\usepackage{selinput}
\usepackage{upquote}
\usepackage[margin=2cm, rmargin=4cm, tmargin=3cm]{geometry}
\usepackage{tcolorbox}
\usepackage{xspace}
\usepackage[french]{babel}
\usepackage{url}
\usepackage{hyperref}
\usepackage{fontawesome5}
\usepackage{marginnote}
\usepackage{ulem}
\usepackage{tcolorbox}
\usepackage{graphicx}
%\usepackage[top=Bcm, bottom=Hcm, outer=Ccm, inner=Acm, heightrounded, marginparwidth=Ecm, marginparsep=Dcm]{geometry}


\newtcolorbox{Example}[1]{colback=white,left=20pt,colframe=slideblue,fonttitle=\bfseries,title=#1}
\newtcolorbox{Solutions}[1]{colback=white,left=20pt,colframe=green,fonttitle=\bfseries,title=#1}
\newtcolorbox{Conseils}[1]{colback=white,left=20pt,colframe=slideblue,fonttitle=\bfseries,title=#1}
\newtcolorbox{Warning}[1]{colback=white,left=20pt,colframe=warning,fonttitle=\bfseries,title=#1}

\setlength\parindent{0pt}

  %Exercice environment
  \newcounter{exercice}
  \newenvironment{Exercice}[1][]
  {
  \par
  \stepcounter{exercice}\textbf{Question \arabic{exercice}:} (\faClock \enskip \textit{#1})
  }
  {\bigskip}
  

% Title
\newcommand{\titre}{\begin{center}
  \section*{Algorithmes et Pensée Computationnelle}
\end{center}}
\newcommand{\cours}[1]
{\begin{center} 
  \textit{#1}\\
\end{center}
  }


\newcommand{\exemple}[1]{\newline~\textbf{Exemple :} #1}
%\newcommand{\attention}[1]{\newline\faExclamationTriangle~\textbf{Attention :} #1}

% Documentation url (escape \# in the TP document)
\newcommand{\documentation}[1]{\faBookOpen~Documentation : \href{#1}{#1}}

% Clef API
\newcommand{\apikey}[1]{\faKey~Clé API : \lstinline{#1}}
\newcommand{\apiendpoint}[1]{\faGlobe~Url de base de l'API \href{#1}{#1}}

%Listing Python style
\usepackage{color}
\definecolor{slideblue}{RGB}{33,131,189}
\definecolor{green}{RGB}{0,190,100}
\definecolor{blue}{RGB}{121,142,213}
\definecolor{grey}{RGB}{120,120,120}
\definecolor{warning}{RGB}{235,186,1}

\usepackage{listings}
\lstdefinelanguage{texte}{
    keywordstyle=\color{black},
    numbers=none,
    frame=none,
    literate=
           {é}{{\'e}}1
           {è}{{\`e}}1
           {ê}{{\^e}}1
           {à}{{\`a}}1
           {â}{{\^a}}1
           {ù}{{\`u}}1
           {ü}{{\"u}}1
           {î}{{\^i}}1
           {ï}{{\"i}}1
           {ë}{{\"e}}1
           {Ç}{{\,C}}1
           {ç}{{\,c}}1,
    columns=fullflexible,keepspaces,
	breaklines=true,
	breakatwhitespace=true,
}
\lstset{
    language=Python,
	basicstyle=\bfseries\footnotesize,
	breaklines=true,
	breakatwhitespace=true,
	commentstyle=\color{grey},
	stringstyle=\color{slideblue},
  keywordstyle=\color{slideblue},
	morekeywords={with, as, True, False, Float, join, None, main, argparse, self, sort, __eq__, __add__, __ne__, __radd__, __del__, __ge__, __gt__, split, os, endswith, is_file, scandir, @classmethod},
	deletekeywords={id},
	showspaces=false,
	showstringspaces=false,
	columns=fullflexible,keepspaces,
	literate=
           {é}{{\'e}}1
           {è}{{\`e}}1
           {ê}{{\^e}}1
           {à}{{\`a}}1
           {â}{{\^a}}1
           {ù}{{\`u}}1
           {ü}{{\"u}}1
           {î}{{\^i}}1
           {ï}{{\"i}}1
           {ë}{{\"e}}1
           {Ç}{{\,C}}1
           {ç}{{\,c}}1,
    numbers=left,
}

\newtcbox{\mybox}{nobeforeafter,colframe=white,colback=slideblue,boxrule=0.5pt,arc=1.5pt, boxsep=0pt,left=2pt,right=2pt,top=2pt,bottom=2pt,tcbox raise base}
\newcommand{\projet}{\mybox{\textcolor{white}{\small projet}}\xspace}
\newcommand{\optionnel}{\mybox{\textcolor{white}{\small Optionnel}}\xspace}
\newcommand{\advanced}{\mybox{\textcolor{white}{\small Pour aller plus loin}}\xspace}
\newcommand{\auto}{\mybox{\textcolor{white}{\small Auto-évaluation}}\xspace}


\usepackage{environ}
\newif\ifShowSolution
\NewEnviron{solution}{
  \ifShowSolution
	\begin{Solutions}{\faTerminal \enskip Solution}
		\BODY
	\end{Solutions}
  \fi}


  \usepackage{environ}
  \newif\ifShowConseil
  \NewEnviron{conseil}{
    \ifShowConseil
    \begin{Conseils}{\faLightbulb \quad Conseil}
      \BODY
    \end{Conseils}

    \fi}

    \usepackage{environ}
  \newif\ifShowWarning
  \NewEnviron{attention}{
    \ifShowWarning
    \begin{Warning}{\faExclamationTriangle \quad Attention}
      \BODY
    \end{Warning}

    \fi}
  

%\newcommand{\Conseil}[1]{\ifShowIndice\ \newline\faLightbulb[regular]~#1\fi}


\usepackage{array}
\newcolumntype{C}[1]{>{\centering\let\newline\\\arraybackslash\hspace{0pt}}m{#1}}

\begin{document}
% Change the following values to true to show the solutions or/and the hints
\ShowSolutionfalse
\ShowConseiltrue
\titre
\cours{Algorithmes de recherche - Exercices avancés}


\begin{comment}
\begin{enumerate}
    \item Recherche séquentielle
    \item Recherche binaire
    \item Arbres de recherche binaire\\
\end{enumerate}
\end{comment}

Le but de cette séance est de se familiariser avec les algorithmes de recherche. Dans la série d'exercices, nous manipulerons des listes et collections en Java et Python. Nous reviendrons sur la notion de récursivité et découvrirons les arbres de recherche. Au terme de cette séance, l'étudiant sera en mesure d'effectuer des recherches de façon efficiente sur un ensemble de données.

Le code présenté dans les énoncés se trouve sur Moodle, dans le dossier \lstinline{Ressources}.


\section{Recherche binaire}

% ADVANCED
\begin{Exercice}[10 minutes] Recherche binaire (Python)\\

    Soient une liste d’entiers triés \lstinline{L} et un entier \lstinline{e}. Écrivez un programme qui retourne l'index de l'élément \lstinline{e} de la liste L en utilisant une recherche binaire. Si \lstinline{e} n’est pas dans \lstinline{L}, retournez \lstinline{-1}.\\
    
    \begin{Example}{\faTerminal \quad Exemple}
        L = [123, 321, 328, 472, 549] \\
        e = 328\\
    
    Résultat attendu: 2
    \end{Example}
    
    \lstinputlisting{resources/question1_advanced.py}
    
    \begin{conseil}
        Inspirez-vous des conseils des exercices précédents (exercices basiques).
    \end{conseil}
    
        \begin{solution}
            \textbf{Python :}
            \lstinputlisting{solutions/question1_advanced.py}
        \end{solution}
    
    \end{Exercice}
    

% ADVANCED
\begin{Exercice}[15 minutes] Recherche binaire - plus proche élément (Python)\\

Soient une liste d’entiers \textbf{triés} \lstinline{L} et un entier \lstinline{e}. Écrivez un programme retournant la valeur dans \lstinline{L} la plus proche de \lstinline{e} en utilisant une recherche binaire (binary search).\\

\lstinputlisting{resources/question2_advanced.py}

\begin{conseil}
    Pensez à définir des variables \lstinline{min} et \lstinline{max} délimitant l'intervalle de recherche et une variable booléenne \lstinline{found} initialisée à \lstinline{false} et qui devient \lstinline{true} lorsque l'algorithme trouve la valeur la plus proche de \lstinline{e}. 
    
\end{conseil}

    \begin{solution}
        \textbf{Python :}
        \lstinputlisting{solutions/question2_advanced.py}
    \end{solution}

\end{Exercice}



% ADVANCED
\begin{Exercice}[20 minutes] Recherche dans une matrice (Python) %\optionnel\\
    
    %\textbf{Matrice en Python}

    Considérez une matrice ordonnée \lstinline{m} et un élément \lstinline{l}.\\
    
    Pour rappel, une matrice ordonnée répond aux critères suivants :\\
     \lstinline{[i][j]<=m[i+1][j]} (une ligne va du plus petit au plus grand)\\
     \lstinline{[i][j]<=m[i][j+1]} (une colonne va du plus petit au plus grand)\\
 
    
    Écrivez un algorithme qui retourne la position de l’élément \lstinline{l} dans \lstinline{m}. Si \lstinline{l} n’est pas présent dans \lstinline{m} alors, votre programme retournera \lstinline{(-1, -1)}.\\
    
    \begin{Example}{\faTerminal \quad Exemples}
        \textbf{Exemple 1}: si \lstinline{m=[[1,2,3,4],[4,5,7,8],[5,6,8,10],[6,7,9,11]]} et que \lstinline{l=7}. Nous souhaitons avoir la réponse \lstinline{(1,2)} OU \lstinline{(3,1)} (l’une des deux, pas besoin de retourner les deux résultats).\\

        \textbf{Exemple 2}: si m=[[1,2],[3,4]] et que l=7. Nous souhaitons avoir la réponse (-1,-1) car 7 n’est pas dans la matrice m.
    \end{Example}

    \lstinputlisting{resources/question3_advanced.py}

    \begin{conseil}
    Pour cet exercice, il est nécessaire d'utiliser des boucles \lstinline{for} imbriquées, c'est-à-dire: une boucle \lstinline{for} dans une autre boucle \lstinline{for}. Cela permet de parcourir tous les éléments d'une liste (ou d'un tableau) à deux dimensions (dans notre cas une matrice). 
    \end{conseil}

    \begin{solution}
        \textbf{Python :}
        \lstinputlisting{solutions/question3_advanced.py}
    \end{solution}

\end{Exercice}

\end{document}
