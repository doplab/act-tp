\documentclass[a4paper]{article}
\usepackage{times}
\usepackage[utf8]{inputenc}
\usepackage{selinput}
\usepackage{upquote}
\usepackage[margin=2cm, rmargin=4cm, tmargin=3cm]{geometry}
\usepackage{tcolorbox}
\usepackage{xspace}
\usepackage[french]{babel}
\usepackage{url}
\usepackage{hyperref}
\usepackage{fontawesome5}
\usepackage{marginnote}
\usepackage{ulem}
\usepackage{tcolorbox}
\usepackage{graphicx}
\usepackage{verbatimbox}
\usepackage{amsmath}
\usepackage{hyperref}
%\usepackage[top=Bcm, bottom=Hcm, outer=Ccm, inner=Acm, heightrounded, marginparwidth=Ecm, marginparsep=Dcm]{geometry}


\newtcolorbox{Example}[1]{colback=white,left=20pt,colframe=slideblue,fonttitle=\bfseries,title=#1}
\newtcolorbox{Solutions}[1]{colback=white,left=20pt,colframe=green,fonttitle=\bfseries,title=#1}
\newtcolorbox{Conseils}[1]{colback=white,left=20pt,colframe=slideblue,fonttitle=\bfseries,title=#1}
\newtcolorbox{Warning}[1]{colback=white,left=20pt,colframe=warning,fonttitle=\bfseries,title=#1}

\setlength\parindent{0pt}

  %Exercice environment
  \newcounter{exercice}
  \newenvironment{Exercice}[1][]
  {
  \par
  \stepcounter{exercice}\textbf{Question \arabic{exercice}:} (\faClock \enskip \textit{#1})
  }
  {\bigskip}
  

% Title
\newcommand{\titre}{\begin{center}
  \section*{Algorithmes et Pensée Computationnelle}
\end{center}}
\newcommand{\cours}[1]
{\begin{center} 
  \textit{#1}\\
\end{center}
  }


\newcommand{\exemple}[1]{\newline~\textbf{Exemple :} #1}
%\newcommand{\attention}[1]{\newline\faExclamationTriangle~\textbf{Attention :} #1}

% Documentation url (escape \# in the TP document)
\newcommand{\documentation}[1]{\faBookOpen~Documentation : \href{#1}{#1}}

% Clef API
\newcommand{\apikey}[1]{\faKey~Clé API : \lstinline{#1}}
\newcommand{\apiendpoint}[1]{\faGlobe~Url de base de l'API \href{#1}{#1}}

%Listing Python style
\usepackage{color}
\definecolor{slideblue}{RGB}{33,131,189}
\definecolor{green}{RGB}{0,190,100}
\definecolor{blue}{RGB}{121,142,213}
\definecolor{grey}{RGB}{120,120,120}
\definecolor{warning}{RGB}{235,186,1}

\usepackage{listings}
\lstdefinelanguage{texte}{
    keywordstyle=\color{black},
    numbers=none,
    frame=none,
    literate=
           {é}{{\'e}}1
           {è}{{\`e}}1
           {ê}{{\^e}}1
           {à}{{\`a}}1
           {â}{{\^a}}1
           {ù}{{\`u}}1
           {ü}{{\"u}}1
           {î}{{\^i}}1
           {ï}{{\"i}}1
           {ë}{{\"e}}1
           {Ç}{{\,C}}1
           {ç}{{\,c}}1,
    columns=fullflexible,keepspaces,
	breaklines=true,
	breakatwhitespace=true,
}
\lstset{
    language=Python,
	basicstyle=\bfseries\footnotesize,
	breaklines=true,
	breakatwhitespace=true,
	commentstyle=\color{grey},
	stringstyle=\color{slideblue},
  keywordstyle=\color{slideblue},
	morekeywords={with, as, True, False, Float, join, None, main, argparse, self, sort, __eq__, __add__, __ne__, __radd__, __del__, __ge__, __gt__, split, os, endswith, is_file, scandir, @classmethod},
	deletekeywords={id},
	showspaces=false,
	showstringspaces=false,
	columns=fullflexible,keepspaces,
	literate=
           {é}{{\'e}}1
           {è}{{\`e}}1
           {ê}{{\^e}}1
           {à}{{\`a}}1
           {â}{{\^a}}1
           {ù}{{\`u}}1
           {ü}{{\"u}}1
           {î}{{\^i}}1
           {ï}{{\"i}}1
           {ë}{{\"e}}1
           {Ç}{{\,C}}1
           {ç}{{\,c}}1,
    numbers=left,
}

\newtcbox{\mybox}{nobeforeafter,colframe=white,colback=slideblue,boxrule=0.5pt,arc=1.5pt, boxsep=0pt,left=2pt,right=2pt,top=2pt,bottom=2pt,tcbox raise base}
\newcommand{\projet}{\mybox{\textcolor{white}{\small projet}}\xspace}
\newcommand{\optionnel}{\mybox{\textcolor{white}{\small Optionnel}}\xspace}
\newcommand{\auto}{\mybox{\textcolor{white}{\small Auto-évaluation}}\xspace}


\usepackage{environ}
\newif\ifShowSolution
\NewEnviron{solution}{
  \ifShowSolution
	\begin{Solutions}{\faTerminal \enskip Solution}
		\BODY
	\end{Solutions}
  \fi}


  \usepackage{environ}
  \newif\ifShowConseil
  \NewEnviron{conseil}{
    \ifShowConseil
    \begin{Conseils}{\faLightbulb \quad Conseil}
      \BODY
    \end{Conseils}

    \fi}

    \usepackage{environ}
  \newif\ifShowWarning
  \NewEnviron{attention}{
    \ifShowWarning
    \begin{Warning}{\faExclamationTriangle \quad Attention}
      \BODY
    \end{Warning}

    \fi}
  

%\newcommand{\Conseil}[1]{\ifShowIndice\ \newline\faLightbulb[regular]~#1\fi}


\usepackage{array}
\usepackage{tabto}
\newcolumntype{C}[1]{>{\centering\let\newline\\\arraybackslash\hspace{0pt}}m{#1}}

\begin{document}

% Change the following values to true to show the solutions or/and the hints
\ShowSolutiontrue
\ShowConseiltrue
\titre
\cours{Spatial Algorithms}

Le but de cette séance est de comprendre et d'implémenter des algorithmes spatiaux. \\

\section{Nearest-Neighbor}
Nous allons chercher à implémenter une recherche du plus proche voisin. Le but de cette méthode est de, étant donné un point de "départ" et un ensemble de point, trouver le voisin le plus proche du point de départ. Nous allons implémenter cette algorithme et ensuite l'étendre à un algorithme des k plus proches voisins (recherche des k voisins les plus proches plutôt que du seul voisin le plus proche).\\

\begin{Exercice}[5 minutes]\textbf{La fonction de distance : Python}\\

Pour implémenter notre recherche, nous avons besoin de coder une fonction permettant de calculer la distance entre 2 points. Programmez une fonction qui permet de calculer la distance entre 2 points.\\

\begin{conseil}
    Soit 2 points en 2 dimensions $({x_1},{y_1})$ et $({x_2},{y_2})$, la distance euclidienne entre ces 2 points est donnée par : $\sqrt{(x_2-x_1)^2+(y_2-y_1)^2}$
\end{conseil}

\begin{solution}
   \lstinputlisting[language = python]{Question1_solution.py}
   Note : Vous auriez pu utiliser (...)**0.5 en lieu et place de la fonction math.sqrt(.)
\end{solution}
\end{Exercice}

\begin{Exercice}[10 minutes]\textbf{Nearest-neighbor search}\\

Implémentez la recherche du voisin le plus proche. Ce dernier fonctionne de la façon suivante :
\begin{enumerate}
    \item Traversez chaque point.
    \item Pour chaque point, calculez la distance entre le point et le point de départ.
    \item Retournez les coordonnées du point le plus proche.
\end{enumerate}
    
\begin{conseil}
    Utilisez la fonction de distance de la question 1 et parcourez les points à l'aide d'une boucle for. Si votre input est [[2,3],[5,6],[1,4],[2,4],[3,5]] et que le point de départ est [4,4], alors l'output devrait-être ([3,5] 1.414).\\
    
    Note : Pour que votre programme fonctionne, programmez votre algorithme du plus proche voisin dans le même programme que celui de la Question 1.
\end{conseil}
\begin{solution}
    \lstinputlisting[language = python]{Question2_solution.py}
\end{solution}
\end{Exercice}


\begin{Exercice}[15 minutes]\textbf{K-nearest-neighbor search}\\


Étendez l'algorithme du voisin le plus proche à un algorithme des K plus proches voisins.\\

\begin{conseil}
Appliquez l'algorithme du plus proche voisin K-fois. A la fin de chaque itération retirez le voisin le plus proche de l'ensemble des points sur lequel l'algorithme s'applique. De cette façon vous trouverez le second voisin le plus proche, le troisième, etc...\\

Votre fonction devrait retourner une liste de la forme : $[[x_1,y_2, distance1],[x_2,y_2,distance2],...]$. Avec comme input [[2,3],[5,6],[1,4],[2,4],[3,5]], comme point de départ [4,4] et le nombre de voisins K=2, l'output de votre algorithme devrait-être : [[3, 5, 1.4142135623730951], [2, 4, 2.0]].\\

Note : Pour que votre programme fonctionne, programmez votre algorithme du plus proche voisin dans le même programme que celui de la Question 1 et la Question 2. Vous pouvez réutilisez le code des questions 1) et 2) pour cet exercice.
\end{conseil}

\begin{solution}
    \lstinputlisting[language = python]{Question3_solution.py}
\end{solution}
\end{Exercice}
\newpage
\section{K-dimensional tree}

\begin{Exercice}[5 minutes]\textbf{KD-Tree, un échauffement : Papier}\\
Vous trouverez ci-dessous une liste de points numérotés de 1 à 10. Placez-les dans un KD-Tree et dessinez la séparation de l'espace qui en résulte.\\

\includegraphics[]{KD_points.PNG}

\begin{conseil}
La première division se fait de façon verticale. Veillez à bien insérer les points dans l'ordre (point 1, point 2, etc..). Les noeuds se situant au même niveau devrait diviser l'espace selon le même axe.
\end{conseil}
\begin{solution}
    Voici le KD-Tree correspondant :\\
    \includegraphics[]{Kd-tree.PNG}\\
    Et la division de l'espace qui en résulte :\\
    \includegraphics[]{Division espace.PNG}
\end{solution}
\end{Exercice}



\end{document}
