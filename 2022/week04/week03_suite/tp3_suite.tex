\documentclass[a4paper]{article}
\usepackage{times}
\usepackage[utf8]{inputenc}
\usepackage{selinput}
\usepackage{upquote}
\usepackage[margin=2cm, rmargin=4cm, tmargin=3cm]{geometry}
\usepackage{tcolorbox}
\usepackage{xspace}
\usepackage[french]{babel}
\usepackage{url}
\usepackage{hyperref}
\usepackage{fontawesome5}
\usepackage{marginnote}
\usepackage{ulem}
\usepackage{tcolorbox}
\usepackage{graphicx}
%\usepackage[top=Bcm, bottom=Hcm, outer=Ccm, inner=Acm, heightrounded, marginparwidth=Ecm, marginparsep=Dcm]{geometry}


\newtcolorbox{Example}[1]{colback=white,left=20pt,colframe=slideblue,fonttitle=\bfseries,title=#1}
\newtcolorbox{Solutions}[1]{colback=white,left=20pt,colframe=green,fonttitle=\bfseries,title=#1}
\newtcolorbox{Conseils}[1]{colback=white,left=20pt,colframe=slideblue,fonttitle=\bfseries,title=#1}
\newtcolorbox{Warning}[1]{colback=white,left=20pt,colframe=warning,fonttitle=\bfseries,title=#1}

\setlength\parindent{0pt}

  %Exercice environment
  \newcounter{exercice}
  \newenvironment{Exercice}[1][]
  {
  \par
  \stepcounter{exercice}\textbf{Question \arabic{exercice}:} (\faClock \enskip \textit{#1})
  }
  {\bigskip}
  

% Title
\newcommand{\titre}{\begin{center}
  \section*{Algorithmes et Pensée Computationnelle}
\end{center}}
\newcommand{\cours}[1]
{\begin{center} 
  \textit{#1}\\
\end{center}
  }


\newcommand{\exemple}[1]{\newline~\textbf{Exemple :} #1}
%\newcommand{\attention}[1]{\newline\faExclamationTriangle~\textbf{Attention :} #1}

% Documentation url (escape \# in the TP document)
\newcommand{\documentation}[1]{\faBookOpen~Documentation : \href{#1}{#1}}

% Clef API
\newcommand{\apikey}[1]{\faKey~Clé API : \lstinline{#1}}
\newcommand{\apiendpoint}[1]{\faGlobe~Url de base de l'API \href{#1}{#1}}

%Listing Python style
\usepackage{color}
\definecolor{slideblue}{RGB}{33,131,189}
\definecolor{green}{RGB}{0,190,100}
\definecolor{blue}{RGB}{121,142,213}
\definecolor{grey}{RGB}{120,120,120}
\definecolor{warning}{RGB}{235,186,1}

\usepackage{listings}
\lstdefinelanguage{texte}{
    keywordstyle=\color{black},
    numbers=none,
    frame=none,
    literate=
           {é}{{\'e}}1
           {è}{{\`e}}1
           {ê}{{\^e}}1
           {à}{{\`a}}1
           {â}{{\^a}}1
           {ù}{{\`u}}1
           {ü}{{\"u}}1
           {î}{{\^i}}1
           {ï}{{\"i}}1
           {ë}{{\"e}}1
           {Ç}{{\,C}}1
           {ç}{{\,c}}1,
    columns=fullflexible,keepspaces,
	breaklines=true,
	breakatwhitespace=true,
}
\lstset{
    language=Python,
	basicstyle=\bfseries\footnotesize,
	breaklines=true,
	breakatwhitespace=true,
	commentstyle=\color{grey},
	stringstyle=\color{slideblue},
  keywordstyle=\color{slideblue},
	morekeywords={with, as, True, False, Float, join, None, main, argparse, self, sort, __eq__, __add__, __ne__, __radd__, __del__, __ge__, __gt__, split, os, endswith, is_file, scandir, @classmethod},
	deletekeywords={id},
	showspaces=false,
	showstringspaces=false,
	columns=fullflexible,keepspaces,
	literate=
           {é}{{\'e}}1
           {è}{{\`e}}1
           {ê}{{\^e}}1
           {à}{{\`a}}1
           {â}{{\^a}}1
           {ù}{{\`u}}1
           {ü}{{\"u}}1
           {î}{{\^i}}1
           {ï}{{\"i}}1
           {ë}{{\"e}}1
           {Ç}{{\,C}}1
           {ç}{{\,c}}1,
    numbers=left,
}

\newtcbox{\mybox}{nobeforeafter,colframe=white,colback=slideblue,boxrule=0.5pt,arc=1.5pt, boxsep=0pt,left=2pt,right=2pt,top=2pt,bottom=2pt,tcbox raise base}
\newcommand{\projet}{\mybox{\textcolor{white}{\small projet}}\xspace}
\newcommand{\optionnel}{\mybox{\textcolor{white}{\small Optionnel}}\xspace}
\newcommand{\advanced}{\mybox{\textcolor{white}{\small Pour aller plus loin}}\xspace}
\newcommand{\auto}{\mybox{\textcolor{white}{\small Auto-évaluation}}\xspace}


\usepackage{environ}
\newif\ifShowSolution
\NewEnviron{solution}{
  \ifShowSolution
	\begin{Solutions}{\faTerminal \enskip Solution}
		\BODY
	\end{Solutions}
  \fi}


  \usepackage{environ}
  \newif\ifShowConseil
  \NewEnviron{conseil}{
    \ifShowConseil
    \begin{Conseils}{\faLightbulb \quad Conseil}
      \BODY
    \end{Conseils}

    \fi}

    \usepackage{environ}
  \newif\ifShowWarning
  \NewEnviron{attention}{
    \ifShowWarning
    \begin{Warning}{\faExclamationTriangle \quad Attention}
      \BODY
    \end{Warning}

    \fi}
  

%\newcommand{\Conseil}[1]{\ifShowIndice\ \newline\faLightbulb[regular]~#1\fi}



\usepackage{listings}
\usepackage{array}
\newcolumntype{C}[1]{>{\centering\let\newline\\\arraybackslash\hspace{0pt}}m{#1}}

\begin{document}

% Change the following values to true to show the solutions or/and the hints
\ShowSolutiontrue
\ShowConseiltrue
\titre
\cours{Programmation de base - suite}

Le but de cette séance est d'aborder des notions de base en programmation et de consolider les connaissances acquises lors de la séance de TP précédente.
Au terme de cette séance, l'étudiant sera capable de:
\begin{itemize}
    \item interagir avec un utilisateur à travers des inputs et outputs,
    \item définir et manipuler des variables en Java et Python,
    \item manipuler des chaînes de caractères,
    \item utiliser des branchements conditionnels.
\end{itemize}

\section{Input / Output}

\begin{Exercice}[5 minutes] \textbf{Output (Java ou Python)}\\
   Créez une variable \textit{nom} (str) contenant votre nom, et une autre \textit{prenom} (str) contenant votre prénom puis affichez : "Bonjour, \textit{prenom nom}". \\
   
    \begin{conseil}
        Utilisez la fonction \lstinline{print()} de Python et \lstinline{System.out.println()} de Java. 
        
    \end{conseil}
    \begin{solution}
    
    \textbf{Python:} 
    \lstinputlisting{solutions/question1.py}
    
    \textbf{\\Java:} 
    \lstinputlisting{solutions/question1.java}  
    \end{solution}   
\end{Exercice}

\begin{Exercice}[5 minutes] \textbf{Input (Java ou Python)}\\
   En vous référant à l'exercice précédent (\textbf{Output (Java ou Python)}), demandez à l'utilisateur d'entrer son nom et son prénom via la fonction \lstinline{input()} au lieu d'initialiser vous-même les variables. \\
   
    \begin{conseil}
       Utilisez la fonction \lstinline{input()} en Python, la classe \lstinline{Scanner()} en Java (n'oubliez pas d'ajouter \lstinline{import java.util.Scanner;}) tout au début de votre code. 
        
    \end{conseil}
    \begin{solution}
    
    \textbf{Python:} 
    \lstinputlisting{solutions/question2.py}
    
    \textbf{\\Java:}
    \lstinputlisting{solutions/question2.java}  
       
        
    \end{solution}   
\end{Exercice}

\section{Utilisation de variables}

\begin{Exercice}[3 minutes] \textbf{Type (Python uniquement)}\\
    Déclarez deux variables \textit{nom} (String) et \textit{age} (int), puis affichez le type de chacune de ces deux variables.
    
     \begin{conseil}
        Vous pouvez contrôler le type de vos variables via la fonction \lstinline{type()}.
         
     \end{conseil}
     \begin{solution}
      
     \lstinputlisting{solutions/Question3.py}
            
     \end{solution}   
 \end{Exercice}
 
 \begin{Exercice}[5 minutes] \textbf{Conversion des variables (Type casting) (Java ou Python)}\\
    Il est possible de convertir une variable d'un certain type vers un autre type. Il est par exemple possible de changer un \lstinline{int} en \lstinline{float} ou un \lstinline{float} en \lstinline{int}. Cette opération se nomme le \textit{Type Casting}. \\
    Déclarez une variable \textit{nombre\_entier} de type \lstinline{int}, puis une autre variable \textit{nombre\_decimal} de type \lstinline{float}. Affichez \textit{nombre\_entier} en le convertissant en \lstinline{float} et \textit{nombre\_decimal} en le convertissant en \lstinline{int}. \\
    
     \begin{conseil}
        Utilisez la fonction \lstinline{int(float)} et \lstinline{float(int)} en Python / Utilisez \lstinline{(int) float} et (float) int en Java.
         
     \end{conseil}
     \begin{solution}
     
     \textbf{Python:}
     
     \lstinputlisting{solutions/question4.py}
     
     \textbf{\\Java:}
     \lstinputlisting{solutions/question4.java}
            
     \end{solution}   
 \end{Exercice}

 \begin{Exercice}[3 minutes] \textbf{Calculs (multiplication) (Java ou Python)}\\
 
    Créez 2 variables \textit{facteur\_1} (= 11) et \textit{facteur\_2} (= 3). Multipliez la première variable par la deuxième et stockez le résultat dans une nouvelle variable \textit{produit}. Vous pouvez afficher les différentes variables pour voir leurs valeurs. Vous pouvez répéter l'exercice avec l'addition et la soustraction. \\
    
     \begin{conseil}
           L'opérateur de multiplication est le *, celui d'addition est le + et celui de soustraction est le -.
         
     \end{conseil}
     \begin{solution}
     
     \textbf{Python:}
     \lstinputlisting{solutions/question5.py}
     
     \textbf{\\Java:}
     \lstinputlisting{solutions/question5.java}
            
     \end{solution}   
 \end{Exercice}
 
 \begin{Exercice}[10 minutes] \textbf{Calculs (division) (Java ou Python)}\\
     Créez 2 variables \textit{nb\_bonbons} avec pour valeur 11 et \textit{nb\_personnes} avec pour valeur 3. Divisez la première variable par la deuxième et stockez le résultat dans une nouvelle variable \textit{bonbons\_personnes}. Pour finir, calculez le nombre de bonbons restants via l'opérateur \% (modulo) et stockez le résultat dans une nouvelle variable \textit{reste}. Vous pouvez afficher les différentes variables pour voir leurs valeurs. \\
     
      \begin{conseil}
         \begin{itemize}
            \item Attention, en Python il existe 2 opérateurs de division, / effectue une division classique, tandis que // effectue une division entière.
            \item En Java, si vous travaillez uniquement avec des int, / effectuera une division entière tandis que si vous travaillez avec au moins un float, / effectuera une division classique.
            \item L'opérateur \% (modulo) permet de calculer le reste d'une division. Par exemple, 10\%2=0.
         \end{itemize}
      \end{conseil}
      \begin{solution}
      
      \textbf{Python:}
      
      \lstinputlisting{solutions/question6.py}
      
      \textbf{\\Java:}
      \lstinputlisting{solutions/question6.java}
             
      \end{solution}   
  \end{Exercice}
  
 \begin{Exercice}[5 minutes] \textbf{Calculs (incrémentation / décrémentation) (Java ou Python)}\\
    Gardez vos variables de l'exercice précédent (\textbf{Calculs (division) (Java ou Python)}), augmentez la valeur de \textit{nb\_bonbons} de 1, et diminuez celle de \textit{nb\_personnes} de 1.  \\
    
     \begin{conseil}
           Vous pouvez utiliser les opérateurs += et -= en Python, et les opérateurs ++ (incrémentation) et $--$ (décrémentation) en Java.
         
     \end{conseil}
     \begin{solution}
     
     \textbf{Python:}
     \lstinputlisting{solutions/question7.py}
     
     \textbf{\\Java:}
     \lstinputlisting{solutions/question7.java}
            
     \end{solution}   
 \end{Exercice}

\section{Manipulation de chaînes de caractères}

\begin{Exercice}[5 minutes] \textbf{Format d'impression (Python uniquement)}\\
   Créez et assignez des valeurs à 2 variables \textit{prenom} (str) et \textit{age} (int), puis affichez: "Je m'appelle \textit{prenom} et j'ai \textit{age} ans". Gérez le format de l'impression via l'opérateur +, puis en utilisant la fonction \lstinline{format()}. \\
   
    \begin{conseil}
       N'hésitez pas à consulter ce lien pour plus de détails concernant l'utilisation de la fonction format(): \url{https://docs.python.org/fr/3/library/stdtypes.html\#str.format}
    \end{conseil}
    \begin{solution}
     
    \lstinputlisting{solutions/question8.py}
           
    \end{solution}   
\end{Exercice}

\begin{Exercice}[5 minutes] \textbf{Manipulation des chaînes de caractères (indexation) (Java ou Python)}\\
   Créez une variable \textit{mon\_mot} de type chaîne de caractères avec pour valeur \textbf{``Hard But Cool !!''}. Créez ensuite une variable \textit{premiere} contenant la première lettre de \textit{mon\_mot} en utilisant l'indexation. Créez enfin une variable \textit{derniere} contenant la dernière lettre de \textit{mon\_mot} en utilisant l'indexation. Affichez les résultats. Qu'obtenez-vous? \\
   
    \begin{conseil}
      	Pour Python, utilisez \lstinline{[]}, et pour Java, utilisez la fonction \lstinline{substring()} ainsi que la fonction \lstinline{length()} qui permet d'obtenir la taille d'un élément.
        
    \end{conseil}
    \begin{solution}
    
    \textbf{Python:}
    \lstinputlisting{solutions/question9.py}
    
    \textbf{\\Java:}
    \lstinputlisting{solutions/question9.java}
           
    \end{solution}   
\end{Exercice}

\begin{Exercice}[5 minutes] \textbf{Manipulation des chaînes de caractères (indexation 2) (Java ou Python)}\\
   Gardez votre variable, \textit{mon\_mot} et créez une variable \textit{lettre\_5} contenant la cinquième lettre de \textit{mon\_mot} en utilisant l'indexation. Créez ensuite une variable \textit{lettre\_9\_13} contenant les lettres 9, 10, 11, 12, 13 de \textit{mon\_mot}. Afficher les résultats et voyez ce que vous obtenez.  \\
   
    \begin{conseil}
      	Attention, ici les espaces comptent comme des lettres ! \\

		Pour Python, utilisez \lstinline{[:]}, et pour Java, utilisez la fonction \lstinline{substring()}.
        
    \end{conseil}
    \begin{solution}
    
    \textbf{Python:}    
    \lstinputlisting{solutions/question10.py}
    
    \textbf{\\Java:}
    \lstinputlisting{solutions/question10.java}
           
    \end{solution}   
\end{Exercice}
\begin{Exercice}[10 minutes] \textbf{Manipulation des chaînes de caractères (Java ou Python) \optionnel}\\
   Il est possible d'obtenir la longueur d'une chaîne de caractères (ou d'une liste ou d'un dictionnaire) en utilisant la fonction \lstinline{len()}. Gardez votre variable \textit{mon\_mot} et créez une nouvelle variable nommée \textit{ln\_mon\_mot} contenant le nombre de caractères de la variable \textit{mon\_mot}, puis une nouvelle variable \textit{moitie} contenant la première moitié de la variable \textit{mon\_mot} (utilisez la variable que vous venez de créer). Affichez le résultat.   \\
   
    \begin{conseil}
      	La fonction présentée dans l'énoncé de la question n'est valable que pour Python. L'équivalent pour Java est la fonction \lstinline{length()}.
        
    \end{conseil}
    \begin{solution}
    
    \textbf{Python}:
    \lstinputlisting{solutions/question11.py}
    
    \textbf{\\Java}:
    \lstinputlisting{solutions/question11.java}
    \end{solution}   
\end{Exercice}

\newpage

\section{Conditions}
% TODO: Exercice à compléter

\begin{Exercice}[5 minutes] \textbf{Conditions (Python)}\\
    Qu'affiche le programme suivant? \\
    
    \lstinputlisting{resources/Questions/Question12.py}

     \begin{conseil}
         En Python, la fonction \lstinline{not} renvoie l'opposé d'une valeur booléenne. Par exemple, \lstinline{not(False)} renverra \lstinline{True}. En Java, on utilise un point d'exclamation avant la valeur.
     \end{conseil}
     \begin{solution}
        Garbinato : Professeur du cours Algorithmique et pensée computationelle.
     \end{solution}   
 \end{Exercice}

 \begin{Exercice}[10 minutes] \textbf{Branchement conditionnel en Java}\\
   Qu'affiche le programme suivant? 
   
   \lstinputlisting{solutions/question12.java}
    
     \begin{conseil}
          \begin{itemize}
             \item \lstinline{break} indique que l'on sort de l'accolade. Les cas suivants ne seront pas traités.
             \item L'absence de \lstinline{break} indique que l'on va rentrer dans tous les cas suivants, jusqu'à enfin atteindre un \lstinline{break}.
             \item Lorsque l'on pose \lstinline{case n} où \lstinline{n} est un nombre cela est équivalent au test \lstinline{n == numero_mois}. Ce test est aussi valable si on cherche à comparer des chaînes de caractères (par exemple si \lstinline{numero_mois = "Juin"}, à ce moment là \lstinline{n} sera aussi une chaîne de caractères).
          \end{itemize}
         
     \end{conseil}
     \begin{solution}
     
     Juillet
     
    Aout
    
    Décembre \\
    
    Explications:
    \begin{itemize}
             \item Comme le \lstinline{case 7} ne contient pas de \lstinline{break} et modifie \lstinline{numero_mois}, la lecture du code va continuer.
             \item On rentre dans le \lstinline{case 9}, qui contient un \lstinline{break}. Le \lstinline{numero_mois} sera aussi modifié mais cela ne sera pas important car on sort de l'accolade et les cas suivants ne seront pas traités.
       \end{itemize}
     \end{solution}   
 \end{Exercice}

\end{document}
