\documentclass[a4paper]{article}
\usepackage{times}
\usepackage[utf8]{inputenc}
\usepackage{selinput}
\usepackage{upquote}
\usepackage[margin=2cm, rmargin=4cm, tmargin=3cm]{geometry}
\usepackage{tcolorbox}
\usepackage{xspace}
\usepackage[french]{babel}
\usepackage{url}
\usepackage{hyperref}
\usepackage{fontawesome5}
\usepackage{marginnote}
\usepackage{ulem}
\usepackage{tcolorbox}
\usepackage{graphicx}
%\usepackage[top=Bcm, bottom=Hcm, outer=Ccm, inner=Acm, heightrounded, marginparwidth=Ecm, marginparsep=Dcm]{geometry}


\newtcolorbox{Example}[1]{colback=white,left=20pt,colframe=slideblue,fonttitle=\bfseries,title=#1}
\newtcolorbox{Solutions}[1]{colback=white,left=20pt,colframe=green,fonttitle=\bfseries,title=#1}
\newtcolorbox{Conseils}[1]{colback=white,left=20pt,colframe=slideblue,fonttitle=\bfseries,title=#1}
\newtcolorbox{Warning}[1]{colback=white,left=20pt,colframe=warning,fonttitle=\bfseries,title=#1}

\setlength\parindent{0pt}

  %Exercice environment
  \newcounter{exercice}
  \newenvironment{Exercice}[1][]
  {
  \par
  \stepcounter{exercice}\textbf{Question \arabic{exercice}:} (\faClock \enskip \textit{#1})
  }
  {\bigskip}
  

% Title
\newcommand{\titre}{\begin{center}
  \section*{Algorithmes et Pensée Computationnelle}
\end{center}}
\newcommand{\cours}[1]
{\begin{center} 
  \textit{#1}\\
\end{center}
  }


\newcommand{\exemple}[1]{\newline~\textbf{Exemple :} #1}
%\newcommand{\attention}[1]{\newline\faExclamationTriangle~\textbf{Attention :} #1}

% Documentation url (escape \# in the TP document)
\newcommand{\documentation}[1]{\faBookOpen~Documentation : \href{#1}{#1}}

% Clef API
\newcommand{\apikey}[1]{\faKey~Clé API : \lstinline{#1}}
\newcommand{\apiendpoint}[1]{\faGlobe~Url de base de l'API \href{#1}{#1}}

%Listing Python style
\usepackage{color}
\definecolor{slideblue}{RGB}{33,131,189}
\definecolor{green}{RGB}{0,190,100}
\definecolor{blue}{RGB}{121,142,213}
\definecolor{grey}{RGB}{120,120,120}
\definecolor{warning}{RGB}{235,186,1}

\usepackage{listings}
\lstdefinelanguage{texte}{
    keywordstyle=\color{black},
    numbers=none,
    frame=none,
    literate=
           {é}{{\'e}}1
           {è}{{\`e}}1
           {ê}{{\^e}}1
           {à}{{\`a}}1
           {â}{{\^a}}1
           {ù}{{\`u}}1
           {ü}{{\"u}}1
           {î}{{\^i}}1
           {ï}{{\"i}}1
           {ë}{{\"e}}1
           {Ç}{{\,C}}1
           {ç}{{\,c}}1,
    columns=fullflexible,keepspaces,
	breaklines=true,
	breakatwhitespace=true,
}
\lstset{
    language=Python,
	basicstyle=\bfseries\footnotesize,
	breaklines=true,
	breakatwhitespace=true,
	commentstyle=\color{grey},
	stringstyle=\color{slideblue},
  keywordstyle=\color{slideblue},
	morekeywords={with, as, True, False, Float, join, None, main, argparse, self, sort, __eq__, __add__, __ne__, __radd__, __del__, __ge__, __gt__, split, os, endswith, is_file, scandir, @classmethod},
	deletekeywords={id},
	showspaces=false,
	showstringspaces=false,
	columns=fullflexible,keepspaces,
	literate=
           {é}{{\'e}}1
           {è}{{\`e}}1
           {ê}{{\^e}}1
           {à}{{\`a}}1
           {â}{{\^a}}1
           {ù}{{\`u}}1
           {ü}{{\"u}}1
           {î}{{\^i}}1
           {ï}{{\"i}}1
           {ë}{{\"e}}1
           {Ç}{{\,C}}1
           {ç}{{\,c}}1,
    numbers=left,
}

\newtcbox{\mybox}{nobeforeafter,colframe=white,colback=slideblue,boxrule=0.5pt,arc=1.5pt, boxsep=0pt,left=2pt,right=2pt,top=2pt,bottom=2pt,tcbox raise base}
\newcommand{\projet}{\mybox{\textcolor{white}{\small projet}}\xspace}
\newcommand{\optionnel}{\mybox{\textcolor{white}{\small Optionnel}}\xspace}
\newcommand{\advanced}{\mybox{\textcolor{white}{\small Pour aller plus loin}}\xspace}
\newcommand{\auto}{\mybox{\textcolor{white}{\small Auto-évaluation}}\xspace}


\usepackage{environ}
\newif\ifShowSolution
\NewEnviron{solution}{
  \ifShowSolution
	\begin{Solutions}{\faTerminal \enskip Solution}
		\BODY
	\end{Solutions}
  \fi}


  \usepackage{environ}
  \newif\ifShowConseil
  \NewEnviron{conseil}{
    \ifShowConseil
    \begin{Conseils}{\faLightbulb \quad Conseil}
      \BODY
    \end{Conseils}

    \fi}

    \usepackage{environ}
  \newif\ifShowWarning
  \NewEnviron{attention}{
    \ifShowWarning
    \begin{Warning}{\faExclamationTriangle \quad Attention}
      \BODY
    \end{Warning}

    \fi}
  

%\newcommand{\Conseil}[1]{\ifShowIndice\ \newline\faLightbulb[regular]~#1\fi}



\usepackage{listings}
\usepackage{array}
\newcolumntype{C}[1]{>{\centering\let\newline\\\arraybackslash\hspace{0pt}}m{#1}}

\begin{document}

% Change the following values to true to show the solutions or/and the hints
\ShowSolutiontrue
\ShowConseiltrue
\titre
\cours{Fonctions, gestion de la mémoire et des exceptions - Exercices de base}

Le but de cette séance est d'approfondir vos connaissances en programmation. Au terme de cette séance, l'étudiant sera capable de :
\begin{itemize}
    \item Utiliser des librairies contenant des fonctions prédéfinies,
    \item définir une fonction et l'utiliser dans un programme,
    \item connaître quelle est la portée d'une variable,
    \item comprendre comment fonctionne la gestion de la mémoire,
    \item gérer des exceptions.
\end{itemize}


\section{Variables et Fonctions}
\begin{Exercice}[5 minutes] \textbf{Les fonctions (fonctions basiques) (Java ou Python))}\\
    Définissez une fonction nommée \lstinline{ping()} qui, lorsqu'elle est appelée, affiche ``pong''. Appelez la plusieurs fois et observez le résultat.  \\
     
      \begin{conseil}
          \begin{itemize}
              \item Référez vous aux diapositives du cours pour la création et l'appel des fonctions. 
              \item Vous pourriez utiliser une boucle for pour effectuer plusieurs appels à la fonction \lstinline{ping()}.
          \end{itemize}        
      \end{conseil}
      \begin{solution}
      
          \textbf{Python}:
          % \lstinputlisting{resources/solutions/Question19.py}
          
          \textbf{\\Java}:
          % \lstinputlisting{resources/solutions/Question19.java}
             
      \end{solution}   
  \end{Exercice}
  
    \begin{Exercice}[5 minutes] \textbf{Les Fonctions (Fonction multiplication) (Java ou Python)}\\
    Définissez une fonction nommée \lstinline{multiplicateur()} qui prend deux arguments \textit{multiple\_1} et \textit{multiple\_2}, les multiplie et retourne le résultat. Stockez le résultat de \lstinline{multiplicateur(2,3)} dans une variable \textit{resultat} et affichez la.   \\

    \begin{conseil}
        \begin{itemize}
            \item Référez vous au cours pour la création et l'appel des fonctions.
            \item Pour retourner une valeur au lieu de l'imprimer, utilisez le mot clé \lstinline{return} (pour Python et Java).
        \end{itemize}        
    \end{conseil}
    \begin{solution}
        \textbf{Python}:
        % \lstinputlisting{resources/Question23.py}
        
        \textbf{\\Java}:
        % \lstinputlisting{resources/Question23.java}
    \end{solution}   
\end{Exercice} 
  
  \begin{Exercice}[5 minutes] \textbf{Les Fonctions (Fonctions Aire et Périmètre) (Java ou Python)}\\
      Définissez deux fonctions nommées \lstinline{aire()} et \lstinline{perimètre()} qui prennent un argument (\lstinline{rayon}) et renvoient respectivement l'aire et le périmètre d'un cercle. Stockez les résultats dans des variables \lstinline{aire} et \lstinline{perimetre} et affichez le contenu de ces variables.   \\
     
      \begin{conseil}
          \begin{itemize}
              \item Référez vous au cours pour la création et l'appel des fonctions.
              \item Pour retourner une valeur au lieu de l'imprimer, utilisez le mot clé \lstinline{return} (pour Python et Java).
              \item Pour rappel, le périmètre d'un cercle s'obtient en utilisant la formule $P = 2*\pi*r$ et l'aire s'obtient en utilisant la formule $A = r^2*\pi$.
          \end{itemize}        
      \end{conseil}
      \begin{solution}
          \textbf{Python}:
          % \lstinputlisting{resources/solutions/Question20.py}
          
          \textbf{\\Java}:
          % \lstinputlisting{resources/solutions/Question20.java}
      \end{solution}   
  \end{Exercice} 

  \begin{Exercice}[5 minutes] \textbf{Portée des variables}\\
    Qu'affiche le programme suivant? \\

    \lstinputlisting{ressources/Questions/Question4.py}

    
     \begin{conseil}
        \begin{itemize}
            \item Le mot-clé \lstinline{global} permet d'accéder aux variables globales (définies à l'extérieur de la fonction).
            \item Soyez attentif au type des éléments retournés par chacune des fonctions.
        \end{itemize}
     \end{conseil}
     \begin{solution}
            x=2 \\
            x=3 \\
            None   
    \end{solution}
 \end{Exercice}

 \begin{comment}
 \begin{Exercice}[5 minutes] \textbf{Portée des variables (à compléter)}\\
    Exercice 1 - structure \\
    
     \begin{conseil}
        % Ajouter un conseil
         
     \end{conseil}
     \begin{solution}
     
     \textbf{Python:} 
     % Ajouter une solution en Python
     
     \textbf{\\Java:} 
      % Ajouter une solution en Java
        
         
     \end{solution}   
 \end{Exercice}
\end{comment}

\section{Gestion de la mémoire}

\begin{Exercice}[5 minutes] \textbf{Template (langage à utiliser)}\\
    Exercice 1 - structure \\
    
     \begin{conseil}
        % Ajouter un conseil
         
     \end{conseil}
     \begin{solution}
     
     \textbf{Python:} 
     % Ajouter une solution en Python
     
     \textbf{\\Java:} 
      % Ajouter une solution en Java
        
         
     \end{solution}   
 \end{Exercice}

\section{Gestion des exceptions}

\begin{Exercice}[5 minutes] Qu'affiche le programme suivant? \textbf{Types d'erreurs (Python)}\\
    \lstinputlisting{ressources/Questions/Question6.py}  
     \begin{conseil}
        \begin{itemize}
            \item Utilisée sur une chaîne de caractères, la fonction \lstinline{len()} renvoie le nombre de caractères de la chaîne. 
        \end{itemize}  
       
    \end{conseil}
     \begin{solution}
        \lstinline{Le résultat de la division est: 2.5}\\
        \lstinline{On ne peut pas diviser une chaîne de caractères.}
     \end{solution}   
 \end{Exercice}

\begin{Exercice}[5 minutes] Qu'affiche le programme suivant? \textbf{Types d'erreurs (Python)}\\
    \lstinputlisting{ressources/Questions/Question7.py}
    
     \begin{conseil}
         La chaîne de caractères vide est représentée par ``'' 
    \end{conseil}
     \begin{solution}
        Nous ne pouvons pas diviser un nombre par 0.
     \end{solution}   
 \end{Exercice}

 \begin{Exercice}[5 minutes] Qu'affiche le programme suivant? \textbf{Types d'erreurs (Python)}\\
    \lstinputlisting{ressources/Questions/Question8.py}
    
     \begin{conseil}
        \begin{itemize}
            \item La fonction \lstinline{float()} permet de convertir une variable en nombre à décimal.
            \item La fonction \lstinline{int()} permet de convertir une variable en nombre entier.
        \end{itemize}  
    \end{conseil}

     \begin{solution}
        Nous ne pouvons pas convertir une chaîne de caractères.
     \end{solution}

\end{Exercice}
 
\begin{Exercice}[5 minutes] Est ce que le programme suivant s'exécute correctement ? Si la réponse est non, expliquez pourquoi et comment vous corrigerez le programme ? \textbf{Gestion d'erreurs (Java)}\\
    \lstinputlisting{ressources/Questions/Question9.java}

    
     \begin{conseil}
        Référez vous aux cours pour la gestion d'erreur en Java (slide 26)         
     \end{conseil}
     \begin{solution}
     Le programme suivant est faux, le type d'erreur que la fonction \lstinline{division()} retourne est différent de celui qu'on intercepte dans le \lstinline{catch}. \\
     Pour corriger cela, il suffit de remplacer \lstinline{IndexOutOfBoundsException} par \lstinline{ArithmeticException}, qui correspond au type d'erreur retourné par la fonction \lstinline{division}.
     \end{solution}   
 \end{Exercice}

\end{document}
