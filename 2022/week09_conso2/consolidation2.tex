\documentclass[a4paper]{article}
\usepackage{times}
\usepackage[utf8]{inputenc}
\usepackage{selinput}
\usepackage{upquote}
\usepackage[margin=2cm, rmargin=4cm, tmargin=3cm]{geometry}
\usepackage{tcolorbox}
\usepackage{xspace}
\usepackage[french]{babel}
\usepackage{url}
\usepackage{hyperref}
\usepackage{fontawesome5}
\usepackage{marginnote}
\usepackage{ulem}
\usepackage{tcolorbox}
\usepackage{graphicx}
%\usepackage[top=Bcm, bottom=Hcm, outer=Ccm, inner=Acm, heightrounded, marginparwidth=Ecm, marginparsep=Dcm]{geometry}


\newtcolorbox{Example}[1]{colback=white,left=20pt,colframe=slideblue,fonttitle=\bfseries,title=#1}
\newtcolorbox{Solutions}[1]{colback=white,left=20pt,colframe=green,fonttitle=\bfseries,title=#1}
\newtcolorbox{Conseils}[1]{colback=white,left=20pt,colframe=slideblue,fonttitle=\bfseries,title=#1}
\newtcolorbox{Warning}[1]{colback=white,left=20pt,colframe=warning,fonttitle=\bfseries,title=#1}

\setlength\parindent{0pt}

  %Exercice environment
  \newcounter{exercice}
  \newenvironment{Exercice}[1][]
  {
  \par
  \stepcounter{exercice}\textbf{Question \arabic{exercice}:} (\faClock \enskip \textit{#1})
  }
  {\bigskip}
  

% Title
\newcommand{\titre}{\begin{center}
  \section*{Algorithmes et Pensée Computationnelle}
\end{center}}
\newcommand{\cours}[1]
{\begin{center} 
  \textit{#1}\\
\end{center}
  }


\newcommand{\exemple}[1]{\newline~\textbf{Exemple :} #1}
%\newcommand{\attention}[1]{\newline\faExclamationTriangle~\textbf{Attention :} #1}

% Documentation url (escape \# in the TP document)
\newcommand{\documentation}[1]{\faBookOpen~Documentation : \href{#1}{#1}}

% Clef API
\newcommand{\apikey}[1]{\faKey~Clé API : \lstinline{#1}}
\newcommand{\apiendpoint}[1]{\faGlobe~Url de base de l'API \href{#1}{#1}}

%Listing Python style
\usepackage{color}
\definecolor{slideblue}{RGB}{33,131,189}
\definecolor{green}{RGB}{0,190,100}
\definecolor{blue}{RGB}{121,142,213}
\definecolor{grey}{RGB}{120,120,120}
\definecolor{warning}{RGB}{235,186,1}

\usepackage{listings}
\lstdefinelanguage{texte}{
    keywordstyle=\color{black},
    numbers=none,
    frame=none,
    literate=
           {é}{{\'e}}1
           {è}{{\`e}}1
           {ê}{{\^e}}1
           {à}{{\`a}}1
           {â}{{\^a}}1
           {ù}{{\`u}}1
           {ü}{{\"u}}1
           {î}{{\^i}}1
           {ï}{{\"i}}1
           {ë}{{\"e}}1
           {Ç}{{\,C}}1
           {ç}{{\,c}}1,
    columns=fullflexible,keepspaces,
	breaklines=true,
	breakatwhitespace=true,
}
\lstset{
    language=Python,
	basicstyle=\bfseries\footnotesize,
	breaklines=true,
	breakatwhitespace=true,
	commentstyle=\color{grey},
	stringstyle=\color{slideblue},
  keywordstyle=\color{slideblue},
	morekeywords={with, as, True, False, Float, join, None, main, argparse, self, sort, __eq__, __add__, __ne__, __radd__, __del__, __ge__, __gt__, split, os, endswith, is_file, scandir, @classmethod},
	deletekeywords={id},
	showspaces=false,
	showstringspaces=false,
	columns=fullflexible,keepspaces,
	literate=
           {é}{{\'e}}1
           {è}{{\`e}}1
           {ê}{{\^e}}1
           {à}{{\`a}}1
           {â}{{\^a}}1
           {ù}{{\`u}}1
           {ü}{{\"u}}1
           {î}{{\^i}}1
           {ï}{{\"i}}1
           {ë}{{\"e}}1
           {Ç}{{\,C}}1
           {ç}{{\,c}}1,
    numbers=left,
}

\newtcbox{\mybox}{nobeforeafter,colframe=white,colback=slideblue,boxrule=0.5pt,arc=1.5pt, boxsep=0pt,left=2pt,right=2pt,top=2pt,bottom=2pt,tcbox raise base}
\newcommand{\projet}{\mybox{\textcolor{white}{\small projet}}\xspace}
\newcommand{\optionnel}{\mybox{\textcolor{white}{\small Optionnel}}\xspace}
\newcommand{\advanced}{\mybox{\textcolor{white}{\small Pour aller plus loin}}\xspace}
\newcommand{\auto}{\mybox{\textcolor{white}{\small Auto-évaluation}}\xspace}


\usepackage{environ}
\newif\ifShowSolution
\NewEnviron{solution}{
  \ifShowSolution
	\begin{Solutions}{\faTerminal \enskip Solution}
		\BODY
	\end{Solutions}
  \fi}


  \usepackage{environ}
  \newif\ifShowConseil
  \NewEnviron{conseil}{
    \ifShowConseil
    \begin{Conseils}{\faLightbulb \quad Conseil}
      \BODY
    \end{Conseils}

    \fi}

    \usepackage{environ}
  \newif\ifShowWarning
  \NewEnviron{attention}{
    \ifShowWarning
    \begin{Warning}{\faExclamationTriangle \quad Attention}
      \BODY
    \end{Warning}

    \fi}
  

%\newcommand{\Conseil}[1]{\ifShowIndice\ \newline\faLightbulb[regular]~#1\fi}


\usepackage{array}
\usepackage{blindtext}
\usepackage{multicol}
\newcolumntype{C}[1]{>{\centering\let\newline\\\arraybackslash\hspace{0pt}}m{#1}}

\begin{document}
% Change the following values to true to show the solutions or/and the hints
\ShowSolutiontrue
\ShowConseiltrue
\ShowNotefalse
\titre
\cours{Classes abstraites et interfaces}

% Le but de cette séance est d'approfondir les notions de programmation orientée objet vues précédemment. 

Les exercices sont construits autour des concepts d'héritage, de classes abstraites et d'interfaces. Au terme de cette séance, vous devez être en mesure de différencier une classe abstraite d'une interface, savoir à quel moment utiliser l'un ou l'autre, utiliser le concept d'héritage multiple, factoriser votre code afin de le rendre mieux structuré et plus lisible.

Cette série d'exercices est divisée en 3 sections dont les premières portant sur les classes abstraites et les interfaces. La dernière section comporte des exercices pratiques sur les notions abordées précédemment.

Les exercices doivent être faits uniquement en \lstinline{Java}.

Le code présenté dans les énoncés se trouve sur Moodle, dans le dossier \lstinline{Ressources}.

\section{Programmation Orientée Objet}

\begin{Exercice}[10 minutes]{Encapsulation - Java}

L'encapsulation sert à cacher les détails d'implémentation. L'encapsulation sert uniquement à montrer que les informations essentielles aux utilisateurs.\\

En Java, il est recommandé de déclarer les attributs des classes comme étant \lstinline{private} et de mettre à disposition des utilisateurs des méthodes publiques d'accès afin qu'ils puissent accéder ou modifier la valeur des attributs privés.\\

Les méthodes publiques d'accès comme \lstinline{getName}et \lstinline{setName} doivent être nommées avec soit \lstinline{get} soit \lstinline{set} suivi du nom de l'attribut avec la 1ère lettre en majuscule (Java Naming convention)[https://www.oracle.com/java/technologies/javase/codeconventions-namingconventions.html]). 

Le mot clé \lstinline{this} fait référence à l'objet en question.\\

\begin{enumerate}
	\item Dans votre IDE, créez une classe \lstinline{Person},
	\item Ajoutez-y un attribut privé \lstinline{name},
	\item Créez un getter et un setter pour l'attribut \lstinline{name} en suivant la convention de nommage des méthodes.
	\item Dans votre \lstinline{main}, créez une instance de \lstinline{Person}, 
	\item En utilisant le setter défini précédemment, donnez un nom (\lstinline{name}) à votre instance.
	\item Affichez le nom de l'instance en utilisant le getter de l'attribut \lstinline{name}.
\end{enumerate}
	\begin{solution}  
		\lstinputlisting{solutions/Person.java} 
	\end{solution}
\end{Exercice}
	
	
\begin{Exercice}[10 minutes]{Héritage - Java}

	Comme vous le savez, il est possible que des classes-filles héritent des attributs ou méthodes de classes-mères.
	En Java, il faut utiliser le mot-clé \lstinline{extends} lorsqu'on défini une classe-fille. Ainsi, l'héritage permet la réutilisation des attributs et méthodes d'une classe existante.

	\begin{enumerate}
		\item Créez une classe-mère \lstinline{Vehicle} ayant un attribut protégé appelé \lstinline{brand}.
		\item Créez un constructeur pour la classe \lstinline{Vehicle} et assignez une valeur à l'attribut \lstinline{brand}.
		\item Définissez une méthode \lstinline{honk} qui affiche \lstinline{"Tuut, tuut!"}
		\item Créez une classe-fille \lstinline{Car} qui hérite de \lstinline{Vehicle} et ayant pour attribut \lstinline{modelName} avec pour valeur par défaut \lstinline{"Mustang"}.
	\end{enumerate}

	\begin{conseil}
		Dans le constructeur de la classe-fille, n'oubliez pas de faire appel au constructeur de la classe-mère en utilisant le mot-clé \lstinline{super}
	\end{conseil}	
	
	\begin{solution}
		\lstinputlisting{solutions/Vehicle.java} 
	\end{solution}
\end{Exercice}
	
	\newpage
	
\section{Exercices complémentaires}

% Exercice nombres impaire - Maeva
\begin{Exercice}[10 minutes] \textbf{Programmation de base}\\
	Ecrivez un programme Python qui imprime tous les nombres impairs à partir de 1 jusqu’à un nombre \lstinline{n} défini par l’utilisateur. Ce nombre \lstinline{n} doit être supérieur à 1.
	Exemple : si n = 6, résultat attendu : 1, 3, 5
% Solutions nombres impaire - Maeva
	\begin{solution}
   		 \lstinputlisting{solutions/exercise3.py}
	\end{solution}
\end{Exercice}

% Ajouter des questions sur la récursivité

% Question sur la manipulation avancée de listes et dictionnaires



\end{document}