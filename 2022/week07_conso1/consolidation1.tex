\documentclass[a4paper]{article}
\usepackage{times}
\usepackage[utf8]{inputenc}
\usepackage{selinput}
\usepackage{upquote}
\usepackage[margin=2cm, rmargin=4cm, tmargin=3cm]{geometry}
\usepackage{tcolorbox}
\usepackage{xspace}
\usepackage[french]{babel}
\usepackage{url}
\usepackage{hyperref}
\usepackage{fontawesome5}
\usepackage{marginnote}
\usepackage{ulem}
\usepackage{tcolorbox}
\usepackage{graphicx}
%\usepackage[top=Bcm, bottom=Hcm, outer=Ccm, inner=Acm, heightrounded, marginparwidth=Ecm, marginparsep=Dcm]{geometry}


\newtcolorbox{Example}[1]{colback=white,left=20pt,colframe=slideblue,fonttitle=\bfseries,title=#1}
\newtcolorbox{Solutions}[1]{colback=white,left=20pt,colframe=green,fonttitle=\bfseries,title=#1}
\newtcolorbox{Conseils}[1]{colback=white,left=20pt,colframe=slideblue,fonttitle=\bfseries,title=#1}
\newtcolorbox{Warning}[1]{colback=white,left=20pt,colframe=warning,fonttitle=\bfseries,title=#1}

\setlength\parindent{0pt}

  %Exercice environment
  \newcounter{exercice}
  \newenvironment{Exercice}[1][]
  {
  \par
  \stepcounter{exercice}\textbf{Question \arabic{exercice}:} (\faClock \enskip \textit{#1})
  }
  {\bigskip}
  

% Title
\newcommand{\titre}{\begin{center}
  \section*{Algorithmes et Pensée Computationnelle}
\end{center}}
\newcommand{\cours}[1]
{\begin{center} 
  \textit{#1}\\
\end{center}
  }


\newcommand{\exemple}[1]{\newline~\textbf{Exemple :} #1}
%\newcommand{\attention}[1]{\newline\faExclamationTriangle~\textbf{Attention :} #1}

% Documentation url (escape \# in the TP document)
\newcommand{\documentation}[1]{\faBookOpen~Documentation : \href{#1}{#1}}

% Clef API
\newcommand{\apikey}[1]{\faKey~Clé API : \lstinline{#1}}
\newcommand{\apiendpoint}[1]{\faGlobe~Url de base de l'API \href{#1}{#1}}

%Listing Python style
\usepackage{color}
\definecolor{slideblue}{RGB}{33,131,189}
\definecolor{green}{RGB}{0,190,100}
\definecolor{blue}{RGB}{121,142,213}
\definecolor{grey}{RGB}{120,120,120}
\definecolor{warning}{RGB}{235,186,1}

\usepackage{listings}
\lstdefinelanguage{texte}{
    keywordstyle=\color{black},
    numbers=none,
    frame=none,
    literate=
           {é}{{\'e}}1
           {è}{{\`e}}1
           {ê}{{\^e}}1
           {à}{{\`a}}1
           {â}{{\^a}}1
           {ù}{{\`u}}1
           {ü}{{\"u}}1
           {î}{{\^i}}1
           {ï}{{\"i}}1
           {ë}{{\"e}}1
           {Ç}{{\,C}}1
           {ç}{{\,c}}1,
    columns=fullflexible,keepspaces,
	breaklines=true,
	breakatwhitespace=true,
}
\lstset{
    language=Python,
	basicstyle=\bfseries\footnotesize,
	breaklines=true,
	breakatwhitespace=true,
	commentstyle=\color{grey},
	stringstyle=\color{slideblue},
  keywordstyle=\color{slideblue},
	morekeywords={with, as, True, False, Float, join, None, main, argparse, self, sort, __eq__, __add__, __ne__, __radd__, __del__, __ge__, __gt__, split, os, endswith, is_file, scandir, @classmethod},
	deletekeywords={id},
	showspaces=false,
	showstringspaces=false,
	columns=fullflexible,keepspaces,
	literate=
           {é}{{\'e}}1
           {è}{{\`e}}1
           {ê}{{\^e}}1
           {à}{{\`a}}1
           {â}{{\^a}}1
           {ù}{{\`u}}1
           {ü}{{\"u}}1
           {î}{{\^i}}1
           {ï}{{\"i}}1
           {ë}{{\"e}}1
           {Ç}{{\,C}}1
           {ç}{{\,c}}1,
    numbers=left,
}

\newtcbox{\mybox}{nobeforeafter,colframe=white,colback=slideblue,boxrule=0.5pt,arc=1.5pt, boxsep=0pt,left=2pt,right=2pt,top=2pt,bottom=2pt,tcbox raise base}
\newcommand{\projet}{\mybox{\textcolor{white}{\small projet}}\xspace}
\newcommand{\optionnel}{\mybox{\textcolor{white}{\small Optionnel}}\xspace}
\newcommand{\advanced}{\mybox{\textcolor{white}{\small Pour aller plus loin}}\xspace}
\newcommand{\auto}{\mybox{\textcolor{white}{\small Auto-évaluation}}\xspace}


\usepackage{environ}
\newif\ifShowSolution
\NewEnviron{solution}{
  \ifShowSolution
	\begin{Solutions}{\faTerminal \enskip Solution}
		\BODY
	\end{Solutions}
  \fi}


  \usepackage{environ}
  \newif\ifShowConseil
  \NewEnviron{conseil}{
    \ifShowConseil
    \begin{Conseils}{\faLightbulb \quad Conseil}
      \BODY
    \end{Conseils}

    \fi}

    \usepackage{environ}
  \newif\ifShowWarning
  \NewEnviron{attention}{
    \ifShowWarning
    \begin{Warning}{\faExclamationTriangle \quad Attention}
      \BODY
    \end{Warning}

    \fi}
  

%\newcommand{\Conseil}[1]{\ifShowIndice\ \newline\faLightbulb[regular]~#1\fi}


\usepackage{array}
\usepackage{blindtext}
\usepackage{multicol}
\newcolumntype{C}[1]{>{\centering\let\newline\\\arraybackslash\hspace{0pt}}m{#1}}

\begin{document}
% Change the following values to true to show the solutions or/and the hints
\ShowSolutiontrue
\ShowConseiltrue
\titre
\cours{Consolidation 1}

Les exercices de cette série sont une compilation d'exercices semblables à ceux vus lors des semaines précédentes. Le but de cette séance est de consolider les connaissances acquises lors des travaux pratiques des semaines 1 à 6.

Le code présenté dans les énoncés se trouve sur Moodle, dans le dossier \lstinline{Code}.


\section{Introduction et architecture des ordinateurs}

\subsection{Conversion}

\begin{Exercice}[10 minutes]  \textbf{Conversion }\\
    \begin{enumerate}
        \item Convertir le nombre FFFFFF$_{(16)}$ en base 10.
        \item Convertir le nombre 4321$_{(5)}$ en base 10.
        \item Convertir le nombre ABC$_{(16)}$ en base 2.
        \item Convertir le nombre 11101$_{(2)}$ en base 10.
    \end{enumerate}
    
        \begin{conseil}
            N'oubliez pas qu'en Hexadécimal, A vaut 10, B vaut 11, C vaut 12, D vaut 13, E vaut 14 et F vaut 15.
    \end{conseil}

    \begin{solution}
        1. FFFFFF$_{(16)}$ = 16777215$_{(10)}$\\
        2. 4321$_{(5)}$ = 586$_{(10)}$\\
        3. ABC$_{(16)}$ = 101010111100$_{(2)}$\\
        5. 11101$_{(2)}$ = 29$_{(10)}$
    \end{solution}
\end{Exercice}

\subsection{Conversion et arithmétique}
\begin{Exercice}[5 minutes] \textbf{Conversion, addition et soustraction:}\\
    Effectuez les opérations suivantes:
    \begin{enumerate}
        \item 10110101$_{(2)}$ + 00101010$_{(2)}$ = ...$_{(10)}$
        \item 70$_{(10)}$ - 10101010$_{(2)}$ = ...$_{(10)}$
    \end{enumerate}
        \begin{conseil}
        Convertissez dans une base commune avant d'effectuer les opérations.
    \end{conseil}
        
    \begin{solution}
        1. 10110101$_{(2)}$ + 00101010$_{(2)}$ = 223$_{(10)}$\\
        3. 70$_{(10)}$ - 10101010$_{(2)}$ = -100$_{(10)}$
    \end{solution}
\end{Exercice}


\subsection{Modèle de Von Neumann}
Dans cette section, nous allons simuler une opération d'addition dans le \textbf{modèle de Von Neumann}, il va vous être demandé à chaque étape (FDES) de donner la valeur des registres.\\

\textbf{État d'origine:}\\
A l'origine, notre \lstinline{Process Counter (PC)} vaut \lstinline{00100001}.\\

Dans la mémoire, les instructions sont les suivantes:

\begin{tabular}{| C{0.1\textwidth} | C{0.1\textwidth} |} 
    \hline
    \textbf{Adresse} & \textbf{Valeur}\\ [0.5ex]
    \hline
    00100001 & 00110100\\ [0.5ex] 
    \hline
    00101100 & 10100110\\ [0.5ex] 
    \hline
    01110001 & 111111101\\ [0.5ex]
    \hline
\end{tabular}
\\\\
Les registres sont les suivants:

\begin{tabular}{| C{0.1\textwidth} | C{0.1\textwidth} |} 
    \hline
    \textbf{Registre} & \textbf{Valeur}\\ [0.5ex]
    \hline
    00 & 01111111\\ [0.5ex] 
    \hline
    01 & 00100000\\ [0.5ex] 
    \hline
    10 & 00101101\\ [0.5ex] 
    \hline
    11 & 00001100\\ [0.5ex]
    \hline
\end{tabular}
\\\\
Les opérations disponibles pour l'unité de contrôle sont les suivantes:
\\
\begin{tabular}{| C{0.1\textwidth} | C{0.1\textwidth} |} 
    \hline
    \textbf{Numéro} & \textbf{Valeur}\\ [0.5ex]
    \hline
    00 & ADD\\ [0.5ex] 
    \hline
    01 & XOR\\ [0.5ex] 
    \hline
    10 & MOV\\ [0.5ex] 
    \hline
    11 & SUB\\ [0.5ex]
    \hline
\end{tabular}
\\\\


\begin{Exercice}[5 minutes]\textbf{Fetch}\\
    À la fin de l'opération \lstinline{FETCH}, quelles sont les valeurs du \lstinline{Process Counter} et de l'\lstinline{Instruction Register}?
\end{Exercice}
   \begin{conseil}
    Pour rappel, l’unité de contrôle (Control Unit) commande et contrôle le fonctionnement du système. Elle est chargée du séquençage des opérations. Après chaque opération FETCH, la valeur du Program Counter est incrémentée (valeur initiale + 1).
    \end{conseil}
\begin{solution}
    PC = 00100001$_{(2)}$ + 1 = 00100010$_{(2)}$\\
    IR = 00110100$_{(2)}$
\end{solution}

\begin{Exercice}[5 minutes] \textbf{Decode}
    \begin{enumerate}
        \item Quelle est la valeur de l'opération à exécuter?
        \item Quelle est l'adresse du registre dans lequel le résultat doit être enregistré?
        \item Quelle est la valeur du premier nombre de l'opération?
        \item Quelle est la valeur du deuxième nombre de l'opération?
    \end{enumerate}
\end{Exercice}
   \begin{conseil}
    Pensez à décomposer la valeur de l’\lstinline{Instruction Register} pour obtenir toutes les informations demandées.\\
    Utilisez la même convention que celle présentée dans les diapositives du cours (Architecture des ordinateurs (Semaine 2) - Diapositive 15)\\
    Les données issues de la décomposition de l’\lstinline{Instruction Register} ne sont pas des valeurs brutes, mais des références. Trouvez les tables concordantes pour y récupérer les valeurs.

    \end{conseil}
\begin{solution}
    00 : \lstinline{ADD} (valeur de l'opération à exécuter)\\
    11 : Adresse du registre dans lequel le résultat doit être enregistré\\
    01 : 00100000$_{(2)}$ (premier nombre)\\
    00 : 01111111$_{(2)}$ (deuxième nombre)
\end{solution}

\begin{Exercice}[5 minutes] \textbf{Execute}\\
    Quel est résultat de l'opération?
\end{Exercice}

   \begin{conseil}
        Toutes les informations permettant d’effectuer l’opération se trouvent dans les données de l’\lstinline{Instruction Register}.
    \end{conseil}

\begin{solution}
    00100000$_{(2)}$ + 01111111$_{(2)}$ = 10011111$_{(2)}$
\end{solution}


% Week 2
\section{Logiciels système}

\begin{Exercice}[10 minutes]
    En utilisant l’invite de commandes (Terminal), créer un programme Python demandant à l’utilisateur d’entrer son nom et son prénom. Lorsque le programme est exécuté, la phrase suivante doit s’afficher : \textit{``Bonjour, je m’appelle *prénom nom*''}.
\end{Exercice}

\begin{solution}
    \lstinputlisting{solutions/terminal.py}
\end{solution}


\begin{Exercice}[5 minutes]
    Sous Linux et MacOS, laquelle de ces commandes modifie le \lstinline{filesystem}?
    \begin{enumerate}
        \item \lstinline{ls -la}
        \item \lstinline{sudo rm -rf ~/nano}
        \item \lstinline{sudo kill -9 3531}
        \item \lstinline{more nano.txt}
        \item Aucune réponse n'est correcte.
    \end{enumerate}
    \begin{solution}
        La commande \lstinline{sudo rm -rf ~/nano} permet de supprimer le répertoire \lstinline{nano} situé dans le dossier \lstinline{/Users/<Utilisateur\_courant>} en mode super-utilisateur (utilisateur ayant des droits étendus sur le système).
    \end{solution}
    \begin{conseil}
        \textbf{Attention!}\\
        Certaines commandes listées ci-dessus peuvent avoir des conséquences irréversibles.\\
        Pour avoir une description détaillée d'une commande, vous pouvez ajouter \lstinline{man} devant la commande sous Linux/MacOS ou ajouter \lstinline{-h, --help} ou \lstinline{/?} après la commande sous Windows.
    \end{conseil}
\end{Exercice}

% Week 3
\section{Programmation de base}

\begin{Exercice}[10 minutes]\\
    \begin{enumerate}
        \item Convertir 52$_{(10)}$ en base 2 sur 8 bits.
        \item Convertir 100$_{(10)}$ en base 2 sur 8 bits.
        \item Calculer en base 2 la soustraction de 01100100$_{(2)}$ par 00110100$_{(2)}$.
        \item Déterminer au complément à deux l'opposé ( multiplication par -1 en base 10) de 0110000$_{(2)}$.
    \end{enumerate}
\begin{conseil}
   % Un conseil
   \begin{itemize}
       \item Se référer aux techniques apprises dans la série 1 et la série 3
       \item Faire un tableau des puissances de 2 sur 8 bits.
   \end{itemize}
\end{conseil}
    
\begin{solution}
\begin{itemize}
    \item 52$_{(10)}$ = 32$_{(10)}$ + 16$_{(10)}$ + 4$_{(10)}$ = 00110100$_{(2)}$
    \item 100$_{(10)}$ = 64$_{(10)}$ + 32$_{(10)}$ + 4$_{(10)}$ = 01100100$_{(2)}$
    \item 01100100$_{(2)}$ - 00110100$_{(2)}$ = 0110000$_{(2)}$
    \item \textbf{Complément à 1:} not(0110000$_{(2)}$) = 1001111$_{(2)}$ 
    \item \textbf{Complément à 2:} Complément à 1 + 1 = 1001111$_{(2)}$ + 1$_{(2)}$ = 1010000$_{(2)}$
    %\lstinputlisting{solutions/fichier.java}
\end{itemize}   
\end{solution}

\end{Exercice}

\begin{Exercice}[5 minutes]
    Qu'affiche le programme suivant?
    \lstinputlisting{ressources/conditions.py}

    \begin{solution}
        output 2 \\
        output 3
    \end{solution}
\end{Exercice}

\begin{Exercice}[5 minutes]
    À partir du dictionnaire suivant :
    \begin{lstlisting}
        jours_semaine = {1:"lundi", 2:"mardi", 3:"mercredi", 4:"jeudi", 5:"jeudi", 6:"samedi" \end{lstlisting}

Modifiez la valeur associée au nombre 5 pour qu’elle soit ``vendredi''. Ajoutez le couple clé-valeur 7 : ``dimanche''. Affichez la valeur associée à la clé 3.
Grâce à une écriture en compréhension, créez une liste contenant les valeurs associées aux clés qui sont des nombres pairs.

    \begin{solution}
        \lstinputlisting{solutions/dictionary.py}
    \end{solution}
\end{Exercice}


% Week 4
\section{Fonctions, mémoire et exceptions}
\begin{Exercice}[5 minutes] \textbf{Portée des variables} (Python)\\
    Qu'affiche le programme suivant? \\

    \lstinputlisting{ressources/Question4.py}

    \begin{solution}
            x=2 \\
            x=3 \\
            None
    \end{solution}

    
 \end{Exercice}


% Week 5
\section{Itération et récursivité}

\begin{Exercice}[15 minutes] Itération et Récursivité\\

Créez une fonction itérative, puis une fonction récursive qui calculent le nombre de voyelles présentes dans un texte donné. \\

\begin{conseil}
   Pour la version itérative, parcourez toute la chaîne de caractère et incrémentez un compteur lorsque vous avez une voyelle.
   
   Pour la version récursive, diminuez systématiquement la taille de votre chaîne de caractère. Si l'élément actuel est une voyelle, ajoutez 1, sinon, ajoutez 0.
   
   Aidez vous d'une liste de toutes les voyelles et de la fonction \lstinline{in} en \lstinline{Python} (\lstinline{List.contains()} en \lstinline{Java}).
\end{conseil}

Voici les templates : \\

\textbf{Python} \\

    \lstinputlisting{ressources/iteration_recursion_exercice.py} 

\textbf{Java} \\

    \lstinputlisting{ressources/iteration_recursion_exercice.java}

    
\begin{solution}
\textbf{Python :} \\

    \lstinputlisting{solutions/iteration_recursion_solution.py}
    
\end{solution}


\begin{solution}   
\textbf{Java :} \\

    \lstinputlisting{solutions/iteration_recursion_solution.java}
    
    
    
\end{solution}

\end{Exercice}

\begin{Exercice}[5 minutes] Lecture de code (Récursivité)\\

Qu'affichent les programmes suivants ? \\

\begin{conseil}
   Ces deux programmes comportent des fonctions itératives, lisez bien le code de haut en bas et lorsque la fonction fait appel à elle-même, revenez au début de la fonction et effectuez de nouveau les instructions avec les nouveaux paramètres.
   
   Une feuille de papier pourrait vous être utile!\\
\end{conseil}

\textbf{\\Programme 1 :} \\

    \lstinputlisting{ressources/recursion_1_exercice.py} 

\textbf{Programme 2 :} \\

    \lstinputlisting{ressources/recursion_2_exercice.py}

    
\begin{solution}
    \textbf{Programme 1 :} \\

    PythonohtyP
    
\end{solution}


\begin{solution}   
    \textbf{Programme 2 :} \\

    Python
    
    adore
    
    Python
    
    J
    
    Python
    
    adore
    
    Python
    
\end{solution}

\end{Exercice}

% Week 5
\section{Fonctions, mémoire et exceptions}

\begin{Exercice}[10 minutes]Fonctions et Exceptions (Python)\\

    Écrivez une fonction \lstinline{check_name()} qui prend en argument le paramètre \lstinline{name} et qui affiche dans la console ``\textit{Bonjour suivi du nom passé en paramètre}''.
    Levez une exception de type \lstinline{ValueError} si le paramètre n'est pas de type \lstinline{str} ou si c'est une chaîne de caractères vide.

    \begin{conseil}
       Pour trouver le type d'une variable, on utilise la méthode \lstinline{type(variable)}.
    \end{conseil}
    
    \begin{solution}   
        \textbf{Python :}
        \lstinputlisting{solutions/question10.py}
    \end{solution}
    
    \end{Exercice}

\begin{Exercice}[5 minutes]Gestion d'exceptions (Java)\\

    En Java, quel mot-clé est utilisé pour lever explicitement une exception?

    \begin{enumerate}
        \item catch
        \item throw
        \item raise 
        \item try
    \end{enumerate}
    
    \begin{solution}   
    Le mot-clé pour lever une exception en Java est \lstinline{throw}.
        
    \end{solution}
        
 \end{Exercice}

 \begin{Exercice}[5 minutes]Gestion d'exceptions (Java)\\

    Qu'affiche le programme suivant ?
    \lstinputlisting{ressources/Question12.java}

    
        
        \begin{solution}   
        Le programme affiche OUTPUT 2.             
        \end{solution}
        
 \end{Exercice}



    


% Week 6
\section{Programmation orientée objet - classes et héritage}

\begin{Exercice}[15 minutes]\textbf{Poker Game}

    Cet exercice vous demande de construire un jeu de poker en implémentant un programme Java. Celui-ci~comprend trois parties importantes: un paquet de cartes (un ``\textbf{deck}''), un joueur, et une carte, et elles devraient être représentées par trois classes respectives.
    
    Suivez les instructions ci-dessous pour écrire le programme.
    
    \textbf{La classe \lstinline{Carte}}
    \begin{itemize}
        \item Attributs: \lstinline{valeur (int), couleur (int)}. Attention, la couleur d'une carte est représentée ici par une valeur de 0 à 3 au lieu d'une chaîne de caractères.
        \item Implémentez un constructeur qui prend en argument une valeur et une couleur.
        \item Implémentez un getter \lstinline{getInfo()} qui affiche dans la console la valeur et la couleur de la carte, vous pouvez vous aider de fonctions privées
        \lstinline{getCouleur()} et \lstinline{getValeur()} pour afficher les cas particuliers.
    \end{itemize}
    
    \lstinputlisting[language=java]{ressources/PokerCarte.java}
    
    
    \textbf{La classe \lstinline{Joueur}}
    \begin{itemize}
        \item Attributs: \lstinline{mainDeCartes (Carte[])} (un tableau de carte de taille maximal 2), \lstinline{balance (int)} (le solde total du joueur), \lstinline{mise (int)} (la mise du joueur), \lstinline{monTour (boolean)} (valant true si c’est au tour du joueur).
        \item Implémentez le constructeur qui prend en argument \textbf{deux cartes} (puis les ajoute à sa main), une balance initiale, une mise initiale de 0. Vous pouvez initialiser \lstinline{monTour} comme \lstinline{false} au début.
        \item Implémentez une fonction \lstinline{miser()} qui propose une mise sur la table mais ne retourne rien (utiliser le mot \lstinline{void}). Définir la nouvelle mise du joueur.
        \item Implémentez la méthode \lstinline{montrerMain()} qui montre les cartes sur la table.
        \item Implémentez une méthode \lstinline{gagner()} qui prend en argument la somme des gains sur la table que le joueur vient de gagner, ajoutez là à son solde.
        \item \textbf{Attention:} \lstinline{miser()} et \lstinline{gagner()} s'appliquent seulement si c’est au tour du joueur.
    \end{itemize}
    \lstinputlisting[language=java]{ressources/PokerJoueur.java}
    
    \textbf{La classe \lstinline{Paquet}}:
    \begin{itemize}
        \item Attributs: \lstinline{paquet (ArrayList<Carte>)} (un jeu de cartes complet), \lstinline{NBR_CARTES (int)} (un attribut final et static (constante) ayant pour valeur \lstinline{52}), \lstinline{NBR_MELANGEs (int)} (une constante qui correspond au nombre de mélanges effectués de valeur \lstinline{100}).
        \item Dans le constructeur ne prenant aucun argument, générer le deck en créant au fur et à mesure des cartes. 
        \item Implémenter une fonction public \lstinline{melanger()} qui mélange le jeu de carte et appelez la dans le constructeur après avoir construit le jeu de cartes.
        \item Implémenter une fonction getter \lstinline{getCarte()} qui retourne la carte placée en haut de la pile et la retire du jeu.
    \end{itemize}
    \begin{conseil}
        % TODO: Conseil pas clair. Merci d'expliquer à quoi sert Random et comment l'utiliser.

    Utilisez \lstinline{Random r = new Random();} sa fonction \lstinline{nextInt()} et utilisez des index et une variable \lstinline{Carte} temporaire pour pouvoir échanger les positions des cartes.
    \end{conseil}
    
    \lstinputlisting[language=java]{ressources/PokerPaquet.java}
    
    
    \begin{solution}
    \lstinputlisting[language=java]{solutions/PokerCarte.java}
    \end{solution}
    \begin{solution}
    \lstinputlisting[language=java]{solutions/PokerJoueur.java}
    \end{solution}
    \begin{solution}
    \lstinputlisting[language=java]{solutions/PokerPaquet.java}
    \end{solution}
    \end{Exercice}
    
    \begin{note}{Maeva}
        Ceci devrait être un exercice avancé
    \end{note}
    \begin{Exercice}[15 minutes]\textbf{Un jeu de rôle avec des personnages}
    
        % TODO: Attention, les étudiants n'ont pas vu les notions de classes et méthodes abstraites. Dans cet exercice, Personnage doit être défini comme étant une classe-mère normale et sans méthodes abstraites.
    Dans cet exercice, vous allez mettre en place un simple jeu de rôle. Le concept d’héritage sera très utile dans notre jeu de rôle étant donné que les différentes classes de personnages possèdent certains attributs ou actions similaires. Les personnages de notre jeu sont le Guerrier, le Paladin, le Magicien et le Chasseur. 
    
    Ainsi, il semble intéressant de construire une première classe \lstinline{Personnage}. Un personnage est un objet qui possède plusieurs arguments :
    \begin{itemize}
        \item \lstinline{nom (String)} : le nom du personnage
        \item \lstinline{niveau (int)} : le niveau du personnage
        \item \lstinline{pv (int)} : les points de vie du personnage
        \item \lstinline{vitalite (int)} : la vitalité (ou santé) du personnage
        \item \lstinline{force (int)} : la force du personnage
        \item \lstinline{dexterite (int)} : la dextérité du personnage
        \item \lstinline{endurance (int)} : l’endurance du personnage
        \item \lstinline{intelligence (int)} : l’intelligence du personnage\\\\\\
    \end{itemize}
    
    Suivez les instructions ci-dessous pour compléter le programme.
    
    \begin{itemize}
        \item Implémentez le constructeur de la classe \lstinline{Personnage} qui prend tous les attributs cités ci-dessus en argument. 
        
        % Remplacer la méthode getInfo par toString(). On pourrait demander à l'étudiant de redéfinir la méthode toString() de telle sorte qu'on puisse avoir des infos lorsqu'on fait un print sur une instance de Personnage.
        \item Il peut être intéressant d’afficher les caractéristiques de votre personnage. Après avoir implémenté les getters utiles, implémentez une méthode \lstinline{getInfo()} dans la classe \lstinline{Personnage} qui affiche des informations du personnage dans la console.
        
        \item Chaque personnage de ce jeu a un compteur de leur vie restante. Celui-ci va être manipulé par un \lstinline{setter}. Définissez le \lstinline{setter} dans la classe \lstinline{Personnage}.
        \item Implémentez les classes \lstinline{Guerrier, Paladin, Magicien et Chasseur} qui héritent de \lstinline{Personnage} en écrivant tout d’abord leur constructeur respectif. 
        
        \lstinputlisting[language=java]{ressources/RolePlayMere.java}
    
        \item Pour chacun des personnages, implémentez une méthode \lstinline{attaqueBasique()} qui prend un autre personnage en argument et ne retourne rien. Celle-ci crée une attaque de votre choix en fonction des caractéristiques des personnages (ex : l’attaque du guerrier dépendra de sa force, l’attaque du chasseur de son endurance etc..), et détermine les points de vie restants en soustrayant la gravité de l'attaque à la vitalité du personnage. Affichez le nom de celui que vous avez attaqué et ses points de vie restants.
        \item Les méthodes communes à toutes les sous-classes, doivent être déclarées abstraites dans la classe parente \lstinline{Personnage}. Changer cette classe pour qu'elle soit maintenant abstraite avec la/les méthode(s) abstraite(s) correspondante(s).
        \item \textbf{Attention :} Utiliser un ``setter'' pour réduire les \lstinline{pv} de l’autre personnage.
        
        \lstinputlisting[language=java]{ressources/RolePlayFilles.java}
    \end{itemize}
    
    \begin{solution}
        \lstinputlisting[language=java,lastline=70]{solutions/RolePlayMere.java}
    \end{solution}
    \begin{solution}
        \lstinputlisting[language=java,firstline=71, firstnumber=71]{solutions/RolePlayMere.java}
    \end{solution}
    
    \begin{solution}
        \lstinputlisting[language=java,lastline=67]{solutions/RolePlayFilles.java}
    \end{solution}
    \begin{solution}
        \lstinputlisting[language=java,firstline=68, firstnumber=68]{solutions/RolePlayFilles.java}
    \end{solution}
    \end{Exercice}
    
  
    
\section{Quiz général}

\subsection{Python}

\begin{Exercice}[2 minutes]\\
En Python, `Hello' équivaut à ``Hello''. 

\begin{enumerate}
    \item - Vrai
    \item - Faux
\end{enumerate}
\begin{solution}
    \textbf{Vrai}: En Python, les doubles guillemets et les guillemets sont équivalents. 
\end{solution}
\end{Exercice}


\begin{Exercice}[2 minutes]\\
Dans une fonction, nous pouvons utiliser les instructions \lstinline{print()} ou \lstinline{return}, elles ont le même rôle.
\begin{enumerate}
    \item - Vrai
    \item - Faux
\end{enumerate}
\begin{solution}
    \textbf{Faux}\\
    \lstinline{print()} permet uniquement d'afficher un message dans la console. Autrement dit, \lstinline{print()} sert à communiquer un message à l'utilisateur final du programme, celui-ci n'ayant pas accès au code.\\\\
    \lstinline{return} est une déclaration qui s'utilise à l'intérieur d'une fonction pour renvoyer le résultat de la fonction lorsqu'elle a été appelée. Exemple: la fonction \lstinline{len(L)} renvoie la longueur de la liste~L.\\

    Admettons que nous ayons une fonction qui renvoie le double d'un nombre. Dans notre programme, nous souhaitons ajouter le résultat de cette fonction à un nombre quelconque.
    En Python, notre programme ressemblera à ça:
    \lstinputlisting{ressources/question16.py}

    Notre programme renverra une erreur de type \lstinline{TypeError: unsupported operand type(s) for +: 'int' and 'NoneType'} si on remplace la ligne 2 par \lstinline{print(nombre**2)} car la fonction \lstinline{double_nombre} renverra \lstinline{None}.
\end{solution}
\end{Exercice}


\begin{Exercice}[2 minutes]\\
Lorsqu'on fait appel à une fonction, les arguments doivent nécessairement avoir le(s) même(s) noms tel(s) que définit dans la fonction. Exemple:\\
\begin{lstlisting}
def recherche_lineaire(Liste, x):
    for i in Liste:
        if i == x:
            return x in Liste
    return -1

Liste = [1,3,5,7,9]
x = 3

recherche_lineaire(Liste,x)

\end{lstlisting}
\begin{enumerate}
    \item - Vrai
    \item - Faux
\end{enumerate}
\begin{solution}
    \textbf{Faux}\\
    Le nom des variables données en argument n'a aucune importance tant que le type de variable est respecté. Dans notre exemple, la fonction s'attend à recevoir  une \textbf{liste} comme premier argument et un \textbf{entier} comme deuxième argument. Ici, nous aurions pu nommer la liste \lstinline{nbr_impair} et x \lstinline{valeur} et appeler la fonction \lstinline{recherche_lineaire(nbr_impair, valeur)}

\end{solution}
\end{Exercice}

\begin{Exercice}[2 minutes]\\
Si le programme Python contient une erreur, celle-ci sera détectée avant l'exécution du programme. 

\begin{enumerate}
    \item - Vrai
    \item - Faux
\end{enumerate}
\begin{solution}
    \textbf{Faux}: En Python, les erreurs sont détectées pendant l'exécution du programme.
\end{solution}
\end{Exercice}


\begin{Exercice}[2 minutes]\\
Il est possible de faire appel à une fonction définie ``plus bas'' dans le code sans que cela ne pose problème.

\begin{lstlisting}
import math

nombre_decimal_pi(4)

def nombre_decimal_pi(int):
    return round(math.pi,int)
\end{lstlisting}

\begin{enumerate}
    \item - Vrai
    \item - Faux
\end{enumerate}
\begin{solution}
    \textbf{Faux}: À l'exception des fonctions intégrées (il s'agit des fonctions déjà intégrées au langage Python telles que \lstinline{print(), len(), abs(), etc}...). Dans les langages suivant une~logique de~programmation~impérative (c'est le cas de Java et Python), une fonction doit nécessairement être définie \textbf{avant} d'être appelée.
    \url{https://fr.wikipedia.org/wiki/Programmation\_imp\%C3\%A9rative}
\end{solution}
\end{Exercice}


\begin{Exercice}[5 minutes]\\
Les trois fonctions suivantes renvoient-elles systématiquement des résultats identiques ?\\Les fonctions sont censées retourner le nombre \lstinline{pi} avec le nombre de décimales (au moins une et au maximum 15) indiqué en paramètre.
\begin{multicols}{3}
\begin{lstlisting}
import math

def nombre_decimal_pi(int):
    if int > 15:
        int = 15
    elif int < 0:
        int = 1
    resultat = round(math.pi,int) 
    return resultat

print(nombre_decimal_pi(-2))
print(nombre_decimal_pi(4))
print(nombre_decimal_pi(20))



\end{lstlisting}
\columnbreak

\begin{lstlisting}
import math

def nombre_decimal_pi(int):
    if int > 15:
        resultat = round(math.pi,15)
    elif int < 0:
        resultat = round(math.pi,1)
    else: 
        resultat = round(math.pi,int)
    return resultat

print(nombre_decimal_pi(-2))
print(nombre_decimal_pi(4))
print(nombre_decimal_pi(20))

\end{lstlisting}
\columnbreak

\begin{lstlisting}
import math

def nombre_decimal_pi(int):
    if int > 15:
        return round(math.pi,15)
    elif int < 0:
        return round(math.pi,1)
    else: 
        return round(math.pi,int)
    

print(nombre_decimal_pi(-2))
print(nombre_decimal_pi(4))
print(nombre_decimal_pi(20))

\end{lstlisting}
\end{multicols}

\begin{enumerate}
    \item - Vrai
    \item - Faux
\end{enumerate}
\begin{solution}
    \textbf{Vrai}: Les trois fonctions produisent des résultats identiques. Si besoin, exécutez le code dans IntelliJ.
\end{solution}
\end{Exercice}
\subsection{Java}


\begin{Exercice}[3 minutes] \\
Observez les deux programmes suivants en Java. Lequel a-t-il une bonne structure et peut être compilé sans erreur ?

\begin{multicols}{2}
\begin{lstlisting}
//Programme A
public class Main {

    public static void main(String[] args) {
        ma_function();
        autre_fonction();
    

        static void ma_function(){
            System.out.println("Voici ma fonction!");
        }
    
        static void autre_fonction(){
            System.out.println("Une autre fonction!");
        }
    }
}



\end{lstlisting}
\columnbreak

\begin{lstlisting}
//Programme B
public class Main {

    public static void main(String[] args) {
    ma_function();
    autre_fonction();
    }

    static void ma_function(){
        System.out.println("Voici ma fonction!");
    }

    static void autre_fonction(){
        System.out.println("Une autre fonction!");
    }
}
\end{lstlisting}
\columnbreak

\end{multicols}
\begin{enumerate}
    \item Programme A
    \item Programme B
\end{enumerate}
\begin{solution}
    \textbf{Le programme B}\\
    Le fichier dans son ensemble représente une classe Java, ici la classe s'appelle \textbf{Main} (Vous aurez plus d'informations sur le fonctionnement des classes lorsque nous aborderons la programmation orientée objet). À l'intérieur de cette classe se trouve la méthode \lstinline{public static void main()}, il s'agit de la porte d'entrée de notre programme, celle que l'on exécute et celle dans laquelle nous rédigeons notre code.\\
    Les autres fonctions, qui peuvent être appelées, se définissent au sein de la classe au même échelon que la fonction \lstinline{public static void main()} comme dans le programme B ci-dessus. 
\end{solution}
\end{Exercice}

\begin{Exercice}[2 minutes] Exercice 1\\
L'indentation des lignes de code en Java est aussi importante qu'en Python.
\begin{enumerate}
    \item - Vrai
    \item - Faux
\end{enumerate}
\begin{solution}
    \textbf{Faux}
    \\En Java, le compilateur ne prend pas en compte l'indentation pour interpréter le programme, il comprend la structure à l'aide des parenthèses, des accolades et encore des point-virgules qui indiquent la fin d'une instruction. Toutefois, l'indentation est un aspect important de la programmation car elle sert à bien structurer visuellement votre code.\\\\
    En Python, l'indentation définit la structure de votre code. Elle est donc indispensable pour la bonne interprétation et la bonne exécution de votre programme. 
\end{solution}
\end{Exercice}


\end{document}
