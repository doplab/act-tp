\documentclass[a4paper]{article}
\usepackage{times}
\usepackage[utf8]{inputenc}
\usepackage{selinput}
\usepackage{upquote}
\usepackage[margin=2cm, rmargin=4cm, tmargin=3cm]{geometry}
\usepackage{tcolorbox}
\usepackage{xspace}
\usepackage[french]{babel}
\usepackage{url}
\usepackage{hyperref}
\usepackage{fontawesome5}
\usepackage{marginnote}
\usepackage{ulem}
\usepackage{tcolorbox}
\usepackage{graphicx}
\usepackage{verbatimbox}
\usepackage{amsmath}
\usepackage{hyperref}
%\usepackage[top=Bcm, bottom=Hcm, outer=Ccm, inner=Acm, heightrounded, marginparwidth=Ecm, marginparsep=Dcm]{geometry}


\newtcolorbox{Example}[1]{colback=white,left=20pt,colframe=slideblue,fonttitle=\bfseries,title=#1}
\newtcolorbox{Solutions}[1]{colback=white,left=20pt,colframe=green,fonttitle=\bfseries,title=#1}
\newtcolorbox{Conseils}[1]{colback=white,left=20pt,colframe=slideblue,fonttitle=\bfseries,title=#1}
\newtcolorbox{Warning}[1]{colback=white,left=20pt,colframe=warning,fonttitle=\bfseries,title=#1}

\setlength\parindent{0pt}

  %Exercice environment
  \newcounter{exercice}
  \newenvironment{Exercice}[1][]
  {
  \par
  \stepcounter{exercice}\textbf{Question \arabic{exercice}:} (\faClock \enskip \textit{#1})
  }
  {\bigskip}
  

% Title
\newcommand{\titre}{\begin{center}
  \section*{Algorithmes et Pensée Computationnelle}
\end{center}}
\newcommand{\cours}[1]
{\begin{center} 
  \textit{#1}\\
\end{center}
  }


\newcommand{\exemple}[1]{\newline~\textbf{Exemple :} #1}
%\newcommand{\attention}[1]{\newline\faExclamationTriangle~\textbf{Attention :} #1}

% Documentation url (escape \# in the TP document)
\newcommand{\documentation}[1]{\faBookOpen~Documentation : \href{#1}{#1}}

% Clef API
\newcommand{\apikey}[1]{\faKey~Clé API : \lstinline{#1}}
\newcommand{\apiendpoint}[1]{\faGlobe~Url de base de l'API \href{#1}{#1}}

%Listing Python style
\usepackage{color}
\definecolor{slideblue}{RGB}{33,131,189}
\definecolor{green}{RGB}{0,190,100}
\definecolor{blue}{RGB}{121,142,213}
\definecolor{grey}{RGB}{120,120,120}
\definecolor{warning}{RGB}{235,186,1}

\usepackage{listings}
\lstdefinelanguage{texte}{
    keywordstyle=\color{black},
    numbers=none,
    frame=none,
    literate=
           {é}{{\'e}}1
           {è}{{\`e}}1
           {ê}{{\^e}}1
           {à}{{\`a}}1
           {â}{{\^a}}1
           {ù}{{\`u}}1
           {ü}{{\"u}}1
           {î}{{\^i}}1
           {ï}{{\"i}}1
           {ë}{{\"e}}1
           {Ç}{{\,C}}1
           {ç}{{\,c}}1,
    columns=fullflexible,keepspaces,
	breaklines=true,
	breakatwhitespace=true,
}
\lstset{
    language=Python,
	basicstyle=\bfseries\footnotesize,
	breaklines=true,
	breakatwhitespace=true,
	commentstyle=\color{grey},
	stringstyle=\color{slideblue},
  keywordstyle=\color{slideblue},
	morekeywords={with, as, True, False, Float, join, None, main, argparse, self, sort, __eq__, __add__, __ne__, __radd__, __del__, __ge__, __gt__, split, os, endswith, is_file, scandir, @classmethod},
	deletekeywords={id},
	showspaces=false,
	showstringspaces=false,
	columns=fullflexible,keepspaces,
	literate=
           {é}{{\'e}}1
           {è}{{\`e}}1
           {ê}{{\^e}}1
           {à}{{\`a}}1
           {â}{{\^a}}1
           {ù}{{\`u}}1
           {ü}{{\"u}}1
           {î}{{\^i}}1
           {ï}{{\"i}}1
           {ë}{{\"e}}1
           {Ç}{{\,C}}1
           {ç}{{\,c}}1,
    numbers=left,
}

\newtcbox{\mybox}{nobeforeafter,colframe=white,colback=slideblue,boxrule=0.5pt,arc=1.5pt, boxsep=0pt,left=2pt,right=2pt,top=2pt,bottom=2pt,tcbox raise base}
\newcommand{\projet}{\mybox{\textcolor{white}{\small projet}}\xspace}
\newcommand{\optionnel}{\mybox{\textcolor{white}{\small Optionnel}}\xspace}
\newcommand{\auto}{\mybox{\textcolor{white}{\small Auto-évaluation}}\xspace}


\usepackage{environ}
\newif\ifShowSolution
\NewEnviron{solution}{
  \ifShowSolution
	\begin{Solutions}{\faTerminal \enskip Solution}
		\BODY
	\end{Solutions}
  \fi}


  \usepackage{environ}
  \newif\ifShowConseil
  \NewEnviron{conseil}{
    \ifShowConseil
    \begin{Conseils}{\faLightbulb \quad Conseil}
      \BODY
    \end{Conseils}

    \fi}

    \usepackage{environ}
  \newif\ifShowWarning
  \NewEnviron{attention}{
    \ifShowWarning
    \begin{Warning}{\faExclamationTriangle \quad Attention}
      \BODY
    \end{Warning}

    \fi}
  

%\newcommand{\Conseil}[1]{\ifShowIndice\ \newline\faLightbulb[regular]~#1\fi}



% Language setting
% Replace `english' with e.g. `spanish' to change the document language


% Useful packages
\usepackage{amsmath}
\usepackage{graphicx}

\title{Exercices pour la séance de révision}
\author{Abiola Adeye: abiola.adeye@epfl.ch}

% TODO: Add hints and solutions to your exercises



\begin{document}

% Change the following values to true to show the solutions or/and the hints
\ShowSolutiontrue
\ShowConseiltrue

\maketitle

\section*{Semaine 1}
\begin{Exercice}[15 minutes] \textbf{Soustraction de nombres binaires}\\
    Effectuer les opérations suivantes:
    \begin{enumerate}
        \item 01111111$_{(2)}$ - 01000000$_{(2)}$
        \item 10000000$_{(2)}$ - 00000001$_{(2)}$
        \item 10101010$_{(2)}$ - 01010101$_{(2)}$
    \end{enumerate}
    
\end{Exercice}

\section*{Semaine 2}

\begin{Exercice}[10 minutes]
        En utilisant l'invite de commande (Terminal), exécutez le programme Python suivant en lui passant des paramètres.

        \lstinputlisting[language=Python]{question8.py}
        

\end{Exercice}

\section*{Semaine 3}
\begin{Exercice}[10 minutes] \textbf{Floating point}\\
    
    Voici la représentation en binaire d'un nombre à virgule flottante: \\
    
     \begin{tabular}{| p{1cm} | p{3cm} | p{9.5cm} | p{1cm} | p{1cm} | p{1cm} | p{1cm} | p{1cm} | p{1cm} |} 
            \hline
            signe & exposant & mantisse \\ [0.5ex] 
            \hline
            0 & 10110101 & 01000001000000000000001 \\ [0.5ex]
            \hline
	\end{tabular}
	\\
    Que vaut cette représentation en base 10 ? Utiliser la représentation des floating points (avec un biais de 127). Arrondir les résultats intermédiaires et la valeur finale au 3ème chiffre significatif après la virgule. \\
	
    
\end{Exercice}

\section*{Semaine 4}

\begin{Exercice}[5 minutes] \textbf{Portée des variables} (Python)\\
    Qu'affiche le programme suivant? \\

    \lstinputlisting{Question4.py}

    \begin{solution}
            x=2 \\
            x=3 \\
            None
    \end{solution}

    
 \end{Exercice}
 
 
 \section*{Semaine 5}
 
 \begin{Exercice}[10 minutes] \textbf{Manipulation des listes en Java}\\
      	Créez une liste nommée \lstinline{ma_liste} contenant les nombres 1,2,3,4 et 5. Affichez le deuxième élément de la liste ainsi que la taille de la liste. \\
      	
      	Créez une liste \lstinline{ma_liste_m} liée à la liste \lstinline{ma_liste}. Ajoutez le chiffre 6 à la fin de la liste, et le chiffre 0 au début de cette dernière. \\
      	
      	Ajoutez ceci au début de votre code:
      	\begin{lstlisting}[language=Java]
             
import java.util.List;
import java.util.LinkedList; \end{lstlisting}
	     
	     Ajoutez ceci à la fin de votre code:
	     
	    \begin{lstlisting}[language=Java]
for(int i=0;i<ma_liste_m.size();i++){
	System.out.println(ma_liste_m.get(i));
} \end{lstlisting} 
    
\end{Exercice}


\section*{Semaine 6}       

\begin{Exercice}[10 minutes] Création de classe et encapsulation\\
    Créez une classe \lstinline{Dog} contenant les attributs suivants:
    \begin{enumerate}
    \item Un attribut \lstinline{public} String nommé \lstinline{name}
    \item Un attribut \lstinline{private} List nommé \lstinline{tricks}
    \item Un attribut \lstinline{private} String nommé \lstinline{race}
    \item Un attribut \lstinline{private} int nommé \lstinline{age}
    \item Un attribut \lstinline{private} int nommé \lstinline{mood} initialisé à 5 (correspondant à l'humeur du chien)
    \item Un attribut de classe (\lstinline{static}) \lstinline{private} int nommé \lstinline{nb_chiens}
   	\end{enumerate}
   	
   	Créez une méthode publique du même nom que la classe (\lstinline{Dog}). Cette méthode est appelée le \lstinline{constructeur}, elle va servir à initialiser les différentes instances de notre classe. Un \lstinline{constructeur} en \lstinline{Java} aura le même nom que la classe, et le \lstinline{constructeur} en \lstinline{Python} sera défini par la méthode \lstinline{__init__}. Cette méthode prendra en argument les éléments suivants qui seront utilisés pour initialiser les attributs de notre instance :
   	\begin{enumerate}
    \item Une chaîne de caractères \lstinline{name},
    \item Une liste \lstinline{tricks},
    \item Une chaîne de caractères \lstinline{race},
    \item Un entier \lstinline{age}.
   	\end{enumerate}
   	
   	Pour finir, cette méthode doit incrémenter l'attribut de classe \lstinline{nb_chiens} qui va garder en mémoire le nombre d'instances crées.
   	

\end{Exercice}

\end{document}
