\documentclass[a4paper]{article}
\usepackage{times}
\usepackage[utf8]{inputenc}
\usepackage{selinput}
\usepackage{upquote}
\usepackage[margin=2cm, rmargin=4cm, tmargin=3cm]{geometry}
\usepackage{tcolorbox}
\usepackage{xspace}
\usepackage[french]{babel}
\usepackage{url}
\usepackage{hyperref}
\usepackage{fontawesome5}
\usepackage{marginnote}
\usepackage{ulem}
\usepackage{tcolorbox}
\usepackage{graphicx}
%\usepackage[top=Bcm, bottom=Hcm, outer=Ccm, inner=Acm, heightrounded, marginparwidth=Ecm, marginparsep=Dcm]{geometry}


\newtcolorbox{Example}[1]{colback=white,left=20pt,colframe=slideblue,fonttitle=\bfseries,title=#1}
\newtcolorbox{Solutions}[1]{colback=white,left=20pt,colframe=green,fonttitle=\bfseries,title=#1}
\newtcolorbox{Conseils}[1]{colback=white,left=20pt,colframe=slideblue,fonttitle=\bfseries,title=#1}
\newtcolorbox{Warning}[1]{colback=white,left=20pt,colframe=warning,fonttitle=\bfseries,title=#1}

\setlength\parindent{0pt}

  %Exercice environment
  \newcounter{exercice}
  \newenvironment{Exercice}[1][]
  {
  \par
  \stepcounter{exercice}\textbf{Question \arabic{exercice}:} (\faClock \enskip \textit{#1})
  }
  {\bigskip}
  

% Title
\newcommand{\titre}{\begin{center}
  \section*{Algorithmes et Pensée Computationnelle}
\end{center}}
\newcommand{\cours}[1]
{\begin{center} 
  \textit{#1}\\
\end{center}
  }


\newcommand{\exemple}[1]{\newline~\textbf{Exemple :} #1}
%\newcommand{\attention}[1]{\newline\faExclamationTriangle~\textbf{Attention :} #1}

% Documentation url (escape \# in the TP document)
\newcommand{\documentation}[1]{\faBookOpen~Documentation : \href{#1}{#1}}

% Clef API
\newcommand{\apikey}[1]{\faKey~Clé API : \lstinline{#1}}
\newcommand{\apiendpoint}[1]{\faGlobe~Url de base de l'API \href{#1}{#1}}

%Listing Python style
\usepackage{color}
\definecolor{slideblue}{RGB}{33,131,189}
\definecolor{green}{RGB}{0,190,100}
\definecolor{blue}{RGB}{121,142,213}
\definecolor{grey}{RGB}{120,120,120}
\definecolor{warning}{RGB}{235,186,1}

\usepackage{listings}
\lstdefinelanguage{texte}{
    keywordstyle=\color{black},
    numbers=none,
    frame=none,
    literate=
           {é}{{\'e}}1
           {è}{{\`e}}1
           {ê}{{\^e}}1
           {à}{{\`a}}1
           {â}{{\^a}}1
           {ù}{{\`u}}1
           {ü}{{\"u}}1
           {î}{{\^i}}1
           {ï}{{\"i}}1
           {ë}{{\"e}}1
           {Ç}{{\,C}}1
           {ç}{{\,c}}1,
    columns=fullflexible,keepspaces,
	breaklines=true,
	breakatwhitespace=true,
}
\lstset{
    language=Python,
	basicstyle=\bfseries\footnotesize,
	breaklines=true,
	breakatwhitespace=true,
	commentstyle=\color{grey},
	stringstyle=\color{slideblue},
  keywordstyle=\color{slideblue},
	morekeywords={with, as, True, False, Float, join, None, main, argparse, self, sort, __eq__, __add__, __ne__, __radd__, __del__, __ge__, __gt__, split, os, endswith, is_file, scandir, @classmethod},
	deletekeywords={id},
	showspaces=false,
	showstringspaces=false,
	columns=fullflexible,keepspaces,
	literate=
           {é}{{\'e}}1
           {è}{{\`e}}1
           {ê}{{\^e}}1
           {à}{{\`a}}1
           {â}{{\^a}}1
           {ù}{{\`u}}1
           {ü}{{\"u}}1
           {î}{{\^i}}1
           {ï}{{\"i}}1
           {ë}{{\"e}}1
           {Ç}{{\,C}}1
           {ç}{{\,c}}1,
    numbers=left,
}

\newtcbox{\mybox}{nobeforeafter,colframe=white,colback=slideblue,boxrule=0.5pt,arc=1.5pt, boxsep=0pt,left=2pt,right=2pt,top=2pt,bottom=2pt,tcbox raise base}
\newcommand{\projet}{\mybox{\textcolor{white}{\small projet}}\xspace}
\newcommand{\optionnel}{\mybox{\textcolor{white}{\small Optionnel}}\xspace}
\newcommand{\advanced}{\mybox{\textcolor{white}{\small Pour aller plus loin}}\xspace}
\newcommand{\auto}{\mybox{\textcolor{white}{\small Auto-évaluation}}\xspace}


\usepackage{environ}
\newif\ifShowSolution
\NewEnviron{solution}{
  \ifShowSolution
	\begin{Solutions}{\faTerminal \enskip Solution}
		\BODY
	\end{Solutions}
  \fi}


  \usepackage{environ}
  \newif\ifShowConseil
  \NewEnviron{conseil}{
    \ifShowConseil
    \begin{Conseils}{\faLightbulb \quad Conseil}
      \BODY
    \end{Conseils}

    \fi}

    \usepackage{environ}
  \newif\ifShowWarning
  \NewEnviron{attention}{
    \ifShowWarning
    \begin{Warning}{\faExclamationTriangle \quad Attention}
      \BODY
    \end{Warning}

    \fi}
  

%\newcommand{\Conseil}[1]{\ifShowIndice\ \newline\faLightbulb[regular]~#1\fi}



\usepackage{listings}
\usepackage{array}
\newcolumntype{C}[1]{>{\centering\let\newline\\\arraybackslash\hspace{0pt}}m{#1}}

\begin{document}

% Change the following values to true to show the solutions or/and the hints
\ShowSolutiontrue
\ShowConseiltrue
\ShowWarningtrue
\titre
\cours{Fonctions, mémoire et exceptions - suite}

Le but de cette séance est d'approfondir vos connaissances en programmation. Au terme de cette séance, l'étudiant sera capable de :
\begin{itemize}
    \item définir une fonction et l'utiliser dans un programme,
    \item utiliser des librairies contenant des fonctions prédéfinies,
    \item connaître quelle est la portée d'une variable,
    \item comprendre comment fonctionne une pile d'exécution (call stack)
    \item gérer des exceptions.
\end{itemize}

\section{Gestion des exceptions}

\begin{Exercice}[5 minutes] Qu'affiche le programme suivant? \textbf{Types d'erreurs (Python)}\\
    \lstinputlisting{resources/Questions/tp4_Question1.py}  
     \begin{conseil}
        Utilisée sur une chaîne de caractères, la fonction \lstinline{len()} renvoie le nombre de caractères de la chaîne. 
    \end{conseil}
     \begin{solution}
        \lstinline{Le résultat de la division est: 2.5}\\
        \lstinline{On ne peut pas diviser une chaîne de caractères.}
     \end{solution}   
 \end{Exercice}

\begin{Exercice}[5 minutes] Qu'affiche le programme suivant? \textbf{Types d'erreurs (Python)}\\
    \lstinputlisting{resources/Questions/tp4_Question2.py}
    
     \begin{conseil}
         La chaîne de caractères vide est représentée par ``'' 
    \end{conseil}
     \begin{solution}
        Nous ne pouvons pas diviser un nombre par 0.
     \end{solution}   
 \end{Exercice}

 \begin{Exercice}[5 minutes] Qu'affiche le programme suivant? \textbf{Types d'erreurs (Python)}\\
    \lstinputlisting{resources/Questions/tp4_Question3.py}
    
     \begin{conseil}
        \begin{itemize}
            \item La fonction \lstinline{float()} permet de convertir une variable en nombre à décimal.
            \item La fonction \lstinline{int()} permet de convertir une variable en nombre entier.
        \end{itemize}  
    \end{conseil}

     \begin{solution}
        Nous ne pouvons pas convertir une chaîne de caractères.
     \end{solution}

\end{Exercice}
 
\begin{Exercice}[5 minutes] \textbf{Gestion d'erreurs (Java)} 
    Le programme suivant est incorrect. Que devez-vous modifier pour qu'il fonctionne correctement?
    \\
    
    \lstinputlisting{resources/Questions/tp4_Question4.java}

    
     \begin{conseil}
        Référez-vous à la diapositive 26 du cours de cette semaine.           
     \end{conseil}
     \begin{solution}
        À l'intérieur de la fonction \lstinline{division}, lorsque b est égal à 0, le programme lève une exception de type \lstinline{ArithmeticException}. Mais dans le \lstinline{main}, l'erreur qui est interceptée est \lstinline{IndexOutOfBoundsException}. Pour corriger cette erreur, une des solutions est de remplacer \lstinline{IndexOutOfBoundsException} par \lstinline{ArithmeticException}.
     \end{solution}   
 \end{Exercice}

\end{document}
