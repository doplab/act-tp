\documentclass[10pt]{article}
\usepackage[usenames]{color} %pour la couleur
\usepackage{amssymb} %maths
\usepackage{amsmath} %maths
\usepackage[utf8]{inputenc} %utile pour taper directement les caractères accentués
\begin{document}
\begin{align*}\documentclass[a4paper]{article}
\usepackage{times}
\usepackage[utf8]{inputenc}
\usepackage{selinput}
\usepackage{upquote}
\usepackage[margin=2cm, rmargin=4cm, tmargin=3cm]{geometry}
\usepackage{tcolorbox}
\usepackage{xspace}
\usepackage[french]{babel}
\usepackage{url}
\usepackage{hyperref}
\usepackage{fontawesome5}
\usepackage{marginnote}
\usepackage{ulem}
\usepackage{tcolorbox}
\usepackage{graphicx}
\usepackage{verbatimbox}
\usepackage{amsmath}
\usepackage{hyperref}
%\usepackage[top=Bcm, bottom=Hcm, outer=Ccm, inner=Acm, heightrounded, marginparwidth=Ecm, marginparsep=Dcm]{geometry}


\newtcolorbox{Example}[1]{colback=white,left=20pt,colframe=slideblue,fonttitle=\bfseries,title=#1}
\newtcolorbox{Solutions}[1]{colback=white,left=20pt,colframe=green,fonttitle=\bfseries,title=#1}
\newtcolorbox{Conseils}[1]{colback=white,left=20pt,colframe=slideblue,fonttitle=\bfseries,title=#1}
\newtcolorbox{Warning}[1]{colback=white,left=20pt,colframe=warning,fonttitle=\bfseries,title=#1}

\setlength\parindent{0pt}

  %Exercice environment
  \newcounter{exercice}
  \newenvironment{Exercice}[1][]
  {
  \par
  \stepcounter{exercice}\textbf{Question \arabic{exercice}:} (\faClock \enskip \textit{#1})
  }
  {\bigskip}
  

% Title
\newcommand{\titre}{\begin{center}
  \section*{Algorithmes et Pensée Computationnelle}
\end{center}}
\newcommand{\cours}[1]
{\begin{center} 
  \textit{#1}\\
\end{center}
  }


\newcommand{\exemple}[1]{\newline~\textbf{Exemple :} #1}
%\newcommand{\attention}[1]{\newline\faExclamationTriangle~\textbf{Attention :} #1}

% Documentation url (escape \# in the TP document)
\newcommand{\documentation}[1]{\faBookOpen~Documentation : \href{#1}{#1}}

% Clef API
\newcommand{\apikey}[1]{\faKey~Clé API : \lstinline{#1}}
\newcommand{\apiendpoint}[1]{\faGlobe~Url de base de l'API \href{#1}{#1}}

%Listing Python style
\usepackage{color}
\definecolor{slideblue}{RGB}{33,131,189}
\definecolor{green}{RGB}{0,190,100}
\definecolor{blue}{RGB}{121,142,213}
\definecolor{grey}{RGB}{120,120,120}
\definecolor{warning}{RGB}{235,186,1}

\usepackage{listings}
\lstdefinelanguage{texte}{
    keywordstyle=\color{black},
    numbers=none,
    frame=none,
    literate=
           {é}{{\'e}}1
           {è}{{\`e}}1
           {ê}{{\^e}}1
           {à}{{\`a}}1
           {â}{{\^a}}1
           {ù}{{\`u}}1
           {ü}{{\"u}}1
           {î}{{\^i}}1
           {ï}{{\"i}}1
           {ë}{{\"e}}1
           {Ç}{{\,C}}1
           {ç}{{\,c}}1,
    columns=fullflexible,keepspaces,
	breaklines=true,
	breakatwhitespace=true,
}
\lstset{
    language=Python,
	basicstyle=\bfseries\footnotesize,
	breaklines=true,
	breakatwhitespace=true,
	commentstyle=\color{grey},
	stringstyle=\color{slideblue},
  keywordstyle=\color{slideblue},
	morekeywords={with, as, True, False, Float, join, None, main, argparse, self, sort, __eq__, __add__, __ne__, __radd__, __del__, __ge__, __gt__, split, os, endswith, is_file, scandir, @classmethod},
	deletekeywords={id},
	showspaces=false,
	showstringspaces=false,
	columns=fullflexible,keepspaces,
	literate=
           {é}{{\'e}}1
           {è}{{\`e}}1
           {ê}{{\^e}}1
           {à}{{\`a}}1
           {â}{{\^a}}1
           {ù}{{\`u}}1
           {ü}{{\"u}}1
           {î}{{\^i}}1
           {ï}{{\"i}}1
           {ë}{{\"e}}1
           {Ç}{{\,C}}1
           {ç}{{\,c}}1,
    numbers=left,
}

\newtcbox{\mybox}{nobeforeafter,colframe=white,colback=slideblue,boxrule=0.5pt,arc=1.5pt, boxsep=0pt,left=2pt,right=2pt,top=2pt,bottom=2pt,tcbox raise base}
\newcommand{\projet}{\mybox{\textcolor{white}{\small projet}}\xspace}
\newcommand{\optionnel}{\mybox{\textcolor{white}{\small Optionnel}}\xspace}
\newcommand{\auto}{\mybox{\textcolor{white}{\small Auto-évaluation}}\xspace}


\usepackage{environ}
\newif\ifShowSolution
\NewEnviron{solution}{
  \ifShowSolution
	\begin{Solutions}{\faTerminal \enskip Solution}
		\BODY
	\end{Solutions}
  \fi}


  \usepackage{environ}
  \newif\ifShowConseil
  \NewEnviron{conseil}{
    \ifShowConseil
    \begin{Conseils}{\faLightbulb \quad Conseil}
      \BODY
    \end{Conseils}

    \fi}

    \usepackage{environ}
  \newif\ifShowWarning
  \NewEnviron{attention}{
    \ifShowWarning
    \begin{Warning}{\faExclamationTriangle \quad Attention}
      \BODY
    \end{Warning}

    \fi}
  

%\newcommand{\Conseil}[1]{\ifShowIndice\ \newline\faLightbulb[regular]~#1\fi}


\usepackage{array}
\newcolumntype{C}[1]{>{\centering\let\newline\\\arraybackslash\hspace{0pt}}m{#1}}

\begin{document}
% Change the following values to true to show the solutions or/and the hints
\ShowSolutiontrue
\ShowConseiltrue
\titre
\cours{Programmation orientée objet: Héritage et Polymorphisme - exercices basiques}

Le but de cette séance est d'approfondir les notions de programmation orientée objet vues précédemment. Les exercices sont construits autour des concepts d'héritage, de surcharge d'opérateurs/méthodes et de polymorphisme. Au terme de cette séance, vous devez être en mesure de factoriser votre code afin de le rendre mieux structuré et plus lisible. À chaque exercice, le langage de programmation à utiliser sera spécifié.

Le code présenté dans les énoncés se trouve sur Moodle, dans le dossier \lstinline{Ressources}.

\section{Notions d'héritage - Java}

Le but de cette partie est de mettre en pratique les notions liées à l'héritage. 
Nous allons créer une classe \lstinline{Livre()} qui représentera notre classe mère. Nous allons également créer deux classes filles, \lstinline{Livre_Audio()} et \lstinline{Livre_Illustre()}. Les classes filles hériteront des attributs et méthodes de la classe mère. 

\begin{Exercice}[20 minutes] \lstinline{Création des différentes classes}\\
Créez la classe mère \lstinline{Livre} avec les caractéristiques suivantes:
\begin{itemize}
	\item un attribut \lstinline{privé String} nommé \lstinline{titre},
	\item un attribut \lstinline{privé String} nommé \lstinline{auteur},
	\item un attribut \lstinline{privé int} nommé \lstinline{annee},
	\item un attribut \lstinline{privé int} nommé \lstinline{note} (initialisé à \lstinline{-1}),
	\item le \lstinline{constructeur} de la classe qui prendra les trois premiers arguments cités ci-dessus,
	\item une méthode \lstinline{setNote()} qui permet de définir l'attribut \lstinline{note},
	\item une méthode \lstinline{getNote()} qui permet de retourner l'attribut \lstinline{note},
	\item une méthode \lstinline{toString()} qui retournera le titre, l'auteur, l'année et la note d'un ouvrage \lstinline{note} (réécrire cette méthode permettra d'afficher un objet \lstinline{Livre} en utilisant \lstinline{System.out.println()} \\
\end{itemize}

Attention, si la \lstinline{note} n'a pas été modifiée et qu'elle vaut toujours \lstinline{-1}, affichez "Note : pas encore attribuée" au lieu de "Note : \lstinline{note}" via la méthode \lstinline{toString()}. \\


Créez les classes filles avec les caractéristiques suivantes:\\
\lstinline{class Livre_Audio extends Livre}
\begin{itemize}
	\item un attribut \lstinline{privé String} nommé \lstinline{narrateur}

\end{itemize}
\lstinline{class Livre_Illustre extends Livre}
\begin{itemize}	\item un attribut \lstinline{privé String} nommé \lstinline{illustrateur}

\end{itemize}

\begin{conseil}
En Java, lors de la déclaration d'une classe, le mot clé \lstinline{extends} permet d'indiquer qu'il s'agit d'une classe fille de la classe indiquée. 

Le mot clé \lstinline{super} permet à la sous classe d'hériter d'éléments de la classe mère. \lstinline{super} peut être utilisé dans le constructeur de la sous-classe selon l'example suivant: \lstinline{super(attribut_mère_1, attribut_mère_2, attribut_mère_3, etc.);}. Ainsi, il n'est pas nécessaire de redéfinir tous les attributs d'une classe fille !

L'instruction \lstinline{super} doit toujours être la première instruction dans le constructeur d'une sous-classe.

Vous pouvez vous servir de \lstinline{\n} dans une chaîne de caractères pour effectuer un retour à la ligne lors de l'affichage.
\end{conseil}

\begin{solution}
	\lstinputlisting{Solutions/heritage-solution-1.java}
\end{solution}

\end{Exercice}

\begin{Exercice}[5 minutes] \lstinline{Méthode et héritage} \\

Maintenant que vous avez créé la classe mère et les classes filles correspondantes, vous pouvez créer un objet \lstinline{Livre} à l'aide du constructeur de la classe \lstinline{Livre_Audio} (et des arguments donnés lors de la création de l'objet).

En instanciant l'objet, vous pourriez utiliser les valeurs suivantes : titre: "Hamlet", auteur: "Shakespeare", année: "1609" et le narrateur "William.\\

Une fois l'objet créé, attribuez-lui une note à l'aide de la méthode \lstinline{setNote()} définie précédemment.\\ 

Finalement, utilisez la méthode \lstinline{System.out.println()} pour afficher les informations du livre.\\

La méthode étant définie dans la classe mère, elle n'a pas connaissance de la variable \lstinline{narrateur} définie dans la sous-classe. Redéfinissez la méthode dans la classe fille pour y inclure l'information sur le narrateur. \\

Faites pareil avec la classe \lstinline{Livre_Illustre} et son attribut \lstinline{Illustrateur}

\begin{conseil}
\textbf{Attention}, on vous demande de créer un objet \lstinline{Livre} et non pas \lstinline{Livre_Audio}.

Le mot-clé \lstinline{super} peut être utilisé dans la redéfinition d'une méthode selon l'exemple suivant: \lstinline{super.nom_de_la_methode();}. Le mot clé\lstinline{super} représente la classe parent, tout comme le mot clé \lstinline{this} représentait l'instance avec laquelle la méthode était appelée. 


L'instruction \lstinline{super} doit toujours être la première instruction dans le redéfinition d'une méthode dans une classe fille. 
\end{conseil}

\begin{solution}
	\lstinputlisting{Solutions/heritage-solution-2.java}
	\lstinputlisting{Solutions/heritage-solution-3.java}
\end{solution}

\end{Exercice}

\newpage
Lorsque toutes les étapes auront été effectuées, effectuez ce \lstinline{main} : \\

\lstinputlisting{ressources/heritage-main.java}


Vous devriez obtenir : \\


\lstinputlisting{ressources/heritage-main-prompt.java}

\newpage
\section{Polymorphisme - Java}

Les exercices de cette section sont une suite des exercices de la section 2 de la semaine passée.
Dans cette partie, vous serez amenés à créer 2 nouvelles sous-classes de la classe mère \lstinline{Combattant}. La première classe représentera un \lstinline{Soigneur}, qui, lorsqu'il ``attaquera'' un \lstinline{Combattant}, le soignera au lieu de le blesser. La deuxième classe représentera un combattant spécialisé dans l'attaque \lstinline{Attaquant}, qui aura la capacité d'attaquer un certain nombre de fois (ce nombre sera défini au moment où vous l'instancierez). Pensez à télécharger la dernière version de la classe \lstinline{Combattant} dans le dossier ressources.\\

Voici le squelette du code que vous trouverez également dans le dossier ressources du Moodle : \\

\lstinputlisting{ressources/combattant-squelette.java}

\begin{Exercice}[5 minutes] Sous-classe \lstinline{Soigneur} \\

Commencez par déclarer une nouvelle sous-classe \lstinline{Soigneur}. Cette sous-classe prendra un nouvel attribut \lstinline{private}, \lstinline{int}, nommé \lstinline{résurrection}, qui vaudra 1 lors de l'instanciation. \\

Déclarez le \lstinline{constructeur} de cette classe ainsi que les \lstinline{getter} et \lstinline{setter} permettant d'interagir avec ce nouvel attribut (\lstinline{résurrection}). \\

\begin{conseil}
Pensez à utiliser le constructeur de votre classe mère \lstinline{Combattant}
\end{conseil}

\begin{solution}
	\lstinputlisting{Solutions/soigneur.java}\end{solution}

\end{Exercice}

\begin{Exercice}[10 minutes] Méthode \lstinline{résurrection(Combattant other)} de la sous-classe \lstinline{Soigneur} \\

Commencez par déclarer une nouvelle méthode nommée \lstinline{résurrection(Combattant other)}. \\
Cette méthode permettra de faire revenir un \lstinline{Combattant} à la vie, mais le \lstinline{Soigneur} ne pourra le faire qu'une seule fois. \\

Commencez par contrôler que l'instance depuis laquelle la méthode est appelée soit toujours en vie. Si ce n'est pas le cas, indiquez : \lstinline{nom_instance} est mort et ne peut plus rien faire.  \\ 

Contrôlez ensuite que l'instance \lstinline{other} soit vraiment morte. Si ce n'est pas le cas, indiquez le via : \lstinline{nom_other} est toujours en vie. \\

Pour finir, contrôlez que l'attribut \lstinline{résurrection} de l'instance depuis laquelle la méthode est appelée est égale à 1. Si ce n'est pas le cas, indiquez : \lstinline{nom_instance} ne peut plus ressusciter personne.\\

Si tous ces éléments sont réunis, faites revenir le \lstinline{Combattant} \lstinline{other} à la vie en lui remettant 10 points de vie et en l'ajoutant à la liste \lstinline{instances} de la classe \lstinline{Combattant}. Pensez aussi à:\begin{itemize}
	\item Mettre l'attribut \lstinline{résurrection} de l'instance appelée à 0 afin de l'empêcher de réutiliser ce pouvoir,
	\item appeler la méthode \lstinline{checkHealth()}, et à indiquer : \lstinline{nom_other} est revenu à la vie !
\end{itemize}

\begin{conseil}
Utilisez un branchement conditionnel pour les contrôles. \\

Une nouvelle méthode nommée \lstinline{addInstances(Combattant other)} a été créée dans la classe \lstinline{Combattant}. Regardez à quoi elle sert et utilisez la. \\

Pour les indications en fonction des différentes conditions, imprimez simplement la phrase en question. \\
\end{conseil}

\begin{solution}
	\lstinputlisting{Solutions/resurrection.java}
\end{solution}

\end{Exercice}

\begin{Exercice}[10 minutes] Méthode \lstinline{attack} de la sous-classe \lstinline{Soigneur} \\

Réécrivez la méthode \lstinline{attack} de la sous-classe \lstinline{Soigneur} afin d'ajouter des points de vie à \lstinline{other} au lieu de lui en retirer. \\

Le seul argument nécessaire pour cette méthode sera le \lstinline{Combattant} \lstinline{other}. \\

Commencez par contrôler que le \lstinline{Soigneur} depuis lequel la méthode est appelée est encore en vie. Si ce n'est pas le cas, indiquez : \lstinline{nom_instance} est mort et ne peut plus rien faire.
Contrôlez ensuite si \lstinline{other} est toujours en vie. Si ce n'est pas le cas indiquez : \lstinline{nom_other} est déjà mort, ressuscitez le afin de pouvoir le soigner. Contrôlez également qu'il ait moins de 10 points de vie. Si ce n'est pas le cas, indiquez le via : \lstinline{nom_other} a déjà le maximum de points de vie. \\

Si toutes ces conditions sont réunies, ajoutez la valeur de l'attaque de l'instance qui appelle la méthode aux points de vie de \lstinline{other}, puis appelez la méthode de classe \lstinline{checkHealth()}.

\begin{conseil}
Pensez à utiliser du branchement conditionnel pour les contrôles. \\

Le nombre de points de vie à ajouter est simplement égal à l'attaque de l'instance depuis laquelle la méthode est appelée. Ajoutez la valeur de cet attribut \lstinline{attack} au \lstinline{Combattant} \lstinline{other}

\end{conseil}

\begin{solution}
	\lstinputlisting{Solutions/soigneur-attack.java}
\end{solution}

\end{Exercice}

\begin{Exercice}[5 minutes] Sous-classe \lstinline{Attaquant} \\

Commencez par déclarer une nouvelle sous-classe \lstinline{Attaquant}. Cette sous-classe prendra un nouvel attribut \lstinline{private}, \lstinline{int}, nommé \lstinline{multiplicateur}, qui sera passé en argument du \lstinline{constructeur} de la sous-classe. \\
Déclarez le \lstinline{constructeur} de cette classe ainsi que les \lstinline{getter} et \lstinline{setter} permettant d'interagir avec ce nouvel attribut \lstinline{multiplicateur}. \\

\begin{conseil}
Pensez à utiliser le \lstinline{constructeur} de votre classe mère \lstinline{Combattant}.
\end{conseil}

\begin{solution}
	\lstinputlisting{Solutions/attaquant.java}
\end{solution}

\end{Exercice}

\begin{Exercice}[10 minutes] Méthode \lstinline{attack} de la sous-classe \lstinline{Attaquant} \\

Réécrivez la méthode \lstinline{attack} de la sous-classe \lstinline{Attaquant} afin d'effectuer plusieurs attaques sur \lstinline{other} en fonction de l'attribut \lstinline{multiplicateur}. \\

Y'a t-il besoin de contrôler si l'instance depuis laquelle la méthode est appelée est encore en vie ? \\

Indiquez systématiquement le numéro de l'attaque, puis effectuez l'attaque. Répétez le procédé jusqu'à ce que le numéro de l'attaque soit égal à celui de \lstinline{multiplicateur_instance}.

\begin{conseil}
Aidez vous de la méthode \lstinline{attack} de la classe mère \lstinline{Combattant}.\\

Comment peut-on effectuer plusieurs fois une même séquence d'action en programmation ? \\

\end{conseil}

\begin{solution}
	\lstinputlisting{Solutions/attaquant-attack.java}
\end{solution}

\end{Exercice}

Si tout est correct, en utilisant ce \lstinline{main} : \\

\lstinputlisting{ressources/combattant-main.java} 

Vous devriez obtenir : \\

\lstinputlisting{Solutions/combattant-main-solution.java} 

\end{document}
\end{align*}
\end{document}